\hypertarget{sec:2}{\section{Zufallsvariablen}}
\subsection*{Zufallsvariable}
Sein $ (\Omega, \F, P)$ ein Wahrscheinlichkeitsraum. Also $\Omega$ ein
Grundraum, $\F \subseteq 2^\Omega$ die beobachtbaren Ereignisse und $P$ ein
Wahrscheinlichkeitsmass auf $\F$. Eine (reelwertige) Zufallsvariable auf
$\Omega$ ist eine messbare Funktion $X : \Omega \mapsto \R$. Das bedeutet, dass
die Menge $\{X \leq t\} = \{\omega : X (\omega) \leq t\}$ für jedes $t$ ein
beobachtbares Ereigniss sein muss.
\subsection*{Verteilungsfunktion}
Die Verteilungsfunktion von $X$ ist die Abbildung $F_X : \R \mapsto [0, 1]$:
\begin{align*}
  t \mapsto F_X (t) := P[X \leq t] := P[\{\omega : X (\omega) \leq t\}]
\end{align*}
und hat die Eigenschaften:
\begin{itemize}
  \item $F_X$ ist wachsend und rechtsstetig. Das bedeutet,
        dass $F_X (s) \leq F_X (t)$ für $s \leq t$ gilt und $F_X (u) \ra F_X (t)$
        für $u \ra t$ mit $u > t$
  \item $\lim_{t \ra - \infty} F_X (t) = 0$ und $\lim_{t \ra + \infty} F_X (t) = 1$
\end{itemize}
\subsection*{Dichtefunktion}
Das Analogon der Gewichtsfunktion im Diskreten Fall. Eine Zufallsvariable $X$
mit Verteilungsfunktion $F_X (t) = P[X \leq t]$ heisst (absolut) stetig mit
Dichte (funktion) $f_X : \R \mapsto [0, \infty)$, falls gilt:
\begin{align*}
  F_X (t) = \int_{-\infty}^t f_X (s) \; dx &  & \text{für alle } t \in \R
\end{align*}
und hat die Eigenschaften:
\begin{itemize}
  \item $f_X \geq 0$ und $f_X = 0$ ausserhalb von $\W (X)$.
  \item $\int_{-\infty}^\infty f_X (s) \; ds = 1$;
        das folgt aus $\lim_{t \ra + \infty} F_X (t) = 1$
\end{itemize}
\subsection*{Gleichverteilung}
Die Gleichverteilung auf dem Intervall $[a, b]$ ist ein Modell für die
Zufällige Wahl eines Punktes in $[a, b]$. Die zugehörige Zufallsvariable $X$
hat den Wertebereich $\W (X) = [a, b]$, sowie
\begin{align*}
  f_X (t) & =
  \begin{cases}
    \frac{1}{b-a} & \text{für } a \leq t \leq b \\
    0             & \text{sonst.}
  \end{cases} \\
  F_X (t) & =
  \begin{cases}
    0               & \text{für } t < a           \\
    \frac{t-a}{b-a} & \text{für } a \leq t \leq b \\
    1               & \text{für } t > b.
  \end{cases}
\end{align*}
wir schreiben kurz $X \sim U (a, b)$.
\begin{align*}
  E[X] = \frac{a + b}{2} &  & Var[X] = \frac{{(b - a)}^2}{12}
\end{align*}
\subsection*{Exponentialverteilung}
Die Exponentialverteilung mit Parameter $\lambda > 0$ ist das stetige Analogon
der Geometrischen Verteilung. Die zugehörige Zufallsvariable $X$ hat $\W (X) =
  [0, \infty)$, Dichte und Verteilungsfunktion:
\begin{align*}
  f_X (t) & =
  \begin{cases}
    \lambda \cdot e^{-\lambda t} & \text{für } t \geq 0 \\
    0                            & \text{für }t < 0
  \end{cases} \\
  F_X (t) & =
  \int_{-\infty}^t f_X (s) \; ds =
  \begin{cases}
    1 - e^{-\lambda t} & \text{für } t \geq 0 \\
    0                  & \text{für }t < 0
  \end{cases}
\end{align*}
wir schreiben kurz $X \sim Exp (\lambda)$. Weiter ist
die Funktion Gedächtsnislos, dh. $\cond{X > t + s}{X > s} = P[X > t]$.
\begin{align*}
  E[X] = \frac{1}{\lambda} &  & Var[X] = \frac{1}{\lambda^2}
\end{align*}
\subsection*{Normalverteilung}
Die Normalverteilung hat zwei Parameter: $\mu \in \R$ und $\sigma^2 > 0$. Die
zugehörige Zufallsvariable $X$ hat den Wertebereich $\W (X) = \R$ und die
Dichtefunktion:
\begin{align*}
  f_X (t) = \frac{1}{\sigma \sqrt{2 \pi}} e^{- \frac{ (t - \mu)^2}{2 \sigma^2}}
   &  & \text{für } t \in \R
\end{align*}
welche symmetrisch um $\mu$ ist. Wir schreiben kurz: $X \sim \Normalverteilt$.
\subsection*{Standard Normalverteilung}
Wichtige Normalverteilung mit $\Standardnormalverteilt$. Weder für die
zugehörige Dichte $\vp (t)$ noch Verteilungsfunktion $\Phi (t)$ gibt es
geschlossene Ausdrücke, aber das Integral
\begin{align*}
  \Phi (t) = \int_{-\infty}^t \vp (s) \; ds =
  \frac{1}{\sqrt{2\pi}} \int_{-\infty}^t e^{-\frac{1}{2} s^2} \; ds
\end{align*}
ist tabelliert. Ist $X \sim \Normalverteilt$, so ist
$\frac{X - \mu}{\sigma} \sim \Standardnormalverteilt$, also:
\begin{align*}
  F_X (t) = P[X \leq t] = P \left[ \frac{X-\mu}{\sigma} \leq \frac{t - \mu}{\sigma} \right] = \Phi \left  ( \frac{t - \mu}{\sigma} \right)
\end{align*}
deshalb genügt es $\Phi$ zu tabellieren.
\begin{align*}
  \Phi (-z) = 1 - \Phi (z)
\end{align*}
\subsection*{Normalapproximation}
Wenn $S_n \sim Bin (n, p)$ dann
\begin{align*}
  S_n \sim_{approx} N (np, np (1-p))
\end{align*}
\subsection*{Erwartungswert}
Ist $X$ stetig mit Dichte $f_X (x)$, so ist der Erwartungswert:
\begin{align*}
  E[X] = \int_{-\infty}^\infty x \cdot f_X (x) \; dx
\end{align*}
sofern das Integral absolut konvergiert. Ist das Integral nicht
absolut konvergent, so existiert der Erwartungswert nicht.
\subsection*{Erwartungswert einer Funktion}
Sei $X$ eine Zufallsvariable und $Y = g (X)$ eine weitere Zufallsvariable. Ist
$X$ stetig mit Dichte $f_X$, so ist
\begin{align*}
  E[Y] = E[g (X)] = \int_{-\infty}^\infty g (x) \cdot f_X (x) \; dx
\end{align*}
\subsection*{Momente \& Absolute Momente}
Sei $X$ eine Zufallsvariable und $p \in \R^+$. Wir definieren:
\begin{itemize}
  \item $p$-te absolute Moment von $X$: $M_p := E[\abs{X}^p]$
  \item falls $M_n < \infty$ für ein $n$, dann ist das $n$-te (rohe) Moment von $X$
        durch $m_n := E[X^n]$ definiert.
  \item Das $n$-te zentralisierte Moment von $X$ ist durch $\mu_n := E[ (X - E[X])^n]$
        definiert.
\end{itemize}
Es gilt weiter, dass $M_n < \infty$ für $n \in \N \implies \abs{m_m} \leq M_n$.
\begin{align*}
  M_p                         & = \int_{-\infty}^\infty \abs{x}^p f_X (x) \; dx \\
  m_n                         & = \int_{-\infty}^\infty x^n f_X (x) \; dx       \\
  p \leq q \land M_q < \infty & \implies M_p < \infty
\end{align*}
\subsection*{Gemeinsame Verteilung/Dichte}
Die Gemeinsame Verteilungsfunktion von Zufallsvariablen $\zufallsvariablen$ ist
die Abbildung $F: \R^n \mapsto [0, 1]$ mit:
\begin{align*}
  F (x_1, \dots, x_n) & := P[X_1 \leq x_1, \dots, X_n \leq x_n]                                                 \\
                      & = \int_{-\infty}^{x_1} \dots \int_{-\infty}^{x_n} f (t_1, \dots, t_n) \; dt_n \dots t_1
\end{align*}
dann heisst $f (x_1, \dots, x_n)$ die gemeinsame Dichte, welche folgende
Eigenschaften hat:
\begin{itemize}
  \item $f (x_1, \dots, x_n) \geq 0$ und $= 0$ ausserhalb von $\W (\zufallsvariablen)$.
  \item $\int_{-\infty}^\infty \dots \int_{-\infty}^\infty f (t_1, \dots, t_n) \; dt_n \dots t_1 = 1$
  \item $P[ (\zufallsvariablen) \in A] = \int_{ (x_1, \dots, x_n) \in A} f (t_1, \dots, t_n) \; dt_n \dots t_1$ für $A \subseteq \R^n$
\end{itemize}
\subsection*{Randverteilung}
Haben $X, Y$ die gemeinsame Verteilungsfunktion $F$, dann ist:
\begin{align*}
  F_X (x) & = P[X \leq x] = P[X \leq x, Y < \infty] = \lim_{y \ra \infty} F (x, y) \\
  f_X (x) & = \int_{-\infty}^\infty f (x, y) \; dy
\end{align*}
\subsection*{Unabhängigkeit}
Die Zufallsvariablen $\zufallsvariablen$ heissen unabhängig, falls gilt
(äquivalent):
\begin{align*}
  F (x_1, \dots, x_n) = F_{X_1} (x_1) \cdot \hdots \cdot F_{X_n} (X_n) \\
  f (x_1, \dots, x_n) = f_{X_1} (x_1) \cdot \hdots \cdot f_{X_n} (X_n)
\end{align*}
für alle $x_1, \dots, x_n$.
\subsection*{Bedingte Verteilungen}
Es gilt:
\begin{align*}
  f_{X_1 \; | \; X_2} (x_1 \; | \; x_2) & = \frac{f_{X_1,  X_2} (x_1,  x_2)}{f_{X_2} (x_2)}              \\
  \cond{Y > t}{Y < a}                   & = \frac{P[t < Y < a]}{P[Y < a]}                                \\
  E[X_1 \; | \; X_2]                    & = \int x_1 \cdot f_{x_1 \; | \; x_2} (x_1 \; | \; x_2) \; dx_1
\end{align*}
\subsection*{Summen von Zufallsvariablen}
Sei $Z = X + Y$ eine Zufallsvariable mit:
\begin{align*}
  F_Z (z) & = P[Z \leq z] = P[X + Y \leq z]                                     \\
          & = \int_{-\infty}^\infty \int_{-\infty}^{z - x} f (x, y )\; dy \, dx \\
  f_Z (z) & = \int_{-\infty}^\infty f (z - y, y) \; dy
\end{align*}
\subsection*{Transformationen}
Sei $X$ eine Zufallsvariable mit Verteilung und Dichte. Sei $g: \R \mapsto \R$
eine messbare Funktion. Betrachte nun $Y = g (X)$, wir suchen Verteilung und
Dichte von $Y$:
\begin{align*}
  F_Y (t) & = P[Y \leq t] = P[g (Y) \leq t] = \int_{A_g} f_X (s) \; ds \\
  A_g     & := \{s \in \R \; | \; g (s) \leq t\}
\end{align*}
Wobei man die Dichte durch ableiten der Verteilung erhält.
\subsection*{Anwendung von Transformationen}
Sei $F$ eine stetige und streng monoton wachsende Verteilungsfunktion mit
Unkehrfunktion $F^{-1}$. Ist $X \sim \mathcal{U} (0, 1)$ und $Y = F^{-1} (X)$,
so hat $Y$ gerade die Verteilungsfunktion $F$:
\begin{align*}
  F_Y (t) & = P[Y \leq t] = P[F^{-1} (X) \leq t] \\
          & = P[X \leq F (t)] = F (t)
\end{align*}
Mit der Substitution
\begin{align*}
  \phi(X) &= Y\\
  X &= \phi^{-1}(y)\\
  f_Y(y) &= f_X(\phi^{-1}(y))|\text{det }J_{\phi^{-1}}(y)|\\
\end{align*}
\BoxStart{}
Sei $X$ eine Zufallsvariable mit Dichte $f_X(x), x \in \R$ und sei $Y = e^X$. Was ist die Dichte $f_Y(y), y > 0$ der Zufallsvariable $Y$?
\begin{align*}
  Y &= e^X\\
  X &= \ln Y\\
  \frac{dx}{dy} &= \frac{1}{y}\\
  f_Y(y) &= f_X(\ln y) \cdot \frac{1}{y}\\
\end{align*}
\BoxEnd{}
\subsection*{Markov Ungleichung}
Sei $X$ eine Zufallsvariable und ferner $g : \W (X) \mapsto [0, \infty)$ eine
wachsende Funktion. Für jedes $c \in \R$ mit $g (c) > 0$ git dann:
\begin{align*}
  P[X \geq c] \leq \frac{E[g (X)]}{g (c)}
\end{align*}
\subsection*{Chebyshev-Ungleichung}
Sei $Y$ eine Zufallsvariable mit endlicher Varianz. Für jedes $b > 0$ git dann:
\begin{align*}
  P[\abs{Y - E[Y]} \geq b] \leq \frac{Var[Y]}{b^2}
\end{align*}
\subsection*{Schwaches Gesetz der grossen Zahlen}
Sei $X_1, X_2, \dots$ eine Folge von unabhängigen Zufallsvariablen, die alle
den gleichen Erwartungswert $E[X_i] = \mu$ und die gleiche Varianz $Var[X_i] =
  \sigma^2$ haben. Sei
\begin{align*}
  \overline{X}_n = \frac{1}{n} S_n = \frac{1}{n} \Sn X_i
\end{align*}
Dann konvergiert $\overline{X}_n$ für $n \ra \infty$ in Wahrscheinlichkeit/
stochastisch gegen $\mu = E[X_i]$, d.h.:
\begin{align*}
  P \left[ \abs{\overline{X}_n - \mu} > \varepsilon \right] \underset{n \ra \infty}{\longrightarrow} 0
   &  & \text{für jedes } \varepsilon > 0
\end{align*}
(Statt unabhängig genügt auch $Cov (X_i, X_k) = 0$ für $i \neq k$)
\subsection*{Starkes Gesetz der grossen Zahlen}
Sei $X_1, X_2, \dots$ eine Folge von unabhängigen Zufallsvariablen, die alle
dieselbe Verteilung haben, und ihr Erwartungswert $\mu = E[X_i]$ sei endlich.
Für:
\begin{align*}
  \overline{X}_n = \frac{1}{n} S_n = \frac{1}{n} \Sn X_i
\end{align*}
gilt dann
\begin{align*}
  \overline{X}_n \underset{n \ra \infty}{\longrightarrow} \mu &  & \text{P-fastsicher}
\end{align*}
d.h.:
\begin{align*}
  P \left[ \left\{ \omega \in \Omega : \overline{X}_n (\omega) \underset{n \ra \infty}{\longrightarrow} \mu \right\} \right] = 1
\end{align*}
\subsection*{i.i.d. / u.i.v.}
Independent identically distributed
\subsection*{Zentraler Grenzwertsatz}
Sei $X_1, X_2, \dots$ eine Folge von i.i.d. Zufallsvariablen mit $E[X_i] = \mu$
und $Var[X_i] = \sigma^2$. Für die Summe $S_n = \Sn X_i$ gilt dann:
\begin{align*}
  \lim_{n \ra \infty} P \left[ \frac{S_n - n \cdot \mu}{\sigma \sqrt{n}} \leq x \right] = \Phi (x)
   &  & \text{für alle } x \in \R
\end{align*}
wobei $\Phi$ die Verteilungsfunktion von $\Standardnormalverteilt$ ist.