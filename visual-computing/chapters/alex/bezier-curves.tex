\section{Bézier/Hermite Curves}
\greenbf{Spline desired properties:} \\
Interpolation: \graytext{Spline passes exactly through data points}\\
Continuity: \graytext{in $C^2$}\\
Locality: \graytext{moving one point does nto affect whole curve} \\
$\implies$ impossible to have all at once
Cubic polynomials, interpolate + 1st derivate is given tangent. Interpolates, not $C^2$-continuous, local
\greenbf{Maps:} $\mathbb{R}^1 \to \mathbb{R}^3$ : $x(u) = (x(u), y(u), z(u))^T$, $\mathbb{R}^2 \to \mathbb{R}^3$ : $x(u,v) = (x(u,v), y(u,v), z(u,v))^T$
\greenbf{Bezier Curves:}
${x}(t) = b_0 B_0^n(t) + \dots + b_n B_n^n(t)$ where $b_0 ...b_n$ are the control points. 
\\
$n = 3 $ : ${x}(t) = b_0(1-t)^3 + 3b_1t(1-t)^2 + 3b_2t^2(1-t) + b_3t^3$. 
\\
\textbf{Derivative:} $\frac{d}{dt} b^n(t) = n \sum_{i=0}^{n} \left( B_{i-1}^{n-1}(t) - B_i^{n-1}(t) \right) b_i$, which is a Bezier curve with degree $n - 1$\\
\textbf{Properties:} design property: control points give rough sketch, endpoint interpolation, variation diminishing property: intersection of straight line with curve $<=$ \# control points.\\
\textbf{Disadvantages:} global support of basis functions (changing one control point changes entire curve), inserting control points expensive, lack of continuity between different segments\\
\greenbf{Bernstein Polynomial of degree n:}
$B_i^n(t) = \binom{n}{i} t^i (1 - t)^{n-i} \text{ for } 0 \leq i \leq n$ zero else. 
\\
Global support, positive definite, partition of unity, different degrees. 
\\
Derivative: $\frac{d}{dt} B_i^n(t) = n \left( B_{i-1}^{n-1}(t) - B_i^{n-1}(t) \right)$


\greenbf{Binomial coefficient:}\\
$\binom{n}{i} = \frac{n!}{i!(n-i)!} \text{for } 0 \leq i \leq n$ zero else.

\greenbf{DeCasteljau Algorithm:}
Recursive method for computing a point on a bezier curve using a systolic array in O($n^2$):
Given \( n+1 \) control points \( b_0, b_1, \ldots, b_n \) the recursion is defined as follows:
\begin{align*}
b_i^r(t) &= (1 - t) b_i^{r-1}(t) + t b_{i+1}^{r-1}(t) \\
b_i^0(t) &= b_i \\
&\text{for } r = 1, \ldots, n \text{ and } i = 0, \ldots, n-r
\end{align*}
Intuition: Corner cutting until only one line remains whose intersection with the curve is the result. 

\greenbf{Forward difference operator $\Delta:$} $\Delta b_j = b_{j+1} - b_j$
Bezier curve derivative with $\Delta$: \\ $\frac{d}{dt}b^n(t) = n\sum_{j=0}^{n-1} \Delta b_j \cdot B_i^{n-1}$


\greenbf{Recursive $\Delta^r$:} \\
\textbf{recursive:} $\Delta^r b_j = \Delta^{r-1}b_{j+1} - \Delta^{r-1}b_j$\\
\textbf{non-recursive:} $\Delta^r b_i = \sum_{j=0}^r \binom{r}{j} (-1)^{r-j} b_{j+i}$\\
\textbf{Higher order derivative of Bezier curve:} $\frac{d^r}{dt^r} b^n(t) = \frac{n!}{(n-r)!} \sum_{j=0}^{n-r} \Delta^r b_j B_j^{n-r}(t)$\\
\greenbf{Piecewise Bezier Curves / Splines:}\\
Knots: $u_0 < ... <u_L$, \\
Intervals: $[u_i, u_{i+1}]$, \\
local parameter: $t = \frac{u - u_i}{u_{i+1} - u_i} = \frac{u - u_i}{\Delta_i}$. \\
Segment $s(u) = s_i(t)$, \\
a Bezier curve that is a function of the local parameter t.
$\frac{ds(u)}{du} = \frac{ds_i(t)}{dt}\frac{dt}{du} = \frac{1}{\Delta_i}\frac{ds_i(t)}{d_t}$. \\
\textbf{Enforce Continuity:} Curve in $[u_0, u_2]$ decomposed to bezier segments $b_0, ..., b_n$ in $[u_0, u_1]$ and $b_n, ..., b_{2n}$ in $[u_0, u_1]$, $C^r-Continous$ if $b_{n+1} = b_{n-i}^i(t)$ for $i = 0,...,r$ and $t = \frac{u - u_0}{u_1 - u_0 }$. $C^1-Continuity$: Control points $b_n-1, b_n, b_{n+1}$ are colinear. 

\greenbf{Matrix form:} $x(t) = \sum_{i=0}^nc_iC_i(t)$. Basis transform into monomial representation with $M=\{m_{ij}\}$:
$
\begin{bmatrix}
C_0(t) \\
\vdots \\
C_n(t)
\end{bmatrix}
=
\begin{bmatrix}
m_{00} & \cdots & m_{0n} \\
\vdots & \ddots & \vdots \\
m_{n0} & \cdots & m_{nn}
\end{bmatrix}
\begin{bmatrix}
t^0 \\
\vdots \\
t^n
\end{bmatrix}
$
\\
For Bernstein: $m_{ij} = (-1)^{j-i} \binom{n}{j} \binom{j}{i}$
\\
\greenbf{Spline interpolation:} Interpolate a set of points $p_0, ..., p_n$ using basis functions. For monomials as basis: $p_i = x(t_i) = \sum_{j=0}^{n} a_j(t_i)^j, \quad i \in [0,n]$. Resulting in Vandermonde matrix (ill-conditioned): 
$
\begin{bmatrix}
1 & t_0 & \cdots & t_0^n \\
\vdots & \vdots & \ddots & \vdots \\
1 & t_n & \cdots & t_n^n
\end{bmatrix}
\begin{bmatrix}
a_0 \\
\vdots \\
a_n
\end{bmatrix}
=
\begin{bmatrix}
p_0 \\
\vdots \\
p_n
\end{bmatrix}
$

\greenbf{Blossoming:} Generalisation of deCasteljau. TODO
