\section{Beispiele Wahrscheinlichkeit}%
\label{sec:beispiele_wahrscheinlichkeit}

\begin{example}
	$T \sim \expd (\lambda)$, $T' = c \cdot T^2 = h(T)$.
	\begin{enumerate}
		\item Fixiere $\phi$ (messbar, beschränkt). $\phi(T') = \sigma (T)$
		\item \begin{equation*}
				\E [\phi (T')] = \int_{- \infty}^{\infty} \sigma(y) \cdot f_T (y) \dif y = \int_{0}^{\infty}
				\underbrace{\sigma(y)}_{\phi(h(y'))} \lambda e^{-\lambda y} \dif y = \textcolor{lightblue}{(*)}
		\end{equation*}
		\item $z = h(y) = c \cdot y^2$, $\od{z}{y} = 2 cy \Leftrightarrow \dif y = \frac{1}{2cy} \dif z = \frac{1}{2
			\sqrt{cz}} \dif z $
		\item \begin{align*}
				\textcolor{lightblue}{(*)} &= \int_{0}^{\infty} \phi (z) \lambda e^{-\lambda \sqrt{z/c}} \frac{1}{2
					\sqrt{cz}} \dif z \\
													&= \int_{- \infty}^{\infty} \phi (z) \underbrace{\indf[{[0, + \infty[}] (z) 
													\lambda e^{-\lambda \sqrt{z/c}} \frac{1}{2 \sqrt{cz}}}_{\tilde{f}(z) = f_{T'} (z)}\dif z
		\end{align*}
	\end{enumerate}
	\tcblower
	Berechne $\E [T']$. \hspace{10mm} Aufgrund der  Linearität des $\E$ gilt, $\E[T'] = \E \left[ c T^2 \right] = c \cdot
	\E \left[ T^2 \right]$
	\begin{align*}
		\E \left[ T^2 \right] &= \int_{0}^{\infty} x^2 \lambda e^{-\lambda x} \dif x = \left[ -x^2 e^{-\lambda x}
		\right]_0^\infty + \underbrace{\int_{0}^{\infty} 2x e^{-\lambda x} \dif x}_{\frac{2}{\lambda} \E[T]} = \frac{2}{\lambda^2} \\
			\Longrightarrow & \E\left[T'\right] = \frac{2c}{\lambda^2}
	\end{align*}
\end{example}
\begin{example}
	$U \sim \unif \left( \left[ - \frac{\pi}{2} , \frac{\pi}{2} \right] \right)$, $Y = \sin (U)$, $ h(x) = \sin (x)$
	\begin{enumerate}
		\item Fixiere $\phi$ (messbar, beschränkt).
		\item \begin{equation*}
			\E [\phi (Y)] = \int_{- \infty}^{\infty} \phi(h(y)) \indf[{[-\pi/2 , \pi/2]}] (y) \frac{1}{\pi} \dif y
			= \int_{- \pi/2}^{\pi/2} \phi (\sin (y)) \frac{1}{\pi} \dif y
		\end{equation*}
	\item $ z = \sin (y) \Rightarrow y = \arcsin (z)$, $\od{z}{y} = \cos (y) \Rightarrow \dif y = \frac{1}{\sqrt{1 - z^2}} \dif z $
		\item \begin{equation*}
				\int_{-1}^{1} \phi (z) \frac{1}{\pi} \frac{1}{\sqrt{1 -z^2}} \dif z ) = \int_{- \infty}^{\infty} \phi (z)
				\underbrace{\frac{1}{\pi} \frac{1}{\sqrt{1-z^2}} \indf[{[-1,1]}] {z}}_{f_Y} \dif z
		\end{equation*}
	\end{enumerate}
\end{example}

\begin{example}
	Sei $S \sim \geom(p)$ mit $p=3/4$. Berechne $\E \left[ S^2 \right]$ und $\sigma_S^2$.
	\tcblower
	\begin{align*}
		\E \left[ S^2 \right] &= \sum_{j=0}^{\infty} j^2 \pr[S = j] = p \cdot \sum_{j=0}^{\infty} j^2 (1-p)^{j-1} = p
		\cdot (f'' (p) + f'(p))\\
									 &= \frac{2}{p} - \frac{1}{p} = \frac{2 - p}{p^2} 
	\end{align*}
	Wobei wir die Funktion $f : ]0,1] \rightarrow [0,\infty[$ verwenden
	\begin{align*}
		f(p) &\coloneqq \sum_{j=0}^{\infty} (1-p)^{j+1} = \frac{1-p}{p} \\
		- \frac{1}{p^2} = f'(p) &= \sum_{j=0}^{\infty} \dv{p} (1-p)^{j+1} = - \sum_{j=0}^{\infty} (j+1)(1-p)^j \\
										&= -\sum_{j=0}^{\infty} j(1-p)^{j-1}\\
		\frac{2}{p^3} = f''(p) &= \sum_{j=0}^{\infty} dv[2]{p} (1- p)^{j+1} = \sum_{j=0}^{\infty} j(j+1)(1-p)^{j-1}  
	\end{align*}
	Die Varianz berechnet sich durch
	\begin{equation*}
		\sigma_S^2 = \E \left[ S^2 \right] - \E[S]^2 = \frac{2-p}{p^2} - \frac{1}{p^2} = \frac{4}{9} 
	\end{equation*}
\end{example}
\begin{example}
	Berechne die Laufzeit dieses Alg.
	\begin{algorithm}[H]
		\DontPrintSemicolon
		$i = 1$\;
		\While{$X_i = X_{i+1} = 1$}{
			$i=i+2$\;
		}
	\end{algorithm}
	\tcblower
	Sei
	\begin{gather*}
		\{T \geq j\} = \{\text{Loop j-mal durchlaufen}\} = \bigcap_{i=1}^{2j} \{X_i = 1\}\\
		\pr[T \geq j] = \pr \left[ \bigcap_{i=1}^{2j} \{X_i = 1\} \right] = \prod_{i=1}^{2j} \pr[X_i = 1] = \left(
		\frac{1}{2} \right)^{2j} = \left( \frac{1}{4} \right)^j
	\end{gather*}
	Aus Tailsum folgt:
	\begin{align*}
		\E[T] &= \sum_{i=j}^{\infty} \pr [T \geq j] = \left( \sum_{i=j}^{\infty} \frac{1}{4^j} \right) - 1 =
		\frac{1}{1-\frac{1}{4}} -1 = \frac{1}{3}
	\end{align*}
\end{example}
\begin{example}
	\begin{center}
		\begin{minipage}{0.65\linewidth}
			Die gemeinsame Dichte $f(x,y)$ zweier stetiger ZV $X,Y$ ist in $Q$ konstant und verschwindet ausserhalb.\\
			Bestimme die gemeinsame Dichte von $X,Y$ und die Randdichten $f_X$ und $f_Y$.
		\end{minipage}
		\begin{minipage}{0.3\linewidth}
			\begin{center}
				\begin{tikzpicture}[]
					\begin{axis}[
							height=40mm,
							width=40mm,
							axis lines=middle,
							xmin=-1.25,
							xmax=1.25,
							ymax=1.25,
							ymin=-1.25,
							xlabel=$x$,
							ylabel=$y$,
							clip=false,
							axis on top=true,
						]
						\path[draw=lightblue, thick, fill=lightblue!20] (1,0) -- (0,1) -- (-1,0) -- (0,-1) -- cycle;
						\node[text=lightblue] at (0.25,0.25) {$Q$};
					\end{axis}
				\end{tikzpicture}
			\end{center}
		\end{minipage}
	\end{center}
	\tcblower
	Die Fläche von $Q$ ist $4 \cdot \frac{1 \cdot 1}{2} = 2$.
	\begin{equation*}
		f(x,y) =
		\begin{cases}
			c & \text{falls} ~(x,y) \in Q\\
			0 & \text{sonst}
		\end{cases}
		=
		\begin{cases}
			\frac{1}{2} & \text{falls} ~(x,y) \in Q\\
			0 & \text{sonst}
		\end{cases}
	\end{equation*}
	Für gemeinsame Dichten gilt $\iint_{-\infty}^\infty f(x,y) \dif x \dif y = 1$
	\begin{gather*}
		\iint_{-\infty}^\infty f(x,y) \dif x \dif y = c \iint_{-\infty}^\infty 1 \dif x \dif y = c \cdot \text{Fläche}(Q)
		\\
		c = \frac{1}{\text{Fläche}(Q)} = \frac{1}{2} 
	\end{gather*}
	Für $f_X$ sind zwei Fälle zu unterscheiden (und analog für $f_Y$):
	\begin{itemize}
		\item $-1 \leq x \leq 0$: $f_X(x) = \int_{-\infty}^{\infty} f(x,y) \dif y = \int_{-1-x}^{1+x} \frac{1}{2} \dif y = 1 + x$
		\item $0 \leq x \leq 1$: $f_X(x) = \int_{-\infty}^{\infty} f(x,y) \dif y = \int_{-1+x}^{1-x} \frac{1}{2} \dif y = 1 - x$
	\end{itemize}
	\begin{equation*}
		f_X(x) =
		\begin{cases}
			1 + x & \text{falls} ~-1 \leq x \leq 0\\
			1 - x & \text{falls} ~0 \leq x \leq 1\\
			0 & \text{sonst}
		\end{cases}
		f_Y(y) =
		\begin{cases}
			1 + y & \text{falls} ~-1 \leq y \leq 0\\
			1 - y & \text{falls} ~0 \leq y \leq 1\\
			0 & \text{sonst}
		\end{cases}
	\end{equation*}
\end{example}

\section*{Beispiele Statistik}%
\label{sec:beispiele_statistik}

\begin{example}
	Sei $T_2^{(n)} = \max (X_{1}, \ldots, X_{n})$ ein Schätzer und $X_{1}, \ldots, X_{n}$ \iid mit $X_i \sim \unif
	([\theta - 1, \theta])$. Berechne $\E [T_2^{(n)}]$.
	\tcblower
	Definiere $Y_i = X_i -(\theta - 1)$. $Y_{1}, \ldots, Y_{n}$ sin \iid und $\unif([0,1])$ verteilt.
	\begin{gather*}
		Y^{(n)} \coloneqq \max \{Y_1 , \ldots , Y_n\} = \max \{X_{1}, \ldots, X_{n}\} - (\theta - 1 ) = T_2^{(n)} -
		(\theta - 1)\\
		F_{Y^{(n)}}(a) = \pr_\theta \left[ Y^{(n)} \leq a \right] = \prod_{i=1}^n \pr_\theta [Y_i \leq a] = \pr_\theta
		[Y_1 \leq a]^n = 
		\begin{cases}
			0 & a < 0\\
			a^n & a \in [0,1]\\
			1 & a > 1
		\end{cases}\\
		f_{Y^{(n)}}(a) = na^{n - 1} \indv_{a \in [0,1]}\\[10pt]
		\E_\theta \left[ Y^{(n)} \right] = \int_{0}^{1} a \cdot na^{n-1} \dif a = n \left[ \frac{1}{n+1} a^{n+1}
		\right]_{a=0}^{a=1} = 1 - \frac{1}{n+1} \\
		\E_\theta \left[ T_2^{(n)} \right] = \E_\theta \left[ Y^{(n)} \right] + (\theta - 1) = \theta - \frac{1}{n+1} 
	\end{gather*}
\end{example}
\begin{example}
	Sei $\Theta = [\frac{1}{2} ,1]$. Betrachte Modellfamilie $(\pr_\theta)_{\theta \in \Theta}$ wobei $X_{1}, \ldots,
	X_{n}$ unter $\pr_\theta$ \iid mit $X_1 \sim \geom(\theta)$. Der ML-Schätzer ist gegeben durch $T_{\text{ML}} = n/
	\sum_{i=1}^{n} X_i$.\\
	Bestimme ein approximatives Konfidenzintervall für $\theta$ mit Niveau $95\%$.\\
	$\forall \theta \in [\frac{1}{2} ,1]: \frac{\sqrt{1-\theta}}{\theta}  \leq \sqrt{2}$.
	\tcblower
	Verwende ZGWS um aus Schätzer eine $\normd(0,1)$ ZV zu machen.
	\begin{gather*}
		\pr_\theta \left[ -1.96 \leq \frac{\sum_{i=1}^{n} X_i - m/\theta}{\sqrt{2n}} \leq 1.96  \right] \\ 
		\geq \pr_\theta \left[ -1.96 \leq \frac{\sum_{i=1}^{n} X_i - m/\theta}{\sqrt{n(1-\theta) / \theta^2}} \leq 1.96  \right]
		\geq 0.95\\[10pt]
		-1.96 \leq \frac{\sum_{i=1}^{n} X_i - m/\theta}{\sqrt{2n}} \leq 1.96\\
		\Longleftrightarrow
		\frac{1}{(T_{\text{ML}})^{-1} + \frac{1.96 \sqrt{2}}{\sqrt{n}}} \leq \theta \leq \frac{1}{(T_{\text{ML}})^{-1} - \frac{1.96 \sqrt{2}}{\sqrt{n}}}
	\end{gather*}
	Daraus ergibt sich ein approximatives $95\%$-Konf.I für $\theta$
	\begin{equation*}
		\left[ \frac{1}{(T_{\text{ML}})^{-1} + \frac{1.96 \sqrt{2}}{\sqrt{n}}}, \frac{1}{(T_{\text{ML}})^{-1} - \frac{1.96 \sqrt{2}}{\sqrt{n}}} \right]
	\end{equation*}
\end{example}
\begin{example}
	Seien $X_{1}, \ldots, X_{12}$ unabhängig und je $\normd(\mu , \sigma^2)$-verteilt unter $\pr_\theta$, wobei $\theta =
	\mu$ ein unbekannter Parameter ist.$\sigma = 0.0499$
	Wir testen die Hypothese $H_0 : \mu = \mu_0 = 1.0085$ gegen $H_A : \mu \neq \mu_0$.
	\begin{equation*}
		\begin{array}{|cccccc|}
			\hline x_1 & x_2 & x_3 & x_4 & x_5 & x_6\\
			1.00781 & 1.00646 & 1.00801 & 1.00833 & 1.00738 & 1.00687\\\hline
			x_7 & x_8 & x_9 & x_{10} & x_{11} & x_{12}\\
			1.00783 & 1.00936 & 1.00564 & 1.00543 & 1.00794 & 1.01060\\\hline
		\end{array}
	\end{equation*}
	\begin{itemize}
		\item Wähle $K \coloneqq ]-\infty,-c_{\neq}[ \cup ]c_{\neq},\infty[$ als Verwerfungsbereich für ein zu bestimmendes
			$c_{\neq} \geq 0$. Teste $H_0$ gegen $H_A$ für das Signifikantsniveau $5\%$.
		\item Berechne die Macht des Tests an der Stelle $\mu = 1.008$.
	\end{itemize}
	\tcblower
	\begin{equation*}
		T \coloneqq \frac{\sum_{i=1}^{12} X_i - 12 \mu_0}{\sqrt{12} \sigma} 
	\end{equation*}
	Verwerfe $H_0$ falls $ |T| > c_{\neq}$.
	\begin{align*}
		\alpha &= \pr_{\mu_0} [T \in K] = \pr_{\mu_0} [T \not\in [-c_{\neq}, c_{\neq}]] = \pr_{\mu_0} [T < c_{\mu}] +
		\pr_{\mu_0} [T > c_{\neq}]\\
				 &= \Phi (-c_{\neq}) + 1 - \Phi (c_{\neq}) = 2 - 2 \Phi (c_{\neq})
	\end{align*}
	Mit $c_{\neq} = \Phi^{-1} (0.975) = 1.96$. Berechne mit realisierten Werten:
	\begin{equation*}
		T(\omega) = \frac{\sum_{i=1}^{12} x_i - 12 \mu_0}{\sigma \sqrt{12}} = \frac{12.09 - 12 \cdot 1.0085}{0.0499 \cdot
		\sqrt{12}} = - 0.00598
	\end{equation*}
	Die Macht des Tests an der Stelle $\mu = 1.008$ ist
	\begin{tikzpicture}[overlay]
		\node[rotate=75, font=\small] at (-5.2,-2) {$T (\omega)$ mit $n \cdot (\mu - \mu)$ erweitern};
	\end{tikzpicture}
	\begin{gather*}
		\pr_\mu [T \in K] = \pr_\mu [T \not\in [-c_{\neq},c_{\neq}]] \\
		= \pr_\mu \left[ \frac{\sum_{i=1}^{12} X_i - 12 \mu}{\sigma \sqrt{12}} < -c_{\neq} + \frac{\sqrt{12}}{\sigma}
		(\mu_0 - \mu) \right] \\
		+ \pr_\mu \left[ \frac{\sum_{i=1}^{12} X_i - 12 \mu}{\sigma \sqrt{12}} > c_{\neq} +
		\frac{\sqrt{12}}{\sigma} (\mu_0 - \mu) \right]\\
		= \Phi \left( -c_{\neq} +  \frac{\sqrt{12}}{\sigma} (\mu_0 - \mu)\right) + 1 -\Phi \left( -c_{\neq} +
		\frac{\sqrt{12}}{\sigma} (\mu_0 - \mu)\right)\\
		= \Phi (-1.93) + 1 - \Phi (1.99) = 0.0501
	\end{gather*}
\end{example}
\begin{example}
	Eine Klimaanlage schafft es, die Raumtemperatur bis auf eine Standardabweichung $\sigma$ von $0.5$ Grad Celsius
	konstant zu halten. Die angestrebte Raumtemperatur beträgt $20.00$ Grad Celsius. Die folgenden Temperaturen wurden gemessen:
	(Die Temperaturen sind \iid)
	\begin{equation*}
		\begin{array}{ccccccccc}
			x_1 & x_2 & x_3 & x_4 & x_5 & x_6 & x_7 & x_8 & x_9 \\
			20.71 & 19.76 & 20.56 & 21.39 & 21.00 & 19.67 & 20.92 & 20.31 & 20.39 \\
			x_{10} & & & & & & & & \\
			20.72 & & & & & & & & \\
		\end{array}
	\end{equation*}
	\begin{itemize}
		\item Führe einen geeigneten Test auf dem $5\%$-Niveau durch, um zu beurteilen, ob die Klimaanlage wirklich auf
			den Sollwert von $20.00$ Grad geeicht ist.
		\item Berechne den realisierten P-Wert.
	\end{itemize}
	\tcblower
	$X_{1}, \ldots, X_{10}$ sind unabhängig und je $\normd(\mu, \sigma^2)$ verteilt unter $\pr_\mu$, wobei $\mu$
	unbekannt und $\sigma = 0.5$.
	\begin{equation*}
		H_0 : \mu = \mu_0 = 20 \quad \text{und} \quad H_A : \mu \neq \mu_0
	\end{equation*}
	Mache $z$-Test
	\begin{equation*}
		T = \frac{\overline{X}_n - \mu_0}{\sigma / \sqrt{n}} = \frac{\overline{X}_{10} - 20}{0.5/\sqrt{10}} 
	\end{equation*}
	Unter $H_0$ gilt $T \sim \normd(0,1)$ und wähle $K_{\neq} = ]-\infty,-c_{\neq}[ \cup ]c_{\neq},\infty[$. Verwerfe
	$H_0$ falls $|T| > c_{\neq}$
	\begin{gather*}
		0.05 = \alpha = \pr_{\mu_0} [T \in K_{\neq}] = 2(1- \Phi (c_{\neq})) \Rightarrow c_{\neq} = 1.96\\
		T(\omega) = t(x_{1}, \ldots, x_{10}) = 3.4342
	\end{gather*}
	Da $|T(\omega)| > 1.96$ verwerfen wir $H_0$.\\
	Der realisierte P-Wert ist
	\begin{equation*}
		G(T(\omega)) = \pr_{\mu_0} [|T| > 3.43] = 2 \pr_{\mu_0 [T > 3.43]} = 2(1 - \Phi(3.43)) \approx 0.0006
	\end{equation*}
\end{example}
