\section{Geometry \& Textures}
\greenbf{Challenges, texture:} Noisy captured images, visual redundancy over space, callibration inaccuracies, reconstruction inaccuracies, occlusions, visual redundancy over time geometric noise \graytext{(reconstruction noise \& callibrating noise)}\\
\greenbf{Ways to encode geometry:} \\
Explicit: Vertex positions are given explicitly $\rightarrow$ good for sampling, bad for testing whether inside or outside object.\\
Implicit: Vertex positions fulfil some equation. $\rightarrow$ good to test inside/outside object, compact description, tough to model complex shapes, finding all points is expensive.\\
\greenbf{Geometry representations implicit}
\begin{compactitem}
    \item Algebraic surfaces: \graytext{surface is zero set of polynomial in $x, y, z$}
    \item Constructive solid geometry: \graytext{build complicated shapes via Boolean operations}
    \item Blobby surfaces: \graytext{gradually blend surfaces together (levels of sum of gaussians)}
    \item Blending distance functions: \graytext{a distance function gives distance to closest point on object}
    \item Level set methods: \graytext{store a grid of values approximating function}
    \item Fractals and L-systems: \graytext{no precise definition, structures that exhibit self-similarity, details at all scales, self-similarity, details at all scales}
    \item Signed Distance Function
\end{compactitem}
\greenbf{Geometry representations explicit}
\begin{compactitem}
    \item Point cloud: \graytext{list of points $(x, y, z)$, often augmented with normals can represent any gemetry, need large dataset, hard to do processing/simulation, hard to draw if undersampled}
    \item Polygonal mesh: \graytext{Store vertices and polygons, easier processing simulation, more cimplicated DS, most common}
    \item Triangle mesh: \graytext{store vertices as triplets $(x, y, z)$ triangles as triples of indices $(i, j, k)$}
    \item Subdivision surfaces: \graytext{smooth out a control curve, insert new vertex at each edge midpoint and update vertex positions according to fixed rule}
 \end{compactitem}

\subsection*{Mesh Datastructure}
\greenbf{Triangle List:} List containing ($v_1, v_2, v_3$) where $v_i$ is the coordinates $\Rightarrow$ easy query, but redundant.\\
\greenbf{Indexed Face Set:} List containing vertex ids and another list of vertices with their coordinates $\Rightarrow$ less storage space.
\subsection*{Polygonal Mesh} Set of connected polygons where every edge belongs to at least one polygon and the intersection of two polygons either empty, a vertex or and edge.\\
\greenbf{Manifolds:} surface homeomorphic to a disk, closed manifolds divides space into two.

\subsection*{Texture Mapping}
Enhance details without increasing geometric complexity. Desirable properties: low distortion, bijective mapping, efficiency. \\
\greenbf{Parametrization:} Map $(u,v)$ coordinates of texture to 3D vertex coordinates. E.g. for spheres
$\begin{bmatrix}
    %\begin{smallmatrix}
        u \\ v
    %\end{smallmatrix}    
\end{bmatrix} \mapsto 
\begin{bmatrix}
    %\begin{smallmatrix}
        sin(u)sin(v) \\
        cos(v)\\
        cos(u)sin(v)
    %\end{smallmatrix}
\end{bmatrix}$\\
\greenbf{Texture Filtering:} To prevent aliasing, we should apply low pass filter to the texture. \\
\greenbf{Maps:}
\begin{compactitem}
    \item Light map: simulates effect of a local light source
    \item Environment map: render reflective object efficiently
    \item Bump mapping
    \item Normal mapping
    \item Mipmapping
\end{compactitem}
\greenbf{Bump Mapping:}
Perturbs surface normal. Encodes height difference (grayscale) from mesh. Illusion of geometry, but (self-)shadows and silhouette unchanged. \\
\greenbf{Normal mapping:}
Very similar to bump mapping but now stored as $(r,g,b)$ color $\Rightarrow$ directional perturbations. More detailed \\
\greenbf{Mipmapping:}
Store down-sampled versions of a texture using Gaussian Pyramid. Choose resolution based on projected size of triangle. Use linear interpolation between resolutions. Prevents aliasing!\\
\greenbf{Magnification:}
Pixel in texture image maps to area larger than one pixel $\rightarrow$ Jaggies. Can be solved by bilinear interpolation. \\
\greenbf{Minification:}
Pixel in texture image maps to area smaller than one pixel $\rightarrow$ moiré patterns. Solution: mipmapping.