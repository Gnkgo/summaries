\section{Ray Tracing}
\subsection*{Rasterization vs Raycasting}
\greenbf{Rasterization:} Proceed in triangle order, most processing based on 2D primitives (3D that was projected. Store depth buffer).\\
\greenbf{Raytracing:} Proceeds in screen sample order, never have to store depth buffer (just current ray), natural order for rendering transparent surfaces. Must store entire scene.\\
\greenbf{Shadow mapping:} Render scene (depth buffer only) from location of light. Everything "seen" (depth test success) from this PoV is directly lit, if depth test fail $\rightarrow$ shadow.\\
\greenbf{Shadows ray tracing:} shoot "shadow" rays towards light source from points where camera rays intersect scene. If nothing in the way $\rightarrow$ lighted, else $\rightarrow$ shadow.\\
\greenbf{Environment mapping:} approximate appearance of reflective surface by placing a ray origin at location of reflective object, render six views (for a cube). Use camera ray reflected about surface normal to determine which texel in cube is "hit".\\
\greenbf{Reflections: ray tracing:} recursive ray tracing, compute a secondary ray from surface in reflection direction.

\greenbf{Ray Casting}
    Shoot ray through from the camera through the pixels and in first intersection, evaluate the illumination model.\\
  \greenbf{Forward Raytracing}
    Rays from light source (not efficient).\\
  \greenbf{Backward Raytracing}
    Shoot rays from the camera.\\
  \greenbf{The Pipeline}
    \begin{compactenum}
      \item \textit{Ray Generation}: Shoot ray from origin.
      \item \textit{Intersection}: Calculate first intersection. Calculate illumination at that point by recursion (either reflect or refract).
      \item \textit{Shading}: Shoot ray from intersection to directly to light source. Intersection \(\implies\) Point in shadow.
    \end{compactenum}
  \greenbf{Supersampling} Shoot multiple rays to remove aliasing.\\
%\greenbf{Ray-Surface Intersections:} For origin $o$ and direction $d$ ray: $r(t) = o + td$. For sphere with center $c$ and radius $r$ solve $\|r(t) - c\|^2 - r^2 = 0$ for t. For triangle with corners $p_1, p_2, p_3$ use barycentric coordinates with parameters $s_1, s_2, s_3$: $x = s_1p_1 + s_2p_2 + s_3p_3$, if all $s_i \geq 0$ and $s_1 + s_2 + s_3 = 1$ point $x$ is on triangle. Intersect ray with triangle plane: $t = -\frac{(o-p_1)n}{dn}$ where $n = (p_2 - p_1) \times (p_3 - p_1)$, Compute $s_i$, test $s_1 + s_2 + s_3 = 1$ and $s_i \geq 0$. If successful ray intersects triangle. To check if the intersection is in shadow, check if vector from intersection to light source is blocked by an object.
\greenbf{Shading:} physically correct too costly, instead assume surface reflectance (diffuse, specular, ambient, transparent), use shadow rays for shadows. Extensions: model refraction,  multiple light sources, area light for soft shadows, sample and intersect in time for motion blur, depth of field.\\
\greenbf{Acceleration:} Cost for ray tracing O(\#rays * \#objects). \\
\textbf{Uniform grids}:
\begin{compactitem}
    \item Preprocess: Bounding box, grid resolution, rasterize objects, store references to objects. 
    \item Incrementally rasterize ray and stop at intersection with rasterized object. \\Advantages: fast to build, easy to code. 
    \\Disadvantages: not adaptive to scene geometry.
\end{compactitem}
\textbf{Space partitioning trees}: octree, kd-tree, bsp-tree.