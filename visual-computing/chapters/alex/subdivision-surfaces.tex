\section{Subdivision Surfaces}

\greenbf{Corner Cutting:} Insert two new vertices at $\frac{1}{4}$ and $\frac{3}{4}$ of each edge. Remove old and connect new vertices.
\includegraphics*[width = \columnwidth]{alex/cornerCutting.png}

\greenbf{Subdivision surfaces:} Generalisation of spline curves/surfaces, arbitrary control meshes, successive refinement, converges to smooth limit surface, connection between splines and meshes. In a sense similar to deCasteljau (corner cutting). No regular structure like curves (arbitrary number of edge neighbours, different subdivision rules for each valence). \\
\textbf{Classification:} Primal: faces are split into sub-faces. Dual: Vertices are split into multiple vertices. Approximating: Control points not interpolated. Interpolating: Control points interpolated.
\includegraphics*[width = \columnwidth]{alex/subdivisionClassification.png}
\textbf{Doo-Sabin:} generalisation of bi-quadratic B-Splines, for polygonal meshes, generates $G^1$ continuous surfaces.\\
\textbf{Catmull-Clark:} generalisation of bi-cubic B-Spline, polygonal meshes, $G^2$\\
\textbf{Loop Subdivision:} generalisation of box splines, triangle meshes, $G^2$\\
\textbf{Butterfly:} triangle meshes, $G^1$\\
\includegraphics*[width = 0.5\columnwidth]{alex/subdivisionExamples.png}

\greenbf{Local Subdivision matrix:} TODO?

