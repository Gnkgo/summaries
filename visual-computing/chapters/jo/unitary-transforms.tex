\section{Unitary Transforms}
\greenbf{Vectorization:} interpret image as vector row-by-ow: \graytext{$I = \begin{bmatrix}
    \begin{smallmatrix}
        1 & 2 & 3\\
        4 & 5 & 6
    \end{smallmatrix}
\end{bmatrix} \rightarrow \begin{bmatrix}
    \begin{smallmatrix}
        1 & 2 & 3 & 4 & 5 & 6
    \end{smallmatrix}
\end{bmatrix}$}\\
\greenbf{linear image processing:} can be written as $\vec{g} = H\vec{f}$\\
\greenbf{Image collection (IC):} $F = [f_1, f_2... f_n]$\\
\greenbf{Autocorrelation matrix} $Rff = \frac{F \cdot F^H}{N}$ its Eigenvector with largest Eigenvalue is direction of largest variance among pictures.\\
\greenbf{Unitary transform:} for transform $A$ iff $A^H = A^{-1}$ \graytext{if real-valued $\rightarrow$ orthonormal}\graytext{every unitary transform is a rotation + sign flip, length conserved}\\
\greenbf{Energy conservation:} $||\vec{C}||^{2} = \vec{C}^{H}C = \vec{f}^{H}A^{H}Af = ||\vec{f}||^{2}$ 

\subsection*{Karhunen-Loeve Transform \graytext{Same as PCA. Order by decreasing eigenvalues}}
\greenbf{Energy concentration property:} no other unitary transform packs as much energy in the first $J$ coefficients \graytext{(for arbitrary $J$)} and mean squared approximation error by choosing only first $J$ coefficients is minimized.\\
\greenbf{Optimal energy concentration of KLT} consider truncated coefficient vector $\vec{b} = I_J \vec{c}$ \graytext{($I_J$: identity matrix with first J columns)} Energy in first $J$ coefficients for an arbitrary transform $A : E = Tr(R_{bb}) = Tr(I_J R_{cc} I_{J}) = Tr(I_J A R_{ff} A^H I_J) = \sum_{k = 0}{J = 1} a_k^T R_{ff} a_k^*$ where $a_k^T$ is $k-th$ row of $A$. Lagrangian cost function to enforce unit-length basis vectors: 
$L = E + \sum_{k = 0}^{J - 1} \lambda_k (1 - a_k^T a_k^*) = \sum_{k = 0}^{J - 1} a_k^T R_{ff} a_k^* + \sum_{k = 0}^{J - 1} \lambda_k (1 - a_k^T a_k^*)$\\ 
Differentiating $L$ with respect to $a_j$: $R_{ff} a_j^* = \lambda_i a_j^* \quad \forall_j < J$ \graytext{necessary condition}
\subsection*{Simple recognition}
SSD between images, best match wins \graytext{very expensive, since need to correlate with every image}
\subsection*{Principle Component analysis \graytext{PCA}}

\greenbf{Linear dimension reduction method}

\greenbf{Optimization goal:}

$\argmin{||w||_2 = 1, z} \sum_{i=1}^n ||x_i - z_i w||_2^2$

The optimal solution is given by \\
$z_i = w^\top x_i$.  

Substituting gives us:\\
\qquad \qquad $\hat{w} = \text{argmax}_{||w||_2=1} \; w^\top \Sigma w$

Where $\Sigma = \frac{1}{n} \sum_{i=1}^n x_i x_i^\top$ is the empirical covariance. Closed form solution given by the principal eigenvector of $\Sigma$, i.e. $w = v_1$ for $\lambda_1 \geq \cdots \geq \lambda_d \geq 0$:
$\Sigma = \sum_{i=1}^d \lambda_i v_i v_i^\top$

For $k > 1$ we have to change the normalization to $W^\top W = I$ then we just take the first $k$ principal eigenvectors so that $W = [v_1, \ldots, v_k]$.

\greenbf{Steps:} 

\begin{compactitem}
    \item Center image
    \item Normalize data and subtract mean \graytext{necessary to ensure first principal component describes direction of maximum variance. Otherwise, first principal component would correspond to mean}
    \item Get Eigenvectors and values from covariance matrix or do SVD
    \item Sort Eigenvalues and vectors in descending order
    \item Get $j$ largest components
    \item Construct projection matrix from selected $j$ Eigenvectors ($U_j$)
    \item Transform dataset by multiplying with projection matrix
\end{compactitem}

\greenbf{PCA through SVD}
\begin{compactitem}
	\item The first $k$ col of $V$ where $X = U S V^\top$.
	\item first principal component eigenvector of data covariance matrix with largest eigenvalue
	\item covariance matrix is symmetric $\rightarrow$ all principal components are mutually orthogonal
\end{compactitem}	


\subsection*{Kernel PCA}

$\Sigma = \frac{1}{n} \sum_{i=1}^n x_i x_i^\top = X^\top X \Rightarrow$  kernel trick:

\qquad \qquad $\hat{\alpha} = \text{argmax}_\alpha \; \frac{\alpha^\top K^\top K \alpha}{\alpha^\top K \alpha}$

Closed form solution:

$\alpha^{(i)} = \frac{1}{\sqrt{\lambda_i}}v_i \quad K = \sum_{i = 1}^n \lambda_i v_i v_i^\top, \lambda_1 \geq \cdots \geq 0$

A point $x$ is projected as:
$z_i = \sum_{j=1}^n \alpha_j^{(i)} k(x_j, x)$

\includegraphics*[width = 0.4\columnwidth]{jo/pca.png}\\
\greenbf{Uses of PCA:} lossy compression by keeping only the most important $k$ components.
\begin{compactitem}
    \item take the original image I
    \item apply PCA on the original image, if you do not have a PCA already.
    % \item decide of much variance you want to retain and from that build your compressed data
    \item Compress the image by projecting the image into the PCA subspace. $(I - \mu)U_k$ where $U_k$ is the matrix of the k Principal components.
    \item apply the inverse PCA 
    transformation from point 2. on the compressed data to get the reconstructed image. $I \cdot U_k^T + \mu$
\end{compactitem}
\greenbf{Calculate units of PCA}\\
Assume dataset of 1000 images, with size $50 \times 50$
\begin{compactenum}
    \item dataset mean = $50 \times 50 = 2500$
    \item Truncated eigenmatrix $2500 \times K$
    \item Compressed images $1000 \times K$
    \item $I_K = (I - \bar{I})\Phi$
    \item $\hat{I} = I_K \Phi^T + \bar{I}$
\end{compactenum}
PCA is just a linear transformation from one coordinate system to another, which can easily be "undone" in a lossless manner by reversing the transformation. The dimensionality reduction aspect comes when you start dropping the last principal components, which are the dimensions which capture the least variance.\\
Face recognition \graytext{eigenfaces} and face detection.
\subsection*{Eigenspace matching}
Do PCA \graytext{with mean subtraction} and get closest rank-$k$ approximation of database images \graytext{(eignfaces)} \\
For a new query: normalize, subtract mean \graytext{(of database)} project to subspace then do similarity matching with eigenfaces.
\subsection{Fischerfaces:} 
Find directions where ratio between / within individual variance is maximized. Linearly project to basis where dimension with good signal: noise ratio is maximized. \\
$W_{\text{opt}} = \argmax{W} \frac{\det(W R_B W^H)}{\det(W R_W W^H)}, R_b = \sum R_B \sum_{i = 1}{c} N_i (\vec{\mu_i} - \vec{\mu})(\vec{\mu_i} - \vec{\mu})^H, R_W = \sum_{i = 1}^{c} \sum_{\Gamma_l \in Class} (\Gamma_l - \mu_i)(\Gamma_l - \mu_i)^H$\\
\greenbf{Fischer linear discriminant analysis \graytext{(LDA)}:} maximize between class scatter, while minimizing within less scatter
\subsection*{JPEG Compression}
Divide image into $8 \times 8$ block:\\
\includegraphics*[width = \columnwidth]{jo/jpg.png}\\
\greenbf{Discrete cosine transform (DCT):} uses only real values and is easier to compute than a Fourier transform.\\
\greenbf{DC:} First coefficient (general intensity)\\
\greenbf{ZigZag:} \includegraphics*[width = 1.5cm]{jo/zig-zag.png}\\
\greenbf{Quantization Table:} Divide by this value, round to nearest integer, lossy