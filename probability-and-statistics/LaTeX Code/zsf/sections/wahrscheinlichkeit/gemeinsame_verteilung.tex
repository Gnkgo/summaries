\section{Gemeinsame Verteilung}%
\label{sec:gemeinsame_verteilung}

\subsection{Diskrete gemeinsame Verteilungen}%
\label{sub:diskrete_gemeinsame_verteilungen}

\begin{definition}{Gemeinsame Verteilung}
	Seien $X_1 , \ldots X_n$ $n$ diskrete ZV mit $X_i \in W_i$ fast sicher, für beliebige $W_i$ endlich/abzählbar. Die
	gemeinsame Verteilung von $(X_1 , \ldots X_n)$ ist die Menge $p = \big( p(x_1 , \ldots , x_n)\big)_{x_1 \in W_1 ,
	\ldots , x_n \in W_n}$ ist definiert durch
	\begin{equation*}
		p (x_1 , \ldots , x_n) = \pr [X_1 = x_1 , \ldots , X_n = x_n]
	\end{equation*}
\end{definition}
\begin{center}
	\begin{minipage}{0.6\linewidth}
		Seien $X \sim \bern (1/2)$ und $Y \sim \bern (1/2)$ unabhängig.
		Die Verteilung von $(X,Y)$ ist gegeben durch 
		\begin{equation*}
			\forall x,y \in \{0,1\} ~ p(x,y) = \frac{1}{4} 
		\end{equation*}
		Die Verteilung von $(X,X)$ ist gleich
		\begin{equation*}
			\forall x,y \in \{0,1\} ~ p(x,y) = 
			\begin{cases}
				\frac{1}{2} & x = y\\
				0 & x \neq y
			\end{cases}
		\end{equation*}
	\end{minipage}
	\hfill\vline\hfill
	\begin{minipage}{0.35\linewidth}
		Die Verteilung von \\
		$(X, X + Y)$ ist
		\begin{equation*}
			\begin{array}{c|c|c}
				x & y & p(x,y)\\
				\hline
				0 & 0 & 1/4\\
				0 & 1 & 1/4\\
				0 & 2 & 0\\
				1 & 0 & 1/4\\
				1 & 1 & 1/4\\
				1 & 2 & 0
			\end{array}
		\end{equation*}
	\end{minipage}
\end{center}
\begin{prop}
	Die gemeinsame Verteilung für beliebige ZV $X_1, \ldots , X_n$ erfüllt 
	\begin{equation*}
		\sum_{x_1 \in W_1 , \ldots , x_n \in W_n} p (x_1 , \ldots , x_n) = 1
	\end{equation*}
\end{prop}
\begin{prop}
	Sei $n \geq 1$ und $\phi : \R^n \rightarrow \R$ eine beliebige Funktion. Seien $X_1 , \ldots , X_n$ $n$ diskrete ZV
	auf $(\Omega, \F , \pr)$ mit Werten in endlichen/abzählbaren Mengen $W_1 , \ldots , W_n$ fast sicher. Dann ist $Z =
	\phi (X_1 , \ldots , X_n)$ eine diskrete ZV mit Werten in $W = \phi (W_1 \times \ldots \times W_n)$ fast sicher und
	eine Verteilung 
	\begin{equation*}
		\forall z \in W ~ \pr [Z = z] = \sum_{\substack{x_1 \in W_1, \ldots , x_n \in W_n \\ \phi(x_1 , \ldots , x_n) =
		z}} \pr [X_1 = x_1 , \ldots , X_n = x_n]
	\end{equation*}
\end{prop}
Seien $X \sim \bern (1/2)$ und $Y \sim \bern (1/2)$ unabhängig und $X \coloneqq X + Y$. Durch Anwendung auf $\phi (x,y)
= x+y$ ergibt sich:
\begin{align*}
	\pr [Z = 0] &= \sum_{\substack{x,y \in \{0,1\}\\ x + y = 0}} \pr [X = x, Y = y] = \pr [X = 0, Y = 0] = \frac{1}{4} \\
	\pr [Z = 1] &= \sum_{\substack{x,y \in \{0,1\}\\ x + y = 1}} \pr [X = x, Y = y] = \pr [X = 0, Y = 1] + \pr [X = 1, Y
	= 0] \\
					&= \frac{1}{2} \\
	\pr [Z = 2] &= \sum_{\substack{x,y \in \{0,1\}\\ x + y = 2}} \pr [X = x, Y = y] = \pr [X = 1, Y = 1] = \frac{1}{4}
\end{align*}
\begin{tprop}{Randverteilung}
	Seien $X_1 , \ldots , X_n$ $n$ diskrete ZV mit gemeinsame Verteilung $p = \big( p(x_1 , \ldots , x_n)\big)_{x_1 \in W_1 ,
	\ldots , x_n \in W_n}$. Für jedes $i$ gilt
	\begin{equation*}
		\forall z \in W_i ~\pr [X_i = z] = \sum_{x_1 , \ldots x_{i-1} , x_{i +1} , \ldots , x_n} p(x_1 , \ldots x_{i-1} ,
		z, x_{i +1} , \ldots , x_n)
	\end{equation*}
\end{tprop}
\begin{tprop}{Erwartungswert der Abbildung}
	Seien $X_1 , \ldots , X_n$ $n$ diskrete ZV mit gemeinsame Verteilung $p = \big( p(x_1 , \ldots , x_n)\big)_{x_1 \in W_1 ,
	\ldots , x_n \in W_n}$. Sei $\phi : \R^n \rightarrow \R$, dann
	\begin{equation*}
		\E \big[\phi (X_1 , \ldots , X_n)\big] = \sum_{x_1 , \ldots , x_n} \phi (x_1 , \ldots , x_n) p(x_1 , \ldots , x_n)
	\end{equation*}
	wenn die Summe wohldefiniert ist.
\end{tprop}
\begin{tprop}{Unabhängigkeit}
	Seien $X_1 , \ldots , X_n$ $n$ diskrete ZV mit gemeinsame Verteilung $p = \big( p(x_1 , \ldots , x_n)\big)_{x_1 \in W_1 ,
	\ldots , x_n \in W_n}$. Die folgenden Aussagen sind äquivalent:
	\begin{itemize}
		\item $X_1 , \ldots , X_n$ sind unabhängig.
		\item $p (x_1 , \ldots , x_n) = \pr [X_1 = x_1] \cdot \ldots \cdot \pr[X_n = x_n]$ \\ für jedes $x_1 \in W_1 , \ldots
			, x_n \in W_n$
	\end{itemize}
\end{tprop}


\subsection{Stetige gemeinsame Verteilung}%
\label{sub:stetige_gemeinsame_verteilung}

\begin{definition}{Stetige gemeinsame Verteilung}
	Sei $n \geq 1$ für ZVs $X_1 , \ldots , X_n : \Omega \rightarrow \R$ haben eine \emph{gemeinsame, stetige Verteilung
} falls eine Funktionen f $f : \R^n \rightarrow \R_+$ existiert, sodass
	\begin{equation*}
		\pr [X_1 \leq a_1, \ldots , X_n \leq a_n] = \int_{-\infty}^{a} \ldots \int_{- \infty}^{b} f(x_1 ,\ldots , x_n)
		\dif x_1 \ldots \dif x_n
	\end{equation*}
	für jedes $a_1 , \ldots , a_n \in \R$. Eine solche Funktion $f$ heisst \emph{gemeinsame Dichte} von $(X_1 , \ldots ,
	X_n)$.
\end{definition}
\begin{prop}
	Sei $f$ eine gemeinsame Dichte von $(X_1, \ldots , X_n)$. Dann gilt
	\begin{equation*}
		\int_{-\infty}^{\infty} \ldots \int_{-\infty}^{\infty} f(x_1, \ldots x_n) \dif x_n \ldots \dif x_1 = 1
	\end{equation*}
\end{prop}
\begin{tprop}{Erwartungswert der Abbildung}
	Sei $\phi : \R^2 \rightarrow \R$. Falls $X_1, \ldots , X_n$ eine gemeinsame Dichte haben, dann kann der der Erwartungswert der ZV $Z =
	\phi (X_1 , \ldots , X_n)$ mit
	\begin{equation*}
		\E [\phi (X_1, \ldots , X_n)] = \int_{-\infty}^{\infty} \ldots \int_{-\infty}^{\infty} \phi(x_1, \ldots x_n) \cdot
		f(x_1, \ldots , x_n) \dif x_n \ldots \dif x_1
	\end{equation*}
	berechnet werden. (Integral muss Wohldefiniert sein.)
\end{tprop}
\begin{prop}
	Seien $X_1 , \ldots , X_n$ $n$ ZV mit einer gemeinsamen Dichte $f = f_{X_1 , \ldots , X_n}$. Dann für jedes $i$, ist
	$X_i$ eine stetige ZV mit Dichte $f_i$, definiert durch
	\begin{gather*}
		f_i(z) = \int_{(x_1 , \ldots , x_{i-1}, x_{i+1}, \ldots , x_n) \in \R^{n-1}} f (x_1 , \ldots , x_{i-1}, z,
		x_{i+1}, \ldots , x_n)\\
		\dif x_1 \ldots \dif x_{i-1} \dif x_{i+1} \dif x_n 
	\end{gather*}
	für jedes $z \in \R$
\end{prop}
Falls $X,Y$ eine gemeinsame Dichte haben, gilt 
\begin{equation*}
	\pr [X \leq a] = \pr \big[ X \in [-\infty,a], Y \in [-\infty,\infty]\big] = \int_{-\infty}^{a} \left(
	\int_{-\infty}^{\infty} f(x,y) \dif y \right) \dif x
\end{equation*}
\begin{tcolorbox}[lemmacore]
	Falls $X,Y$ eine gemeinsame Dichte $f_{X,Y}$ haben, gilt für $X$ resp. $Y$
	\begin{equation*}
		f_X (x) = \int_{-\infty}^{\infty} f_{X,Y}(x,y) \dif y
		\qquad
		f_Y (y) = \int_{-\infty}^{\infty} f_{X,Y}(x,y) \dif x
	\end{equation*}
\end{tcolorbox}
\begin{theorem}{Unabhängigkeit stetiger ZV}
	Seien $X_1, \ldots X_n$ $n$ stetige ZV mit den jeweiligen Dichten  $f_1 , \ldots , f_n$. Die folgenden Aussagen sind äquivalent:
	\begin{enumerate}
		\item $X_1 , \ldots , X_n$ sind unabhängig.
		\item $X_1 , \ldots , X_n$ sind gemeinsam stetig mit gemeinsamer Dichte\\ $f(x_1 , \ldots, x_n) = f_1 (x_1) \cdot
			\ldots \cdot f_n (x_n)$
	\end{enumerate}
\end{theorem}
Zwei unabhängige, stetige ZV sind automatisch gemeinsam stetig.
