\documentclass[9pt]{extarticle}
% Article
% \documentclass[9pt]{extarticle}
\usepackage[landscape, left=0.10cm, top=0.1cm, right=0.1cm, bottom=0.3cm, footskip=2pt]{geometry}
\usepackage{background}
\usepackage{etoolbox}
\usepackage{graphicx}
\usepackage{totcount}
\usepackage{lipsum}
\usepackage{hyperref}
\usepackage{amsmath}
\usepackage{amssymb}
\usepackage{physics}
\usepackage{enumerate}

\usepackage{xcolor}
\usepackage{tcolorbox}

\usepackage[compact]{titlesec}
\usepackage{paralist}

\usepackage{tabularx}
\usepackage{ctable}

% Set page margins

% Tables
\usepackage{tabularx, multirow}
\usepackage{booktabs}
\renewcommand*{\arraystretch}{2}

\def\BoxStart{\begin{tcolorbox}[colback=blue!5!white,colframe=blue!75!black]}
    \def\BoxEnd{\end{tcolorbox}}

% image directory
\graphicspath{ {./assets/} }

% For accessing arrays
\usepackage{etoolbox}

% for emumerating
\usepackage{enumitem}

% for color coding
\usetikzlibrary{backgrounds}

% for light font +C
\usepackage{color}
\definecolor{light}{rgb}{0.5, 0.5, 0.5}
\def\light#1{{\color{light}#1}}

% for multicolumn
\usepackage{multicol}
\setlength{\columnseprule}{0.4pt}

% to have access to the total number of subsection*s
%\regtotcounter{subsection*}

% every subsection* starts on a new page
%\pretocmd{\subsection*}{\clearpage}{}{}

% auxiliary lengths for the height of the frame and the width of each tab
\newlength\mylen{}
\newlength\mylena{}


\renewcommand*{\arraystretch}{2}
\allowdisplaybreaks{}

\def\getnthelement#1{\csname mylist#1\endcsname}

% the main part; as background material we place the border, 
% the subsection*  (current and other) tabs and the page number
\backgroundsetup{
  scale=1,
  color=black,
  angle=0,
  opacity=1,
  contents={}
}

% Set indentation
\setlength{\parindent}{2pt}
\setlength{\parskip}{0.02cm}
\setlength{\columnsep}{5pt}
\raggedcolumns
\setlength{\abovedisplayskip}{2pt}
\setlength{\belowdisplayskip}{2pt}
\setlength{\abovedisplayshortskip}{2pt}
\setlength{\belowdisplayshortskip}{2pt}



\newcommand{\N}{\mathbb{N}}
\newcommand{\R}{\mathbb{R}}
\newcommand{\F}{\mathcal{F}}
\newcommand{\W}{\mathcal{W}}
\newcommand{\X}{\mathcal{X}}
\newcommand{\vp}{\varphi}
\newcommand{\vt}{\vartheta}
\newcommand{\ra}{\rightarrow}
\newcommand{\Ra}{\Rightarrow}
\newcommand{\Sn}{\sum_{i = 1}^n}
\newcommand{\Sinfty}{\sum_{i = 1}^\infty}
\newcommand{\Pn}{\prod_{i = 1}^n}
\newcommand{\Pinfty}{\prod_{i = 1}^\infty}
\newcommand{\cond}[2]{P[#1 \; | \; #2]}
\newcommand{\ereignisse}{A_1, \dots, A_n}
\newcommand{\zufallsvariablen}{X_1, \dots, X_n}
\newcommand{\bigunion}{\bigcup_{i = 1}^n}
\newcommand{\bigsubsection}{\bigcap_{i = 1}^n}
\newcommand{\Normalverteilt}{\mathcal{N}  (\mu, \sigma^2)}
\newcommand{\Standardnormalverteilt}{\mathcal{N}  (0, 1)}
\newcommand{\bs}{\boldsymbol}
\newcommand{\with}{\;|\;}

\titlespacing{\section}{0pt}{0.3ex}{0ex}
\titlespacing{\subsection}{0pt}{0ex}{0ex}
\linespread{0.7}
\allowdisplaybreaks{}

\titleformat*{\section}{\fontsize{10}{10}\bfseries\color{blue}}
\titleformat*{\subsection}{\fontsize{10}{10}\bfseries\color{purple}}
% define box
\begin{document}
\setlength{\columnseprule}{0.4pt}
\pagenumbering{arabic}
\begin{multicols*}{3}



% chktex-file 12
% chktex-file 8


{\section{Wahrscheinlichkeiten}}
\subsection*{Ereignisraum}
Die Menge $\Omega \neq \emptyset$ aller möglichen Ergebnisse des betrachteten
Zufallsexperiments. Die Elemente $\omega \in \Omega$ heissen
Elementarereignisse.
\subsection*{Potenzmenge}
Die Potenzmenge von $\Omega$, bezeichnet mit $\mathcal{P} (\Omega)$ oder
$2^\Omega$ ist die Menge aller Teilmengen von $\Omega$. Ein Prinzipielles
Ereignis ist eine Teilmenge $A \subseteq \Omega$, also eine Kollektion von
Elementarereignissen. Die Klasse aller beobachtbaren Ereignisse ist
$\mathcal{F}$.
\subsection*{$\sigma$-Algebra}
Grundraum: Menge $\Omega$. $\omega \in \Omega$ heisst Elementarereigniss.
\begin{itemize}
  \item $\Omega \in \F$
  \item $A \in \F \Rightarrow A^c \in \F$
  \item $A_1 , A_2 , \ldots \in \F \Rightarrow \bigcup_{i=1}^\infty A_i \in \F$
  \item $\emptyset \in \F$
  \item $A_1 , A_2 , \ldots \in \F \Rightarrow \bigcap_{i=1}^\infty A_i \in \F$
  \item $A,B \in \F \Rightarrow A \cup B \in \F$
  \item $A,B \in \F \Rightarrow A \cap B \in \F$
\end{itemize}	


\subsection*{Wahrscheinlichkeitsmass}
Eine Abbildung $\mathcal{P}: \F \to [0, 1]$ mit folgenden Eigenschaften:
\begin{enumerate}[label= (\arabic*)]
  \item $\mathcal{P}[A] \geq 0 \text{ für alle Ereignisse } A \in \F$
  \item $P[\Omega] = 1$
  \item Für $A_i \in \F$ paarweise disjunkt gilt $P[\bigcup_{i = 1}^\infty A_i] =
          \sum_{i = 1}^\infty \mathcal{P}[A_i]$
\end{enumerate}
Es gelten weiter folgende Rechenregeln:
\begin{itemize}
  \item $\mathcal{P}[A^\complement] = 1 - \mathcal{P}[A]$
  \item $\mathcal{P}[\emptyset] = 0$
  \item Für $A \subseteq B$ gilt $\mathcal{P}[A] \leq \mathcal{P}[B]$
  \item $\mathcal{P}[A \cup B] = \mathcal{P}[A] + \mathcal{P}[B] - \mathcal{P}[A \cap B]$
\end{itemize}
\subsection*{Diskrete Wahrscheinlichkeitsräume}
Impliziert:
\begin{itemize}
  \item $\Omega$ ist endlich oder abzählbar unendlich
  \item $\F = 2^{\Omega}$
\end{itemize}
\subsection*{Laplace Raum}
Ist $\Omega = \{\omega_1, \dots, \omega_N\}$ endlich mit $\abs*{\Omega} = N$
und $\F = 2^\Omega$ sowie alle $\omega_i$ gleich wahrscheinlich mit $p_i =
  \frac{1}{n}$, so heisst $\Omega$ ein Laplace Raum und $P$ die diskrete
Gleichverteilung auf $\Omega$. Dann ist für $A \subseteq \Omega$:
\begin{align*}
  P[A] = \frac{\abs{A}}{\abs{\Omega}}
\end{align*}
\subsection*{Bedingte Wahrscheinlichkeit}
Seien $A, B$ Ereignisse mit $P[A] > 0$. Die bedingte Wahrscheinlichkeit von $B$
unter der Bedingung, dass $A$ eintritt wird definiert durch:
\begin{align*}
  P[B \;|\; A] & := \frac{P[B \cap A]}{P[A]}            \\
               & = \frac{P[A \;|\; B] \cdot P[B]}{P[A]}
\end{align*}
\subsection*{Multiplikationsregel}
Es gilt:
\begin{align*}
  P[A \cap B] = P[A \;|\; B] \cdot P[B] = P[B \;|\; A] \cdot P[A]
\end{align*}
\subsection*{Satz der totalen Wahrscheinlichkeit}
Sei $A_1, \dots, A_n$ eine Zerlegung von $\Omega$ in paarweise disjunkte
Ereignisse, d.h. $\bigcup_{i = 1}^n A_i = \Omega$ und $A_i \cap A_k = \emptyset
  \quad \forall i \neq k$. Dann gilt:
\begin{align*}
  P[B] = \sum_{i = 1}^n P[B \; | \; A_i] \cdot P[A_i]
\end{align*}
\emph{Beweis.}
Da $B \subseteq \Omega \implies B = B \cap \Omega
  = B \cap  (\bigcup_{i=1}^n A_i) = \bigcup_{i = 1}^n  (B \cap A_i)$.
Weiter sind alle Mengen der Art $ (B \cap A_i)$ paarweise disjunkt,
was bedeutet, dass $ (B \cap A_i)$ eine disjunkte Zerlegung von $B$
bilden. Damit folgt:
\begin{align*}
  P[B] = P \left[ \bigcup_{i = 1}^n  (B \cap A_i)\right] \\
  = \Sn P[B \cap A_i] = \sum_{i = 1}^n P[B \; | A_i] \cdot P[A_i]
\end{align*}
\subsection*{Satz von Bayes}
Sei $A_1, \dots, A_n$ eine Zerlegung von $\Omega$ mit $P[A_i] > 0$ für $i \in
  \{1, \dots, n\}$. Sei $B$ ein Ereignis mit $P[B > 0]$. Dann gilt für jedes $k$:
\begin{align*}
  \cond{A_k}{B} = \frac{\cond{B}{A_k} \cdot P[A_k] }{\Sn \cond{B}{A_i} \cdot P[A_i]}
\end{align*}
\emph{Beweis.} Verwende Definition Bedingte Wahrscheinlichkeit,
im Zähler Multiplikationsregel und im Nenner Satz der totalen Wahrscheinlichkeit.
\subsection*{Unabhängige Ereignisse  (2)}
Zwei ereignisse heissen (stochastisch) Unabhängig, falls
\begin{align*}
  P[A \cap B] = P[A] \cdot P[B]
\end{align*}
Ist $P[A] = 0$ oder $P[B] = 0$, so sind $A, B$ immer unabhängig.
Für $P[A] \neq 0$ gilt:
\begin{align*}
  A, B \text{ unabhängig} \Longleftrightarrow \cond{A}{B} = P[A]
\end{align*}
\subsection*{Unabhängige Ereignisse  ($\infty$)}
Die Ereignisse $\ereignisse$ heissen (stochastisch) unabhängig, wenn für jede
endliche Teilfamilie der Produktformel gilt, d.h. für $m \in \N$ und $\{k_1,
  \dots, k_m\} \subseteq \{1, \dots, n\}$:
\begin{align*}
  P \left[\bigsubsection* A_{k_i} \right] = \Pn P[A_{k_i}]
\end{align*}

\subsection*{Diskrete Zufallsvariable}
Eine reelwertige diskrete Zufallsvariable auf $\Omega$ ist eine Funktion $X :
  \Omega \mapsto \R$. Mit $\Omega$ ist natürlich auch $\W (X) = \{x_1, x_2,
  \dots\}$ endlich oder abzählbar.
\begin{itemize}
  \item Die Verteilungsfunktion von $X$ ist die Abbildung $F_X : \R \mapsto [0, 1]$,
        definiert durch:
        \begin{align*}
          t \mapsto F_X (t) := P[X \leq t] := P[\{\omega : X (\omega) \leq t\}]
        \end{align*}
  \item Die Gewichtsfunktion oder diskrete Dichte von $X$ ist die Funktion $p_X : \W
          (X) \mapsto [0, 1]$, definiert durch:
        \begin{align*}
          p_X (X_k) := P[X = x_k] = P[\{\omega : X (\omega) = x_k\}]
        \end{align*}
\end{itemize}
Wobei gilt:
\begin{itemize}
  \item $F_X (t) = P[X \leq t] = \sum_{k \text{ mit } x_k \leq t} p_X (x_k)$
  \item Für jedes $x_k \in \W (X)$ gilt $0 \leq p_X (x_k) \leq 1$ und $\sum_{x_k \in \W
            (X)} p_X (x_k) = 1$
  \item $\mu_X (B) := P[X \in B] = \sum_{x_k \in B} p_X (x_k)$
  \item $\sum_{x_k \in \W (X)} p_X (x_k) = P[X \in \W (X)] = 1$
\end{itemize}
\subsection*{Indikatorfunktion}
Für jede Teilmenge $A \subseteq \Omega$ ist die Indikatorfunktion $I_A$ von $A$
definiert durch:
\begin{align*}
  I_A (\omega) :=
  \begin{cases}
    1 \quad \text{falls } \omega \in A             \\
    0 \quad \text{falls } \omega \in A^\complement \\
  \end{cases}
\end{align*}
\subsection*{Erwartungswert}
Sei $X$ eine diskrete Zufallsvariable mit Gewichtsfunktion $p_X (x)$, dann ist
der Erwartungswert definiert durch:
\begin{align*}
  E[X] := \sum_{x_k \in \W (X)} x_k \cdot p_X (x_k)
\end{align*}
und hat folgende Eigenschaften:
\begin{itemize}
  \item Linearität: $E[a \cdot X + b] = a \cdot E[X] + b$
  \item Monotonie: $X \leq Y \implies E[X] \leq E[Y]$
  \item Nimmt $X$ nur Werte in $\N $ an:
        \begin{align*}
          E[X] = \Sinfty P[X \geq i]
        \end{align*}
\end{itemize}
\subsection*{Erwartungswert von Funktionen}
Sei $X$ eine Diskrete Zufallsvariable mit Gewichtsfunktion $p_X (x)$ und $Y = g
  (X)$ für eine Funktion $Y: \R \mapsto \R$. Dann gilt:
\begin{align*}
  E[Y] = E[g (X)] = \sum_{x_k \in \W (X)} g (x_k) \cdot p_X (x_k)
\end{align*}
\subsection*{Varianz}
Sei $X$ eine diskrete Zufallsvariable. Ist $E[X^2] < \infty$, so heisst:
\begin{align*}
  Var[X] & := E[ {X - E[X]}^2]                                     \\
         & = \sum_{x_k \in \W (X)}  {x_k - E[X]}^2 \cdot p_X (x_k)
\end{align*}
die Varianz von $X$. Es gilt weiter:
\begin{itemize}
  \item $Var[X] = E[X^2] - {E[X]}^2$
  \item $Var[a \cdot X + b] = a^2 \cdot Var[X]$
  \item $Var[X - Y] = Var[X] +  {-1}^2 \cdot Var[Y]$
  \item $Var(X+Y) = Var(X) + Var(Y) + 2Cov(X,Y)$
\end{itemize}
\subsection*{Standardabweichung}
$$
  \sigma (X) = \sqrt{Var[X]}
$$
\subsection*{Gemeinsame Verteilung}
Seien $\zufallsvariablen$ beliebige Zufallsvariablen. Die Gemeinsame
Verteilungsfunktion von $\zufallsvariablen$ ist die Abbildung $F: \R^n \mapsto
  [0, 1]$, definiert durch:
\begin{align*}
  (x_1, \dots, x_n) \mapsto F (x_1, \dots, x_n) & := P[X_1 \leq x_1, \dots, X_n \leq x_n]                        \\
                                                & = \sum_{y_1 \leq x_1, \dots, y_n \leq x_n} p (y_1, \dots, y_n)
\end{align*}
Die Gemeinsame Gewichtsfunktion ist:
\begin{align*}
  p (x_1, \dots, x_n) := P[X_1 = x_1, \dots, X_n = x_n]
\end{align*}

\subsection*{Unabhängige Zufallsvariablen}
Zufallsvariablen $\zufallsvariablen$ heissen Unabhängig, falls gilt
(äquivalent):
\begin{align*}
  F (x_1, \dots, x_n) & = F_{X_1} (x_1) \cdot \hdots \cdot F_{X_n} (x_n) \\
  p (x_1, \dots, x_n) & = p_{X_1} (x_1) \cdot \hdots \cdot p_{X_n} (x_n)
\end{align*}
\subsection*{Unabhängige Ereignisse}
Ereignisse $\ereignisse$ heissen Unabhängig, falls für beliebige Teilmengen
$B_i \subseteq \W (X_i) \quad i = 1, \dots, n$ gilt (äquivalent):
\begin{align*}
  P[X_1 \in B_1, \dots, X_n \in B_n] = \Pn P[X_i \in B_i]
\end{align*}
\subsection*{Funktionen von Zufallsvariablen}
Seien $\zufallsvariablen$ diskrete Unabhängige Zufallsvariablen und $f_i: \R
  \mapsto \R$ irgendwelche Funktionen. Sei weiter $Y_i := f_i (X_i)$. Dann sind
die Zufallsvariablen $Y_1, \dots, Y_n$ ebenfalls unabhängig.
\subsection*{Linearität des Erwartungswertes}
Seien $\zufallsvariablen$ diskrete Zufallsvariablen mit endlichen
Erwartungswerten. $E[X_1], \dots, E[X_n]$. Sei $Y = a + \Sn b_i \cdot X_i$ mit
Konstanten $a, b_1, \dots, b_n$. Dann gilt:
\begin{align*}
  E[Y] = a + \Sn b_i \cdot E[X_i]
\end{align*}
\subsection*{Kovarianz}
Seien $X, Y$ Zufallsvariablen auf einem Wahrscheinlichkeitsraum $ (\Omega, \F,
  P)$ mit endlichen Erwartungswerten. Dann ist die Kovarianz definiert als:
\begin{align*}
  Cov (X, Y) & := E[XY] - E[X]E[Y]          \\
             & = E[ (X - E[X])  (Y - E[Y])]
\end{align*}
Wobei $Cov (X, X) = Var[X]$.
\subsection*{Korrelation}
Die Korrelation von $X, Y$ ist definiert durch
\begin{align*}
  \rho (X, Y) := \begin{cases}
                   \frac{Cov (X, Y)}{\sigma (X) \cdot \sigma (Y)} & \text{falls } \sigma (X) \cdot \sigma (Y) > 0 \\
                   0                                              & \text{sonst.}
                 \end{cases}
\end{align*}
und es gilt $\abs{Cov (X, Y)} \leq \sigma (X) \cdot \sigma (Y)$
beziehungsweise $-1 \leq \rho (X, Y) \leq 1$.
\subsection*{Summenformel für Varianzen}
\begin{align*}
  Var \left[ \Sn X_i \right] = \Sn Var[X_i] + 2 \cdot \sum_{i < j} Cov (X_i, X_j)
\end{align*}
ist aber $Cov (X, Y) = 0$  ($X, Y$ paarweise unkorreliert), so wird
die Summe linear.
\subsection*{Produkte von Zufallsvariablen}
Seien $\zufallsvariablen$ diskrete Zufallsvariablen mit endlichen
Erwartungswerten. Falls $\zufallsvariablen$ unabhängig sind, so ist:
\begin{align*}
  E \left[ \Pn X_i \right] = \Pn E[X_i]
\end{align*}
Dann sind auch $\zufallsvariablen$ paarweise unkorreliert und:
\begin{align*}
  Var \left[ \Sn X_i \right] = \Sn Var[X_i]
\end{align*}
da Unabhängig $\implies$ paarweise Unabhängig $\implies$ unkorreliert.
\subsection*{Bedingte Verteilung}
Seien $X, Y$ diskrete Zufallsvariablen mit gemeinsamer Gewichtsfunktion $p (x,
  y)$. Die bedingte Gewichtsfunktion von $X$, gegeben dass $Y = y$, ist definiert
durch:
\begin{align*}
  p_{X \, | \, Y} (x \; | \; y)    & := \cond{X = x}{Y = y}     \\
  \frac{P[X = x, Y = y]}{P[Y = y]} & = \frac{p (x, y)}{p_Y (y)}
\end{align*}
für $p_Y (y) > 0$ und $0$ sonst.
\subsection*{Kriterium für Unabhängigkeit}
$X, Y$ sind genau dann unabhängig, wenn für alle $y$ mit $p_Y (y) > 0$
gilt:
\begin{align*}
  p_{X\,|\,Y} (x \; | \; y) = p_X (x) &  & \forall x \in \W (X)
\end{align*}
\subsection*{$n$ tief $k$}
\begin{align*}
  \binom{n}{k} = \frac{n!}{k! \cdot  (n - k)!}
\end{align*}

\input{Kapitel3}
\subsection*{Zufallsvariable}
\includegraphics[width=\columnwidth]{diskrete_stetige_verteilung.png}\\
Sein $ (\Omega, \F, P)$ ein Wahrscheinlichkeitsraum. Also $\Omega$ ein
Grundraum, $\F \subseteq 2^\Omega$ die beobachtbaren Ereignisse und $P$ ein
Wahrscheinlichkeitsmass auf $\F$. Eine (reelwertige) Zufallsvariable auf
$\Omega$ ist eine messbare Funktion $X : \Omega \mapsto \R$. Das bedeutet, dass
die Menge $\{X \leq t\} = \{\omega : X (\omega) \leq t\}$ für jedes $t$ ein
beobachtbares Ereigniss sein muss.
\subsection*{Verteilungsfunktion}
Die Verteilungsfunktion von $X$ ist die Abbildung $F_X : \R \mapsto [0, 1]$:
\begin{align*}
  t \mapsto F_X (t) := P[X \leq t] := P[\{\omega : X (\omega) \leq t\}]
\end{align*}
und hat die Eigenschaften:
\begin{itemize}
  \item $F_X$ ist wachsend und rechtsstetig. Das bedeutet,
        dass $F_X (s) \leq F_X (t)$ für $s \leq t$ gilt und $F_X (u) \ra F_X (t)$
        für $u \ra t$ mit $u > t$
  \item $\lim_{t \ra - \infty} F_X (t) = 0$ und $\lim_{t \ra + \infty} F_X (t) = 1$
\end{itemize}
\BoxStart{}
\subsection*{Beispiel: Verteilungsfunktion}
Um zu verhindern, dass ein Gerät infolge eines defekten Halbleiters längere Zeit ausfällt, 
werden zwei identische, parallel geschaltete Halbleiter zu einem Bauteil zusammengefasst.
Eine Kontrolllampe leuchtet auf, wenn einer der beiden Halbleiter ausgefallen ist. 
Wir nehmen an, dass die Lebensdauern der Halbleiter unabhängige, exponentialverteilte Zufallsvariablen mit Erwartungswert 60 Tage sind.
Wie ist die Zeit, nach der die Kontrolllampe ufleuchtet, verteilt? \\

Seien \(H_1\) und \(H_2\) die Lebensdauern der entsprechenden Halbleiter. Nach Voraussetzung sind \(H_1\) und \(H_2\) i.i.d. und 
Exp(\(\lambda\))-verteilt mit \(\lambda = \frac{1}{60}\). Sei \(T\) die Zeit, nach der die Kontrolllampe aufleuchtet; 
also ist \(T = \min\{H_1, H_2\}\).
Die Verteilungsfunktion von \(T\) ist gegeben durch

\begin{align*}
  F_T(t)  &= P [T \leq t] = P [\min\{H_1, H_2\} \leq t]\\
          &= 1 - P [\min\{H_1, H_2\} > t] = 1 - P [H_1 > t, H_2 > t]\\
          &= 1 - P [H_1 > t]P [H_2 > t]\\
          &= (1 - \exp(-2\lambda t)) \mathbf{1}_{[0, \infty)}(t)]
\end{align*}

d.h. \(T\) ist wieder exponentialverteilt mit Parameter \(2\lambda = \frac{1}{30}\).

\BoxEnd{}
\subsection*{Dichtefunktion}
Das Analogon der Gewichtsfunktion im Diskreten Fall. Eine Zufallsvariable $X$
mit Verteilungsfunktion $F_X (t) = P[X \leq t]$ heisst (absolut) stetig mit
Dichte (funktion) $f_X : \R \mapsto [0, \infty)$, falls gilt:
\begin{align*}
  F_X (t) = \int_{-\infty}^t f_X (s) \; dx &  & \text{für alle } t \in \R
\end{align*}
und hat die Eigenschaften:
\begin{itemize}
  \item $f_X \geq 0$ und $f_X = 0$ ausserhalb von $\W (X)$.
  \item $\int_{-\infty}^\infty f_X (s) \; ds = 1$;
        das folgt aus $\lim_{t \ra + \infty} F_X (t) = 1$
\end{itemize}
\subsection*{Gleichverteilung}
Die Gleichverteilung auf dem Intervall $[a, b]$ ist ein Modell für die
Zufällige Wahl eines Punktes in $[a, b]$. Die zugehörige Zufallsvariable $X$
hat den Wertebereich $\W (X) = [a, b]$, sowie
\begin{align*}
  f_X (t) & =
  \begin{cases}
    \frac{1}{b-a} & \text{für } a \leq t \leq b \\
    0             & \text{sonst.}
  \end{cases} \\
  F_X (t) & =
  \begin{cases}
    0               & \text{für } t < a           \\
    \frac{t-a}{b-a} & \text{für } a \leq t \leq b \\
    1               & \text{für } t > b.
  \end{cases}
\end{align*}
wir schreiben kurz $X \sim U (a, b)$.
\begin{align*}
  E[X] = \frac{a + b}{2} &  & Var[X] = \frac{{(b - a)}^2}{12}
\end{align*}
\subsection*{Exponentialverteilung}
Die Exponentialverteilung mit Parameter $\lambda > 0$ ist das stetige Analogon
der Geometrischen Verteilung. Die zugehörige Zufallsvariable $X$ hat $\W (X) =
  [0, \infty)$, Dichte und Verteilungsfunktion:
\begin{align*}
  f_X (t) & =
  \begin{cases}
    \lambda \cdot e^{-\lambda t} & \text{für } t \geq 0 \\
    0                            & \text{für }t < 0
  \end{cases} \\
  F_X (t) & =
  \int_{-\infty}^t f_X (s) \; ds =
  \begin{cases}
    1 - e^{-\lambda t} & \text{für } t \geq 0 \\
    0                  & \text{für }t < 0
  \end{cases}
\end{align*}
wir schreiben kurz $X \sim Exp (\lambda)$. Weiter ist
die Funktion Gedächtsnislos, dh. $\cond{X > t + s}{X > s} = P[X > t]$.
\begin{align*}
  E[X] = \frac{1}{\lambda} &  & Var[X] = \frac{1}{\lambda^2}
\end{align*}
\subsection*{Normalverteilung}
Die Normalverteilung hat zwei Parameter: $\mu \in \R$ und $\sigma^2 > 0$. Die
zugehörige Zufallsvariable $X$ hat den Wertebereich $\W (X) = \R$ und die
Dichtefunktion:
\begin{align*}
  f_X (t) = \frac{1}{\sigma \sqrt{2 \pi}} e^{- \frac{{(t - \mu)}^2}{2 \sigma^2}}
   &  & \text{für } t \in \R
\end{align*}
welche symmetrisch um $\mu$ ist. Wir schreiben kurz: $X \sim \Normalverteilt$.
\subsection*{Standard Normalverteilung}
Wichtige Normalverteilung mit $\Standardnormalverteilt$. Weder für die
zugehörige Dichte $\vp (t)$ noch Verteilungsfunktion $\Phi (t)$ gibt es
geschlossene Ausdrücke, aber das Integral
\begin{align*}
  \Phi (t) = \int_{-\infty}^t \vp (s) \; ds =
  \frac{1}{\sqrt{2\pi}} \int_{-\infty}^t e^{-\frac{1}{2} s^2} \; ds
\end{align*}
ist tabelliert. Ist $X \sim \Normalverteilt$, so ist
$\frac{X - \mu}{\sigma} \sim \Standardnormalverteilt$, also:
\begin{align*}
  F_X (t) = P[X \leq t] = P \left[ \frac{X-\mu}{\sigma} \leq \frac{t - \mu}{\sigma} \right] = \Phi \left  ( \frac{t - \mu}{\sigma} \right)
\end{align*}
deshalb genügt es $\Phi$ zu tabellieren.
\begin{align*}
  \Phi (-z) = 1 - \Phi (z)
\end{align*}
\subsection*{Normalapproximation}
Wenn $S_n \sim Bin (n, p)$ dann
\begin{align*}
  S_n \sim_{approx} N (np, np (1-p))
\end{align*}
\subsection*{Erwartungswert}
Ist $X$ stetig mit Dichte $f_X (x)$, so ist der Erwartungswert:
\begin{align*}
  E[X] = \int_{-\infty}^\infty x \cdot f_X (x) \; dx
\end{align*}
sofern das Integral absolut konvergiert. Ist das Integral nicht
absolut konvergent, so existiert der Erwartungswert nicht.
\subsection*{Erwartungswert einer Funktion}
Sei $X$ eine Zufallsvariable und $Y = g (X)$ eine weitere Zufallsvariable. Ist
$X$ stetig mit Dichte $f_X$, so ist
\begin{align*}
  E[Y] = E[g (X)] = \int_{-\infty}^\infty g (x) \cdot f_X (x) \; dx
\end{align*}
\subsection*{Momente \& Absolute Momente}
Sei $X$ eine Zufallsvariable und $p \in \R^+$. Wir definieren:
\begin{itemize}
  \item $p$-te absolute Moment von $X$: $M_p := E[\abs{X}^p]$
  \item falls $M_n < \infty$ für ein $n$, dann ist das $n$-te (rohe) Moment von $X$
        durch $m_n := E[X^n]$ definiert.
  \item Das $n$-te zentralisierte Moment von $X$ ist durch $\mu_n := E[(X - E[X])^n]$
        definiert.
\end{itemize}
Es gilt weiter, dass $M_n < \infty$ für $n \in \N \implies \abs{m_m} \leq M_n$.
\begin{align*}
  M_p                         & = \int_{-\infty}^\infty \abs{x}^p f_X (x) \; dx \\
  m_n                         & = \int_{-\infty}^\infty x^n f_X (x) \; dx       \\
  p \leq q \land M_q < \infty & \implies M_p < \infty
\end{align*}
\subsection*{Gemeinsame Verteilung/Dichte}
Die Gemeinsame Verteilungsfunktion von Zufallsvariablen $\zufallsvariablen$ ist die Abbildung $F: \R^n \mapsto [0, 1]$ mit:
\begin{align*}
  F (x_1, \dots, x_n) & := P[X_1 \leq x_1, \dots, X_n \leq x_n]                                                 \\
                      & = \int_{-\infty}^{x_1} \dots \int_{-\infty}^{x_n} f (t_1, \dots, t_n) \; dt_n \dots t_1
\end{align*}
dann heisst $f (x_1, \dots, x_n)$ die gemeinsame Dichte, welche folgende
Eigenschaften hat:
\begin{itemize}
  \item $f (x_1, \dots, x_n) \geq 0$ und $= 0$ ausserhalb von $\W (\zufallsvariablen)$.
  \item $\int_{-\infty}^\infty \dots \int_{-\infty}^\infty f (t_1, \dots, t_n) \; dt_n \dots t_1 = 1$
  \item $P[ (\zufallsvariablen) \in A] = \int_{ (x_1, \dots, x_n) \in A} f (t_1, \dots, t_n) \; dt_n \dots t_1$ für $A \subseteq \R^n$
\end{itemize}
\subsection*{Randverteilung}
Haben $X, Y$ die Gemeinsame Verteilungsfunktion $F$, so ist die Funktion $F_X:
  \R \mapsto [0, 1]$,
\begin{align*}
  F_X (x) & = P[X \leq x] = P[X \leq x, Y < \infty] = \lim_{y \ra \infty} F (x, y) \\
  f_X (x) & = \int_{-\infty}^\infty f (x, y) \; dy
\end{align*}

Sind $X, Y$ diskrete Zufallsvariablen mit $\W (Y) = \{y_1, y_2, \dots\}$
und gemeinsamer Gewichtsfunktion $p (x, y)$, so ist die Gewichtsfunktion
der Randverteilung von $X$ gegeben durch:
\begin{align*}
  x \mapsto p_X (x) := \sum_{y_i \in \W (X)} P[X = x, Y = y_i]
\end{align*}
\BoxStart{}
\subsection*{Beispiel: Dichtefunktion berechnen}
Seien $X$ und $Y$ zwei unabhängige Zufallsvariablen, beide exponentialverteilt mit
Parameter $\lambda > 0$. Definiere
\[
  U := \frac{X}{X + Y} und V := X + Y
\]
Seien $f_U$ und $f_V$ die zu $U$ und $V$ gehörigen Dichtefunktionen. Berechne
die Dichtefunktion $f_U$ und Verteilungsfunktion $F_U$
\begin{align*} 
  P[U \leq u] &= \lambda^2 \int_0^\infty e^{-\lambda x} \left(\int_0^\infty 1_{\frac{x}{x + y}\leq u}e^{-\lambda y}dy\right)dx\\ 
  &= \lambda^2 \int_0^\infty e^{-\lambda x} \left(\int_0^\infty 1_{x(u^{-1} - 1)\leq y}e^{-\lambda y}dy\right)dx\\ 
  &= \lambda \int_0^\infty e^{-\lambda x} \left(\int_{x(u^{-1} -1)}^\infty \lambda e^{-\lambda y}dy\right)dx\\ 
  &= \lambda \int_0^\infty e^{-\lambda x} e^{-\lambda x (u^{-1} -1)}dx\\ 
  &= \lambda \int_0^\infty e^{-\lambda u^-1 x}dx\\ 
  &= u 
\end{align*}

\BoxEnd{}
\BoxStart{}
\subsection*{Beispiel: Randverteilung, gemeinsame Dichte}
Man wählt zufällig einen Punkt $P = (U, V)$ in dem Gebiet $D$. Die gemeinsame Dichte von $(U, V)$
\[
  f_{U, V} (u, v) =
  \begin{cases}
    c & \text{falls } (u, v) \in D \\
    0 & \text{sonst}               \\
  \end{cases}
\]
\begin{center}
  \includegraphics[width=0.2\textwidth]{dichte_aufgabe.png}
\end{center}
\begin{itemize}[noitemsep,topsep=0pt,parsep=0pt,partopsep=0pt]  \item bestimme Konstnte $c$
  \item bestimme Randverteiungsfunktion von U und Randdichtefunktion $f_U(u)$.
  \item Sind $U$ und $V$ unabhängig?
\end{itemize}

\begin{align*}
  D =  & \{(a, b) \in \mathbb{R}^2 : -1 \leq a \leq 1 \text{ and } -2 \leq b \leq 2\} \\
  \cup & \{(a, b) \in \mathbb{R}^2 : 1 \leq a \leq 2 \text{ and } -1 \leq b \leq 1\}  \\
       & f_{U,V}(u, v) =
  \begin{cases}
    1/10 & \text{if } (u, v) \in D \\
    0    & \text{otherwise}
  \end{cases}
\end{align*}
For $u < -1$, $F_U(u) = P(U \leq u) = 0$.
For $-1 \leq u \leq 1$, $F_U(u) = P(U \leq u) = $
\[
  \int_u^{-1} \int_{-2}^{2} \frac{1}{10} \mathbf{1}_D(u, v) \, dv \, ds = \frac{4(u + 1)}{10}
\]

For $1 \leq u \leq 2$, $F_U(u) = P(U \leq u) =$
\[
  \int_1^{-1} \int_{-2}^{2} \frac{1}{10} \mathbf{1}_D(u, v) \, dv \, ds + \int_u^1 \int_{-1}^{1} \frac{1}{10} \mathbf{1}_D(u, v) \, dv \, ds
\]
\[
  = \frac{8}{10} + \frac{2(u - 1)}{10}
\]

and $F_U(u) = 1$ for $u > 2$.

\[
  f_U(u) = \begin{cases}
    \frac{4}{10} & \text{if } -1 \leq u \leq 1 \\
    \frac{2}{10} & \text{if } 1 \leq u \leq 2  \\
    0            & \text{otherwise}
  \end{cases}
\]

\[
  f_V(v) = \int_{-\infty}^{\infty} f_{U,V}(u,v) \, du =
\]
\[
  \frac{1}{2}\left(2[v \in [-2,2]] + [v \in [-1,1]]\right)
\]

If U and V are independent, then for all u and v:

\[
  f_U(u) \cdot f_V(v) = f_{U,V}(u,v)
\]

However, we have \(f_{U,V}(2,2) = 0\) and \(f_U(2) \cdot f_V(2) = \frac{1}{5}
\cdot \frac{1}{5} \neq 0\).

Therefore, U and V are not independent.

\BoxEnd{}
\subsection*{Unabhängigkeit}
Die Zufallsvariablen $\zufallsvariablen$ heissen unabhängig, falls gilt
(äquivalent):
\begin{align*}
  F (x_1, \dots, x_n) = F_{X_1} (x_1) \cdot \hdots \cdot F_{X_n} (X_n) \\
  f (x_1, \dots, x_n) = f_{X_1} (x_1) \cdot \hdots \cdot f_{X_n} (X_n)
\end{align*}
für alle $x_1, \dots, x_n$.
\subsection*{Bedingte Verteilungen}
Es gilt:
\begin{align*}
  f_{X_1 \; | \; X_2} (x_1 \; | \; x_2) & = \frac{f_{X_1,  X_2} (x_1,  x_2)}{f_{X_2} (x_2)}              \\
  \cond{Y > t}{Y < a}                   & = \frac{P[t < Y < a]}{P[Y < a]}                                \\
  E[X_1 \; | \; X_2]                    & = \int x_1 \cdot f_{x_1 \; | \; x_2} (x_1 \; | \; x_2) \; dx_1
\end{align*}
\subsection*{Summen von Zufallsvariablen}
Sei $Z = X + Y$ eine Zufallsvariable mit:
\begin{align*}
  F_Z (z) & = P[Z \leq z] = P[X + Y \leq z]                                     \\
          & = \int_{-\infty}^\infty \int_{-\infty}^{z - x} f (x, y )\; dy \, dx \\
  f_Z (z) & = \int_{-\infty}^\infty f (z - y, y) \; dy
\end{align*}
\subsection*{Transformationen}
Sei $X$ eine Zufallsvariable mit Verteilung und Dichte. Sei $g: \R \mapsto \R$
eine messbare Funktion. Betrachte nun $Y = g (X)$, wir suchen Verteilung und
Dichte von $Y$:
\begin{align*}
  F_Y (t) & = P[Y \leq t] = P[g (Y) \leq t] = \int_{A_g} f_X (s) \; ds \\
  A_g     & := \{s \in \R \; | \; g (s) \leq t\}
\end{align*}
Wobei man die Dichte durch ableiten der Verteilung erhält.
\subsection*{Anwendung von Transformationen}
Sei $F$ eine stetige und streng monoton wachsende Verteilungsfunktion mit
Unkehrfunktion $F^{-1}$. Ist $X \sim \mathcal{U} (0, 1)$ und $Y = F^{-1} (X)$,
so hat $Y$ gerade die Verteilungsfunktion $F$:
\begin{align*}
  F_Y (t) & = P[Y \leq t] = P[F^{-1} (X) \leq t] \\
          & = P[X \leq F (t)] = F (t)
\end{align*}
Mit der Substitution
\begin{align*}
  \phi(X) & = Y                                              \\
  X       & = \phi^{-1}(y)                                   \\
  f_Y(y)  & = f_X(\phi^{-1}(y))|\text{det }J_{\phi^{-1}}(y)| \\
\end{align*}
\BoxStart{}
\subsection*{Beispiel: Transformation}
Sei $X$ eine Zufallsvariable mit Dichte $f_X(x), x \in \R$ und sei $Y = e^X$. Was ist die Dichte $f_Y(y), y > 0$ der Zufallsvariable $Y$?
\begin{align*}
  Y             & = e^X                          \\
  X             & = \ln Y                        \\
  \frac{dx}{dy} & = \frac{1}{y}                  \\
  f_Y(y)        & = f_X(\ln y) \cdot \frac{1}{y} \\
\end{align*}
\BoxEnd{}

\subsection*{Markov Ungleichung}
Sei $X$ eine Zufallsvariable und ferner $g : \W (X) \mapsto [0, \infty)$ eine
wachsende Funktion. Für jedes $c \in \R$ mit $g (c) > 0$ git dann:
\begin{align*}
  P[X \geq c] \leq \frac{E[g (X)]}{g (c)}
\end{align*}
\subsection*{Chebyshev-Ungleichung}
Sei $Y$ eine Zufallsvariable mit endlicher Varianz. Für jedes $b > 0$ git dann:
\begin{align*}
  P[\abs{Y - E[Y]} \geq b] \leq \frac{Var[Y]}{b^2}
\end{align*}
\subsection*{Momenterzeugende Funktion}
Die Momenterzeugende Funktion einer Zufallsvariable $X$ ist:
\begin{align*}
  M_X (t) := E[e^{t \cdot X}] &  & \text{für } t \in \R
\end{align*}

\BoxStart{}
\subsection*{Beispiel: momenterzeugende Funktion}
Sei \(X\) eine exponentialverteilte Zufallsvariable mit Parameter \(\lambda\), d.h. \(X \sim \text{Exp}(\lambda)\). Berechnen Sie die momenterzeugende Funktion \(M_X(t)\).

\begin{align*}
  M_X(t) &= E[e^{tX}] \\
         &= \int_{0}^{\infty} e^{tx} \lambda e^{-\lambda x} \, dx \\
         &= \lambda \int_{0}^{\infty} e^{-(\lambda - t)x} \, dx = -\frac{\lambda}{\lambda - t} e^{-(\lambda - t)x} \bigg|_{0}^{\infty} \\
         &\begin{aligned}
           &= \begin{cases}
                \frac{\lambda}{\lambda - t}, & \text{falls } t < \lambda, \\
                \infty, & \text{falls } t \geq \lambda.
              \end{cases}
           \end{aligned}
\end{align*}
\BoxEnd{}

\subsection*{Grosse Summenabweichung}
Seien $\zufallsvariablen$ i.i.d. Zufallsvariablen, für welche die
Momenterzeugende Funktion $M_X (t)$ für alle $t \in \R$ endlich ist. Für jedes
$b \in \R$ gilt dann:
\begin{align*}
  P[S_n \geq b] \leq \exp \left( \inf_{t \in \R}  ( n \cdot \log M_X (t) - t \cdot b ) \right)
\end{align*}

\subsection*{Chernoff Schranken}
Seien $\zufallsvariablen$ unabhängig mit $X_i \sim Be (p)$ und $X = \Sn X_i$.
Sei $\mu_n := E[X] = \Sn p_i$ und $\delta > 0$. Dann gilt:
Suppose $0 < \delta$, then
\[ P(X \geq (1 + \delta)\mu) \leq e^{-\frac{\delta^2\mu}{2+\delta}}, \]
and
\[ P(X \leq (1 - \delta)\mu) \leq e^{-\frac{\delta^2\mu}{2}}. \]

\BoxStart{}
\subsection*{Beispiel: Chernoff Schranke}
Suppose you toss a fair coin 200 times. How likely is it that you see
at least 120 heads?
The Chernoff bound says

\begin{align*}
  P(X \geq 120) &= P(X \geq (1 + \frac{20}{100}) \cdot 100) \leq e^{-\frac{1}{5^2} \cdot \frac{100}{2+\frac{1}{5}} \cdot 100} \\
                &= e^{-\frac{20}{6}} = 0.0356
\end{align*}

\BoxEnd{}
\subsection*{Schwaches Gesetz der grossen Zahlen}
Sei $X_1, X_2, \dots$ eine Folge von unabhängigen Zufallsvariablen, die alle
den gleichen Erwartungswert $E[X_i] = \mu$ und die gleiche Varianz $Var[X_i] =
  \sigma^2$ haben. Sei
\begin{align*}
  \overline{X}_n = \frac{1}{n} S_n = \frac{1}{n} \Sn X_i
\end{align*}
Dann konvergiert $\overline{X}_n$ für $n \ra \infty$ in Wahrscheinlichkeit/
stochastisch gegen $\mu = E[X_i]$, d.h.:
\begin{align*}
  P \left[ \abs{\overline{X}_n - \mu} > \varepsilon \right] \underset{n \ra \infty}{\longrightarrow} 0
   &  & \text{für jedes } \varepsilon > 0
\end{align*}
(Statt unabhängig genügt auch $Cov (X_i, X_k) = 0$ für $i \neq k$)
\subsection*{Starkes Gesetz der grossen Zahlen}
Sei $X_1, X_2, \dots$ eine Folge von unabhängigen Zufallsvariablen, die alle
dieselbe Verteilung haben, und ihr Erwartungswert $\mu = E[X_i]$ sei endlich.
Für:
\begin{align*}
  \overline{X}_n = \frac{1}{n} S_n = \frac{1}{n} \Sn X_i
\end{align*}
gilt dann
\begin{align*}
  \overline{X}_n \underset{n \ra \infty}{\longrightarrow} \mu &  & \text{P-fastsicher}
\end{align*}
d.h.:
\begin{align*}
  P \left[ \left\{ \omega \in \Omega : \overline{X}_n (\omega) \underset{n \ra \infty}{\longrightarrow} \mu \right\} \right] = 1
\end{align*}
\subsection*{i.i.d. / u.i.v.}
Independent identically distributed
\subsection*{Zentraler Grenzwertsatz}
Sei $X_1, X_2, \dots$ eine Folge von i.i.d. Zufallsvariablen mit $E[X_i] = \mu$
und $Var[X_i] = \sigma^2$. Für die Summe $S_n = \Sn X_i$ gilt dann:
\begin{align*}
  \lim_{n \ra \infty} P \left[ \frac{S_n - n \cdot \mu}{\sigma \sqrt{n}} \leq x \right] = \Phi (x)
   &  & \text{für alle } x \in \R
\end{align*}
wobei $\Phi$ die Verteilungsfunktion von $\Standardnormalverteilt$ ist.
\input{Kapitel6}
\input{Kapitel7}
\input{Kapitel8}
\section{Tabellen}
\subsection{Ableitung, Integration}
Es gilt:
\begin{itemize}
  \item \textbf{Summenregel} $ (f (x) + g (x))' = f' (x) + g' (x)$
  \item \textbf{Produktregel} $ (f (x) \cdot g (x))' = f' (x) \cdot g (x) + f (x) \cdot g' (x)$
  \item \textbf{Quotientenregel} $\left( \frac{f (x)}{g (x)} \right)' = \frac{f' (x) \cdot g (x) - f (x) \cdot g' (x)}{g^2 (x)}$ wenn $g (x) \neq 0$
  \item \textbf{Kettenregel} $ (f (g (x)))' = f' (g (x)) \cdot g' (x)$
  \item \textbf{Partielle Integration} $\int_a^b f' (x) \cdot g (x) \; dx = [f (x) \cdot g (x)]_a^b - \int_a^b f (x) \cdot g' (x) \; dx$
  \item \textbf{Substitution} $\int_{\varphi (a)}^{\varphi (b)} f (x) \; dx = \int_a^b f (\varphi (t)) \cdot \varphi' (t) \; dt$
  \item $a+c, b+c \in I: \quad \int_a^b f (t + c) \; dt = \int_{a+c}^{b+c} f (x) \; dx$
  \item \textbf{Logarithmus} $\int \frac{f' (t)}{f (t)} \; dt = \log (\abs{f (x)})$
\end{itemize}
\BoxStart{}
\subsection{Beispiel: Substitution}
\begin{align*}
  \int \cos (x^2) 2x \; dx &  & u = x^2                              \\
  \int \cos (u) du         &  & \frac{du}{dx} = \frac{dx^2}{dx} = 2x
\end{align*}
\BoxEnd{}
\subsection{Ableitungen}
\begin{center}
  \begin{tabularx}{\linewidth}{c>{\centering\arraybackslash}Xc}
    $\mathbf{F(x)}$                        & $\mathbf{f(x)}$          & $\mathbf{f'(x)}$         \\
    %\midrule
    $(x-1)e^x $                            & $xe^x$                   & $(x+1)e^x$               \\
    $\frac{x^{-a+1}}{-a+1}$                & $\frac{1}{x^a}$          & $\frac{a}{x^{a+1}}$      \\
    $\frac{x^{a+1}}{a+1}$                  & $x^a \ (a \ne -1)$       & $a \cdot x^{a-1}$        \\
    $\frac{1}{k \ln(a)}a^{kx}$             & $a^{kx}$                 & $ka^{kx} \ln(a)$         \\
    $\ln |x|$                              & $\frac{1}{x}$            & $-\frac{1}{x^2}$         \\
    $\frac{2}{3}x^{3/2}$                   & $\sqrt{x}$               & $\frac{1}{2\sqrt{x}}$    \\
    $-\cos(x)$                             & $\sin(x)$                & $\cos(x)$                \\
    $ $                                    & $\frac{\sin(x)^2}{2} $   & $\sin(x)\cos(x)$         \\
    $\sin(x)$                              & $\cos(x)$                & $-\sin(x)$               \\
    $\frac{1}{2}(x-\frac{1}{2}\sin(2x))$   & $\sin^2(x)$              & $2 \sin(x)\cos(x)$       \\
    $\tan(x) - x$                          & $\tan(x)^2$              & $2\sec(x)^2 \tan(x)$     \\
    $-\cot(x) - x$                         & $\cot(x)^2$              & $-2 \cot(x) \csc(x)^2$   \\
    $\frac{1}{2}(x + \frac{1}{2}\sin(2x))$ & $\cos^2(x)$              & $-2\sin(x)\cos(x)$       \\
    \multirow{2}*{$-\ln|\cos(x)|$}         & \multirow{2}*{$\tan(x)$} & $\frac{1}{\cos^2(x)}$    \\
                                           &                          & $1 + \tan^2(x)$          \\
    $\cosh(x)$                             & $\sinh(x)$               & $\cosh(x)$               \\
    $\log(\cosh(x))$                       & $\tanh(x)$               & $\frac{1}{\cosh^2(x)}$   \\
    $\ln | \sin(x)|$                       & $\cot(x)$                & $-\frac{1}{\sin^2(x)}$   \\
    $\frac{1}{c} \cdot e^{cx}$             & $e^{cx}$                 & $c \cdot e^{cx}$         \\
    $x(\ln |x| - 1)$                       & $\ln |x|$                & $\frac{1}{x}$            \\
    $\frac{1}{2}(\ln(x))^2$                & $\frac{\ln(x)}{x}$       & $\frac{1 - \ln(x)}{x^2}$ \\
    $\frac{x}{\ln(a)} (\ln|x| -1)$         & $\log_a |x|$             & $\frac{1}{\ln(a)x}$      \\

    %\bottomrule
  \end{tabularx}
\end{center}

%\subsection{Weitere Ableitungen}
\begin{center}
  \begin{tabularx}{\linewidth}{>{\centering\arraybackslash}X>{\centering\arraybackslash}X}

    $\mathbf{F(x)}$                          & $\mathbf{f(x)}$                                \\
    \midrule
    $\arcsin(x) / \arccos(x)$                & $\frac{1 / -1}{\sqrt{1 - x^2}}$                \\
    $\arctan(x)$                             & $\frac{1}{1 + x^2}$                            \\

    $x\arcsin(x) + \sqrt{1 - x^2}$           & $\arcsin(x)$                                   \\
    $x\arccos(x) - \sqrt{1 - x^2}$           & $\arccos(x)$                                   \\
    $x\arctan(x) - \frac{1}{2}\ln(1+x^2)$    & $\arctan(x)$                                   \\
    $\ln(\cosh(x))$                          & $\tanh(x)$                                     \\

    $x^x \ (x > 0)$                          & $x^x \cdot (1 + \ln{x})$                       \\
    $f(x)^{g(x)}$                            & $e^{g(x) ln(f(x))}$                            \\
    $f(x) = cos(\alpha)$                     & $f(x)^n = sin(x + n\frac{\pi}{2})$             \\
    $f(x) = \frac{1}{ax + b}$                & $f(x)^n = (-1)^n * a^n * n! * (ax + b)^{-n+1}$ \\
    $-\ln(\cos(x))$                          & $\tan(x)$                                      \\
    $\ln(\sin(x))$                           & $\cot(x)$                                      \\
    $\ln(\tan(\frac{x}{2}))$                 & $\frac{1}{\sin(x)}$                            \\
    $\ln{\tan(\frac{x}{2} + \frac{\pi}{4})}$ & $\frac{1}{\cos(x)}$                            \\

    \bottomrule
  \end{tabularx}
\end{center}

%\subsection{Integrale}
\begin{center}
  \begin{tabularx}{\linewidth}{>{\centering\arraybackslash}X>{\centering\arraybackslash}X}
    $\mathbf{f(x)}$                        & $\mathbf{F(x)}$                                                  \\
    \midrule
    $\int f'(x) f(x) \, dx$                & $\frac{1}{2}(f(x))^2$                                            \\
    $\int \frac{f'(x)}{f(x)} \, dx$        & $\ln|f(x)|$                                                      \\
    $\int_{-\infty}^\infty e^{-x^2} \, dx$ & $\sqrt{\pi}$                                                     \\
    $\int (ax+b)^n \, dx$                  & $\frac{1}{a(n+1)}(ax+b)^{n+1}$                                   \\
    $\int x(ax+b)^n \, dx$                 & $\frac{(ax+b)^{n+2}}{(n+2)a^2} - \frac{b(ax+b)^{n+1}}{(n+1)a^2}$ \\
    $\int (ax^p+b)^n x^{p-1} \, dx$        & $\frac{(ax^p+b)^{n+1}}{ap(n+1)}$                                 \\
    $\int (ax^p + b)^{-1} x^{p-1} \, dx$   & $\frac{1}{ap} \ln |ax^p + b|$                                    \\
    $\int \frac{ax+b}{cx+d} \, dx$         & $\frac{ax}{c} - \frac{ad-bc}{c^2} \ln |cx +d|$                   \\
    $\int \frac{1}{x^2+a^2} \, dx$         & $\frac{1}{a} \arctan \frac{x}{a}$                                \\
    $\int \frac{1}{x^2 - a^2} \, dx$       & $\frac{1}{2a} \ln\left| \frac{x-a}{x+a} \right|$                 \\
    $\int \sqrt{a^2+x^2} \, dx $           & $\frac{x}{2}f(x) + \frac{a^2}{2}\ln(x+f(x))$                     \\
    \bottomrule
  \end{tabularx}
\end{center}

\includegraphics*[width=\columnwidth]{sin-cos-tan}

\section{Sonstiges}
\[ax^2 + bx + c = 0 \Rightarrow x = \frac{-b \pm \sqrt{b^2 - 4ac}}{2a}\]
\[n! = n \cdot \frac{k=0}{(n - k)!k!}\]
\[\sum_{k=0}^\infty q^k = \frac{1}{1 - q} \quad \text{for} \quad |q| < 1\]
\[\sum_{k=0}^n q^k = \frac{1 - q^{n+1}}{1 - q} \quad \text{for} \quad q \neq 1\]
\[(x + y)^n = \sum_{k=0}^n \binom{n}{k} x^{n-k} y^k\]
\newpage

\input{Kapitel10}

 
\end{multicols*}
\end{document}