\section{Geometry \& Textures}
\greenbf{Ways to encode geometry:} \\
Explicit: Vertex positions are given explicitly $\rightarrow$ good for sampling, bad for testing whether inside or outside object. \\
Implicit: Vertex positions fulfil some equation. $\rightarrow$ good to test inside/outside object, compact description, tough to model complex shapes, finding/sampling all points is expensive.\\
\greenbf{Geometry representations}
\begin{compactitem}
    \item Parametric surfaces (imp): surfaces that are defined by a
parameter space. 
    \item Subdivision surfaces (exp): surfaces that are piecewise linear.
    \item Point set (exp): collection of points that can be
combined to surfaces.
    \item Polygonal Meshes (exp): store the boundary of objects and the connectivity and other attributes (color, normal)
\end{compactitem}

\subsection*{Mesh Datastructure}

\greenbf{Triangle List:} List containing ($v_1, v_2, v_3$) where $v_i$ is the coordinates $\Rightarrow$ easy query, but redundant.\\
\greenbf{Indexed Face Set:} List containing vertex ids and another list of vertices with their coordinates $\Rightarrow$ less storage space.

\subsection*{Texture Mapping}
Enhance details without increasing geometric complexity. Desirable properties: low distortion, bijective mapping, efficiency. \\
\greenbf{Parametrization:} Map $(u,v)$ coordinates of texture to 3D vertex coordinates. E.g. for spheres
$\begin{bmatrix}
    \begin{smallmatrix}
        u \\ v
    \end{smallmatrix}    
\end{bmatrix} \mapsto 
\begin{bmatrix}
    \begin{smallmatrix}
        sin(u)sin(v) \\
        cos(v)\\
        cos(u)sin(v)
    \end{smallmatrix}
\end{bmatrix}$\\
\greenbf{Texture Filtering:} To prevent aliasing, we should apply low pass filter to the texture. \\
\greenbf{Maps:}
\begin{compactitem}
    \item Light map: simulates effect of a local light source
    \item Environment map: render reflective object efficiently
    \item Bump mapping
    \item Normal mapping
    \item Mipmapping
\end{compactitem}
\greenbf{Bump Mapping:}
Perturbs surface normal. Encodes height difference (grayscale) from mesh. Illusion of geometry, but (self-)shadows and silhouette unchanged. \\
\greenbf{Normal mapping:}
Very similar to bump mapping but now stored as $(r,g,b)$ color $\Rightarrow$ directional perturbations. More detailed \\
\greenbf{Mipmapping:}
Store down-sampled versions of a texture using Gaussian Pyramid. Choose resolution based on projected size of triangle. Use linear interpolation between resolutions. Prevents aliasing!\\
\greenbf{Magnification:}
Pixel in texture image maps to area larger than one pixel $\rightarrow$ Jaggies. Can be solved by bilinear interpolation. \\
\greenbf{Minification:}
Pixel in texture image maps to area smaller than one pixel $\rightarrow$ moiré patterns. Solution: mipmapping.