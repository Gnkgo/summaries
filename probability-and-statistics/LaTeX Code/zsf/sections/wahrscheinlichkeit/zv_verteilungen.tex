\section{Zufallsvariablen und Verteilungen}%
\label{sec:zuf[Aallsvariablen_und_verteilungen}
\subsection{Zufallsvariablen}%
\label{sub:zufallsvariablen}

\begin{definition}{Zufallsvariable}
	Sei $(\Omega, \F , \pr)$. Eine \emph{Zufallsvariable} ist eine Abbildung $X : \Omega \rightarrow \R$ sodass für alle
	$a \in \R$
	\begin{equation*}
		\{ \omega \i\in \Omega : X(\omega) \leq a \} \in \F
	\end{equation*}
\end{definition}
\begin{definition}{Verteilungsfunktion}
	von $X$. $F_X : \R \rightarrow [0,1]$:
	\begin{equation*}
		\forall a \in \R \quad F_X (a) = \pr [X \leq a]
	\end{equation*}
\end{definition}
Zufallsexp: Werfe zwei Würfel. $\Omega = \{1, \ldots, 6\}^2$, $\F = \powset (\Omega)$ und $\pr [(\omega_1 ,\omega_2)] =
1/36, \forall \omega = (\omega_1, \omega_2) \in \Omega$.
\begin{equation*}
	X : 
	\begin{cases}
		\Omega & \rightarrow \R\\
		(\omega_1 , \omega_2) & \mapsto \omega_1
	\end{cases}
	\quad
	X^2 : 
	\begin{cases}
		\Omega & \rightarrow \R\\
		(\omega_1 , \omega_2) & \mapsto \omega_1^2
	\end{cases}
	\quad
	S : 
	\begin{cases}
		\Omega & \rightarrow \R\\
		(\omega_1 , \omega_2) & \mapsto \omega_1 + \omega_2
	\end{cases}
\end{equation*}
\begin{center}
	\begin{tikzpicture}[
		]
		\begin{groupplot}[
			group style = {
				group size = 2 by 1,
				horizontal sep = 15mm
			},
			height = 40mm,
			width = 45mm,
			xmin=0,
			xmax=6.5,
			ytick={0, 1/6, 2/6, 3/6, 4/6, 5/6, 1},
			yticklabels={$0$, $1/6$, $2/6$, $3/6$, $4/6$, $5/6$, $1$},
		]
			\nextgroupplot[
				title=$X$,
				xlabel=$x$,
				ylabel=$F_X (x)$
			]
				\addplot[thick, color=lightblue] coordinates {(0,0) (1,0)};
				\addplot[thick, color=lightblue] coordinates {(1,1/6) (2,1/6)};
				\addplot[thick, color=lightblue] coordinates {(2,2/6) (3,2/6)};
				\addplot[thick, color=lightblue] coordinates {(3,3/6) (4,3/6)};
				\addplot[thick, color=lightblue] coordinates {(4,4/6) (5,4/6)};
				\addplot[thick, color=lightblue] coordinates {(5,5/6) (6,5/6)};
				\addplot[thick, color=lightblue] coordinates {(6,1) (6.5,1)};

				\addplot[color=lightblue, incmark] coordinates {(1, 1/6) (2,2/6) (3,3/6) (4,4/6) (5, 5/6) (6,1)};
				\addplot[color=lightblue, exmark] coordinates
					{(1, 0) (2,1/6) (3,2/6) (4,3/6) (5, 4/6) (6,5/6)};
			\nextgroupplot[
				title=$X^2$,
				xlabel=$x$,
				ylabel=$F_{X^2}(x)$,
				xtick={0,1,2,3,4,5,6},
				xticklabels={$0$, $1$, $4$, $9$, $16$, $25$, $36$}
			]
				\addplot[thick, color=lightblue] coordinates {(0,0) (1,0)};
				\addplot[thick, color=lightblue] coordinates {(1,1/6) (2,1/6)};
				\addplot[thick, color=lightblue] coordinates {(2,2/6) (3,2/6)};
				\addplot[thick, color=lightblue] coordinates {(3,3/6) (4,3/6)};
				\addplot[thick, color=lightblue] coordinates {(4,4/6) (5,4/6)};
				\addplot[thick, color=lightblue] coordinates {(5,5/6) (6,5/6)};
				\addplot[thick, color=lightblue] coordinates {(6,1) (6.5,1)};

				\addplot[color=lightblue, incmark] coordinates {(1, 1/6) (2,2/6) (3,3/6) (4,4/6) (5, 5/6) (6,1)};
				\addplot[color=lightblue, exmark] coordinates
					{(1, 0) (2,1/6) (3,2/6) (4,3/6) (5, 4/6) (6,5/6)};
		\end{groupplot}
	\end{tikzpicture}
\end{center}
\begin{center}
	\begin{tikzpicture}[]
		\begin{axis}[
				height=40mm,
				width=80mm,
				title=$S$,
				xmin=0,
				xmax=12.5,
				xlabel=$x$,
				ylabel=$F_S (x)$,
				xtick={0,1,2,3,4,5,6,7,8,9,10,11,12},
				ytick={0, 1/6, 2/6, 3/6, 4/6, 5/6, 1},
				yticklabels={$0$, $1/6$, $2/6$, $3/6$, $4/6$, $5/6$, $1$},
			]
			\addplot[thick, color=lightblue] coordinates {(0,0) (2,0)};
			\addplot[thick, color=lightblue] coordinates {(2,1/36) (3,1/36)};
			\addplot[thick, color=lightblue] coordinates {(3,3/36) (4,3/36)};
			\addplot[thick, color=lightblue] coordinates {(4,6/36) (5,6/36)};
			\addplot[thick, color=lightblue] coordinates {(5,10/36) (6,10/36)};
			\addplot[thick, color=lightblue] coordinates {(6,15/36) (7,15/36)};
			\addplot[thick, color=lightblue] coordinates {(7,21/36) (8,21/36)};
			\addplot[thick, color=lightblue] coordinates {(8,26/36) (9,26/36)};
			\addplot[thick, color=lightblue] coordinates {(9,30/36) (10,30/36)};
			\addplot[thick, color=lightblue] coordinates {(10,33/36) (11,33/36)};
			\addplot[thick, color=lightblue] coordinates {(11,35/36) (12,35/36)};
			\addplot[thick, color=lightblue] coordinates {(12,1) (12.5,1)};

			\addplot[color=lightblue, exmark] coordinates {(2,0) (3, 1/36) (4, 3/36) (5, 6/36) (6, 10/36) (7, 15/36)
				(8,21/36) (9, 26/36) (10, 30/36) (11, 33/36) (12, 35/36)};
			\addplot[color=lightblue, incmark] coordinates {(2, 1/36) (3, 3/36) (4, 6/36) (5, 10/36) (6, 15/36)
				(7,21/36) (8, 26/36) (9, 30/36) (10, 33/36) (11, 35/36) (12,1)};
		\end{axis}
	\end{tikzpicture}
\end{center}
\begin{center}
	\begin{minipage}{0.55\linewidth}
		\begin{center}
			\begin{tikzpicture}[]
				\begin{axis}[
						height=40mm,
						width=60mm,
						title=$F_X(x)$ wobei $X$ weder stetig noch diskret,
						axis lines=middle,
						xtick={-2, -1, 0,1,2,3,4},
						ytick={0, 0.25, 0.5, 0.75, 1},
						xmin=-2.5,
						xmax=4.5,
						ymax=1.25,
						ymin=-0.05,
						xlabel=$x$,
						ylabel=$F_X (x)$,
						clip=false
					]
					\addplot[thick, color=lightblue] coordinates {(-2.5,0) (-1,0) (1,0.5)};
					\addplot[thick, color=lightblue] coordinates {(1,0.75) (3,0.75)};
					\addplot[thick, color=lightblue] coordinates {(3,1) (4.5,1)};

					\addplot[color=lightblue, exmark] coordinates {(1, 0.5) (3, 0.75)};
					\addplot[color=lightblue, incmark] coordinates {(1, 0.75) (3, 1)};

					\node [] (pzero) at (axis cs: 3, 0.25) {$\pr [X = 0] = 0$};
					\draw [->, overlaycolor, thick] (pzero) -- (axis cs: 0, 0.25);

					\draw[decorate, decoration={brace, mirror}, overlaycolor, thick] (axis cs: 3, 0.75) -- (axis cs: 3, 1) node[black,
						midway, right] {$\pr [X = 3] = 1/4$};

				\end{axis}
			\end{tikzpicture}
		\end{center}
	\end{minipage}
	\hfill
	\begin{minipage}{0.4\linewidth}
	\begin{gather*}
		\pr [0 \leq X \leq 1] \\
		= \pr [0 \leq X < 1] + \pr [X = 1]  \\
		= 1/4 + 1/4 \\
		= 1/2
	\end{gather*}
	\end{minipage}
\end{center}

\begin{prop}
	Seien $a < b$ reelle Zahlen
	\begin{equation*}
		\pr [a < X \leq b] = F(b) - F(a)
	\end{equation*}
\end{prop}
\begin{theorem}{Eigenschaften Verteilungsfunktion}
	Sei $X$ eine Zufallsvariable auf $(\Omega, \F , \pr)$. Die Verteilungsfunktion $F = F_X : \R \rightarrow [0,1]$ von
	$X$ erfüllt:
	\begin{properties}
		\item $F$ ist monoton wachsend.
		\item $F$ ist rechtsseitig stetig\footnote{$F(a) = \lim_{h \downarrow 0} F(a+h)$ für jedes $a \in \R$.}.
		\item $\lim_{a \to - \infty} F(a) = 0$ und $\lim_{a \to \infty} F(a) = 1$
	\end{properties}
\end{theorem}


\subsection{Unabhängigkeit}%
\label{sub:unabhangigkeit}

\begin{definition}{Unabhängig}
	Seien $X_1 , \ldots , X_n$ $n$ Zufallsvariablen auf einem $(\Omega, \F , \pr)$. $X_1 , \ldots X_n$ heissen
	\emph{unabhängig} falls
	\begin{equation*}
		\forall x_1 , \ldots , x_n \in \R \quad \pr [X_1 \leq x_1, \ldots , X_n \leq x_n] = \pr [X_1 \leq x_1] \cdot
		\ldots  \cdot \pr [X_n \leq x_n]
	\end{equation*}
\end{definition}
Man kann zeigen, dass ZV $X_{1}, \ldots, X_{n}$ genau dann unabhängig sind, wenn:
\begin{center}
	$\forall I_1 \subset \R , \ldots , I_n \subset R: \{X_1 \in \R\}, \ldots , \{X_n \in \R\}$ unabhängig
\end{center}
gilt. Wobei $I_k$ Intervalle sind.
\begin{tprop}{Gruppierung}
	Seien $X_1 , \ldots , X_n$ $n$ unabhängige ZV und $\phi_1, \ldots , \phi_k$ beliebige Funktionen, dann sind
	\begin{equation*}
		Y_1 = \phi_1 (X_1 , \ldots , X_{i_1}) , \ldots , Y_k = \phi_k (X_{i_{k-1}}, \ldots , X_{i_k})
	\end{equation*}
	\textbf{unabhängig}
\end{tprop}
\begin{definition}{Unabhängig 2}
	Eine unendliche Folge von ZV $X_1 , X_2 , \ldots$ sind:
	\begin{itemize}
		\item \emph{unabhängig}, falls $X_1 , \ldots , X_n$ für jedes $n$ unabhängig sind.
		\item \emph{unabhängig und gleich verteilt} (\iid), falls sie unabhängig sind und die selbe
			Verteilungsfunktion haben. $\forall i,j ~ F_{X_i} = F_{X_j}$
	\end{itemize}
\end{definition}


\subsection{Transformation Zufallsvariablen}%
\label{sub:transformation_zufallsvariablen}

Um ZV wie reelle Zahlen zu verwenden können wir sie mit $\phi : \R \rightarrow \R$ transformieren.
\begin{equation*}
	\phi (X) \coloneqq \phi \circ X \qquad
	\begin{array}{ccccc}
		\Omega & \xrightarrow{X} & \R & \xrightarrow{\phi} & \R\\
		\omega & \mapsto & X(\omega) & \mapsto & \phi (X(\omega))
	\end{array}
\end{equation*}


\subsection{Konstruktion ZV}%
\label{sub:konstruktion_zv}

$X_1, X_2, \ldots$ sind \iid $\bern (1/2)$ falls $\forall x_1 , \ldots , x_n \in \{0,1\}$
\begin{equation*}
	\pr [X_1 = x_1 , \ldots , X_n = x_n] = \frac{1}{2^n} 
\end{equation*}
\begin{theorem}{Kolmogorov}
	Es existiert ein Wahrscheinlichkeitsraum $(\Omega, \F , \pr)$ und eine unendliche Folge von ZV $X_1 ,X_2 , \ldots$
	sodass es sich dabei um eine Folge von \iid Bernoulli ZV mit Parameter $1/2$ handelt.
\end{theorem}
% \begin{definition}[sidebyside, lower separated=false, righthand width = 30mm]{Gleichverteilte ZV}
% 	Eine ZV $U$ ist eine gleich verteilte ZV in $[0,1]$ falls die Verteilungsfunktion gleich
% 	\begin{equation*}
% 		F_U (x) = 
% 		\begin{cases}
% 			0 & x< 0\\
% 			x & 0 \leq x \leq 1\\
% 			1 & x > 1
% 		\end{cases}
% 	\end{equation*}
% 	ist. In diesem Fall schreiben wir $U \sim \mathcal{U} ([0,1])$
% 	\tcblower
% 	\begin{center}
% 		\begin{tikzpicture}[]
% 			\begin{axis}[
% 					height=30mm,
% 					width=45mm,
% 					axis y line=middle,
% 					axis x line=left,
% 					xmin=-0.5,
% 					xmax=1.5,
% 					ymin=0,
% 					ymax=1.2,
% 					xlabel=$x$,
% 					xtick={0,1},
% 					ytick={1},
% 				]
% 				\addplot[dashed] coordinates {(1,1) (1,0)};
% 				\addplot[thick, color=lightblue] coordinates {(-0.5,0) (0,0) (1,1) (1.5, 1)} node[midway, above = 2mm, black] {$F_U (x)$};
% 			\end{axis}
% 		\end{tikzpicture}
% 	\end{center}
% \end{definition}
\begin{prop}
	Die Abbildung $Y : \Omega \rightarrow [0,1]$ definiert durch
	\begin{equation*}
		Y (\omega) = \sum_{n=1}^{\infty} 2^{-n} X_n (\omega)
	\end{equation*}
	ist eine gleich verteilte ZV in $[0,1]$.
\end{prop}
\begin{definition}{Verallgemeinerte Inverse}
	Die generalisierte Inverse von $F$ ist $F^{-1} ]0,1[ \rightarrow \R$ definiert durch
	$ \forall \alpha \in ]0,1[ \quad F^{-1} (\alpha) = \inf \{ x \in \R : F (x) \geq \alpha\} $
\end{definition}
\begin{theorem}{Inv.transformations Sampling}
	Sei $F: \R \rightarrow [0,1]$ eine Verteilungsfunktion. Sei $U$ eine gleich verteilte ZV in $[0,1]$. Dann hat die ZV
	$X = F^{-1} (U)$ die Verteilung $F_X = F$.
\end{theorem}
\begin{example}
	Sei $U \sim \unif([0,1])$. Konstruiere aus $U$ eine $\bern(1/3)$-verteilte ZV $Z$.
	\tcblower
 	Die Vert.funk. $F$ einer $\bern(1/3)$ ZV und ihre verallg. Inverse $F^{-1} : ]0,1[ \rightarrow \R$ sind
 	\begin{equation*}
 		F(a) = 
 		\begin{cases}
 			0, & \text{für}~a < 0\\
 			2/3, & \text{für}~0 \leq a < 0\\
 			1, & \text{für}~a \leq 1\\
 		\end{cases}
		\qquad
		F^{-1}(\alpha) = 
		\begin{cases}
			0, & \text{für} ~0 < \alpha \leq 2/3\\
			1, & \text{für}~2/3 < \alpha < 1
		\end{cases}
 	\end{equation*}
	Es Folgt, dass $Z \coloneqq F^{-1}(U)$ $\bern(1/3)$-verteilt ist.
\end{example}
\begin{example}
	Sei $U \sim \unif([0,1])$. Konstruiere aus $U$ eine gleichverteilte ZV $U'$ in $[-1,2]$. 
	\tcblower
	Für $U' \sim \unif([-1,2])$:
	\begin{equation*}
		F(a) = 
		\begin{cases}
			0, & \text{für}~a < -1\\
			\frac{a + 1}{3}, & \text{für}~ -1 \leq a \leq 2\\
			1, & \text{für} a \geq 2 \\
		\end{cases}
		\qquad
		F^{-1}(\alpha) = 3 \alpha -1
	\end{equation*}
	Es folgt: $U' \coloneqq F^{-1} (U)$ mit $U' \sim \unif([-1,2])$.
\end{example}
\begin{theorem}{}
	Seien $F_1, F_2, \ldots$ eine Folge von Funktionen\footnote{Funktionen müssen die Eigenschaften von Verteilungsfunktion
	erfüllen.} $\R \rightarrow [0,1]$. Dann existiert ein $(\Omega, \F , \pr)$ und eine Folge \iid ZV $X_1 , X_2 , \ldots$
	auf diesem Wahrscheinlichkeitsraum, sodass 
	\begin{itemize}
		\item jedes $i$ $X_i$ hat eine Verteilungsfunktion $F_i$ (also $\pr [X_i \leq x] = F_i (x)$).
		\item $X_1 , X_2 , \ldots$ sind unabhängig.
	\end{itemize}
\end{theorem}
