\section*{Generative Modeling}

Aim to estimate $p(x, y)$ for complex situations using Bayes' rule: $p(x,y) = p(x|y) \cdot p(y)$

\subsection*{Naive Bayes Model}

GM for classification tasks. Assuming for a class label, each feature is independent. This helps estimating $p( x \; | \; y) =\prod_{i=1}^d p(x_i \; | \; y_i)$.

\subsection*{Gaussian Naive Bayes Classifier}

Naive Bayes Model with Gaussian's features. Estimate the parameters via MLE:

MLE for class prior: \color{violet}$p(y) = \hat p_y = \frac{\text{Count}(Y = y)}{n}$

\color{black}MLE for feature distribution:
\begin{scriptsize}
\color{violet}

\begin{align*}
    %p(x_i \; | \; y)    &= \mathcal{N}(x_i; \hat \mu_{y,i}, \sigma^2_{y,i}) \\[-13pt]\\[-4pt]
    %\mu_{y,i}           &= \frac{1}{\text{Count}(Y = y)} \sum_{j \; | \; y_j = y} x_{j,i}\\[-4pt]
    %\sigma^2_{y,i}      &= \frac{1}{\text{Count}(Y = y)} \sum_{j \; | \; y_j = y} (x_{j,i} - \hat \mu_{y, i})^2\\[-4pt]
    P(x_i | y)          &= \frac{Count(X_i = x_i, Y = y)}{Count(Y = y)}
\end{align*}
\color{black}
\end{scriptsize}

Predictions are made by: \\[-15pt]
\color{violet}
\[
y = \argmax{\hat y} \; p(\hat y \; | \; x) = \argmax{\hat y} \; p(\hat y) \cdot \prod_{i=1}^d p(x_i \; | \; \hat y)
\]
\color{Black}
Equivalent to decision rule for bin. class.: \\[-8pt]

\qquad \qquad $y = \sgn \left( \color{Red} \log \frac{p(Y = +1 \; | \; x)}{p(Y = -1 \; | \; x)} \color{Black} \right)$ \\[-3pt]

Where \color{Red}$f(x)$\color{Black} is called the discriminant function. If the conditional independence assumption is violated, the classifier can be overconfident.

\subsection*{Gaussian Bayes Classifier}

No independence assumption, model the features with a multivariant Gaussian $\mathcal{N}(x; \mu_y, \Sigma_y)$:

\quad $\mu_{y} = \frac{1}{\text{Count}(Y = y)} \sum_{j \; | \; y_j = y} x_{j}$

\quad $\Sigma_{y} = \frac{1}{\text{Count}(Y = y)} \sum_{j \; | \; y_j = y} (x_{j} - \hat \mu_{y}) (x_{j} - \hat \mu_{y})^\top$

This is also called the \textbf{quadratic discriminant analysis} (QDA). LDA: $\Sigma_+ = \Sigma_-$, Fisher LDA: $p(y) = \frac{1}{2}$, Outlier detection: $p(x) \leq \tau$.

\subsection*{Avoiding Overfitting}

MLE is prone to overfitting. Avoid this by restricting model class (fewer parameters, e.g. GNB) or using priors (restrict param. values).

\subsection*{Generative vs. Discriminative}

\textbf{Discriminative models}:

$p(y | x)$, can't detect outliers, more robust

\textbf{Generative models}:

$p(x,y)$, can be more powerful (dectect outliers, missing values) if assumptions are met, are typically less robust against outliers
