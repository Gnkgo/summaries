\section{Optical Flow}
Apparent motion of brightness patterns use extracted feature points and commpute their velocity vectors \graytext{projection of 3D velocity vectors on I}\\
\greenbf{Problem:} cannot distingish motion from changing lighting! also estimate observed projected motion field \graytext{normal flow} not always well defined \\
\greenbf{Key assumptions:} \\
Brightness constancy: \graytext{Projection of the same point looks the same in every frame.}\\
Small motion: \graytext{Points do not move far}\\
%Spatial coherence: \graytext{Points move like their neighbors}\\
\greenbf{Brightness constancy constraint:}\\
$I(x + \frac{dx}{dt}\delta t, y + \frac{dy}{dt} \delta t, t + \delta t) = I(x, y, t)
 \quad I = Intensity$\\
Small motion $\rightarrow$ can linearize with Taylor expansion:\\
\graytext{$I(x+u, y + v, t + 1) = I(x,y,t) + I_x u + I_y v + I_t$} \\
\graytext{$ \frac{dI}{dt} = \frac{\partial I}{\partial x} \frac{dx}{dt} + \frac{\partial I}{\partial y} \frac{dy}{dt} + \frac{\partial I}{\partial t} \approx 0$} or shorthand \graytext{$I_x \cdot u + I_y \cdot v + I_t \approx 0$}\\
move $I-t$ on one side, vectorize unknowns. For LK, sum up over a window of pixes\\
\greenbf{Derivation:}
We assume small displacement and use Taylor-Expansion to get:

$ I(x + \frac{dx}{dt}\delta t, y + \frac{dy}{dt}\delta t, t + \delta t) \approx I(x,y,t) + \frac{\partial I}{\partial x}(\frac{dx}{dt}\delta t) + \frac{\partial I}{\partial y}(\frac{dy}{dt}\delta t) + \frac{\partial I}{\partial t}(\delta t) $.

Subtracting the given equation from this equation, we get:

$0 = \frac{\partial I}{\partial x}(\frac{dx}{dt}\delta t) + \frac{\partial I}{\partial y}(\frac{dy}{dt}\delta t) + \frac{\partial I}{\partial t}(\delta t)$,

which can be written as:

$0 = I_x(\frac{dx}{dt}\delta t) + I_y(\frac{dy}{dt}\delta t) + I_t(\delta t)$

Finally, we divide by $\delta t$, and get:

$0 = I_xu + I_yv + I_t$,

as desired.

\greenbf{Sample Exercise:}\\
You have captured a video at 25 frames per second of a car moving at 18 kilometers
per hour. The side of the car is parallel to the image plane and the car is moving
straight. The car is 2.4 meters long, but in your video it is 192 pixels long. Assume
that your optical flow algorithm breaks down for pixel displacements that are larger
than 1 pixel.\\
\graytext{Start with a coarse image $\rightarrow$ compute flow $\rightarrow$ rescale $\rightarrow$ initialize with the last estimate $\rightarrow$ repeat.\\
18 km/h equals 5 meters per second, which equals 20 cm per frame, i.e. $\frac{1}{12}$ of the
length of the car. $\frac{1}{12}$ of 192 pixels is 16 pixels. Going from 16 to 8 to 4 to 2 to 1
leads to 5 levels.}
 
\greenbf{Aperture problem:} The aperture problem refers to the fact that when flow is computed for a point that lies along a linear feature, it is not possible to determine the exact location of the corresponding point in the second image. Thus, it is only possible to determine the flow that is normal to the linear feature. 1 equation, 2 unknowns cannot determine exact location, take normal flow. \includegraphics*[width =0.6 \columnwidth]{jo/feature.png}\\
\subsection*{Horn-Schunck}
Add additional smoothness constraint: \\
\graytext{$e_s = \int \int (u_x^2 + u_y^2 + v_x^2 + v_y^2) dxdy$ $close \approx parallel$}\\
Besides OF constraint: \\
\graytext{$e_c = \int \int (I_x u + I_y v + I_t)^2 dxdy$} Minimize $e_s + \lambda e_c$
\subsection*{Lukas-Kanade}
Works well for textured area, corners. Not for homogeneous areas, edges since $M$ is singular when all gradient vectors point in the same direction.\\
Assume spatial coherence: same displacement for neighbourhood ($N \times M$ window) $\rightarrow$ linear least squares problem:
\graytext{$\begin{bmatrix}
    %\begin{smallmatrix}
        I_x(x_1, y_1) & I_y(x_1, y_1)\\ ... & ... \\
        I_x(x_{NM}, y_{NM}) & I_y(x_{NM}, y_{NM})
    %\end{smallmatrix}
\end{bmatrix}
\begin{bmatrix}
    %\begin{smallmatrix}
        u\\
        v
    %\end{smallmatrix}
\end{bmatrix}
= -\begin{bmatrix}
    %\begin{smallmatrix}
        I_t(x_1, y_1)\\ ...\\
        I_t(x_{NM}, y_{NM})
    %\end{smallmatrix}
\end{bmatrix} \implies 
\begin{bmatrix}
    %\begin{smallmatrix}
        \sum I_x I_x & \sum I_x I_y\\
        \sum I_y I_x & \sum I_y I_y
    %\end{smallmatrix}
\end{bmatrix}
\begin{bmatrix}
    %\begin{smallmatrix}
        u\\
        v
    %\end{smallmatrix}
\end{bmatrix}
= -\begin{bmatrix}
    %\begin{smallmatrix}
        \sum I_x I_t\\
        \sum I_y I_t
    %\end{smallmatrix}
\end{bmatrix}$}\\
\greenbf{When solvable?} $A^TA$ invertible, eigenvalues $\lambda_1, \lambda_2$ large, $\frac{\lambda_1}{\lambda_2}$ small\\
\greenbf{Errors:} motion is large\graytext{(r than a pixel) \\
$\rightarrow$ iterative refinement and coarse-to-fine estimation.} \\
A point does not move like its neighbors \\
\graytext{$\rightarrow$ motion segmentation.}\\
Brightness constancy does not hold:\\
 \graytext{$\rightarrow$ exhaustive neighborhood search with normalized corrolation.}\\
The matrix $M = A^TA$ is singular (for only edges), meaning all gradient vectors point in the same direction.\\
\graytext{$\rightarrow$ No unique solution.}
\greenbf{KLT feature tracker:} to find patches where LSE well-behaved $\rightarrow$ LK-flow\\
\greenbf{Iterative refinement:} Estimate velocity, warp using estimate, refine,...\\
\greenbf{Coarse-to-Fine Estimation:} Image Pyramid. Start small, compute OF, rescale, take larger and initialize with last estimate\\
\greenbf{Applications:} Image stabilization \graytext{(get flow between two frames and warp image using same OF for all pixels s.st. OF close to 0)} frame interpolation, video compression, object tracking, motion segmentation\\
\greenbf{Parametric (Global) Motion models}
They offer more constrained solutions than smoothness (Horn-Schunck) and cover larger area than translational model (LK). An example is: \\
\greenbf{Affine motion:} $I_x(a_1 + a_2x + a_3y) + I_y(a_4 + a_5x + a_6y) + I_t \approx 0$ \\ \includegraphics*[width =\columnwidth]{jo/planes.png}\\
\greenbf{SSD tracking:} For large displacements: match template against each pixel in small area around, match measure can be (normalized) correlation or SSD choose max. as match \graytext{(sub-pixel also possible)}\\
\greenbf{Bayesan Optical Flow:} Some low-level motion illusions can be explained by adding an underlying model to LK-tracking e.g. brightness constancy with noise.
