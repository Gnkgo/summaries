\section{Mathematisches Framework}%
\label{sec:mathematisches_framework}

\subsection{Wahrscheinlichkeitsraum}%
\label{sub:wahrscheinlichkeitsraum}

Grundraum: Menge $\Omega$. $\omega \in \Omega$ heisst Elementarereigniss.

\begin{definition}{$\sigma$-Algebra}
	Subset $\F \subseteq \powset (\Omega)$. Erfüllt
	\begin{properties}
		\item $\Omega \in \F$
		\item $A \in \F \Rightarrow A^c \in \F$
		\item $A_1 , A_2 , \ldots \in \F \Rightarrow \bigcup_{i=1}^\infty A_i \in \F$
		\item $\emptyset \in \F$
		\item $A_1 , A_2 , \ldots \in \F \Rightarrow \bigcap_{i=1}^\infty A_i \in \F$
		\item $A,B \in \F \Rightarrow A \cup B \in \F$
		\item $A,B \in \F \Rightarrow A \cap B \in \F$
	\end{properties}
\end{definition}
\begin{definition}{Wahrsch. Mass}
	Sei $\Omega$ ein Grundraum und $\F$ eine $\sigma$-Algebra. Ein \emph{Wahrsch. Mass} auf $(\Omega, \F)$ ist eine
	Abbildung 
	\begin{align*}
		\pr: \F & \rightarrow [0,1]\\
		A & \mapsto \pr[A]
	\end{align*}
	welche die folgenden Bedingungen erfüllt:
	\begin{properties}
		\item $\pr [\Omega] = 1$
		\item (Zählbare Additivität) $\pr [A] = \sum^{n}_{i=1} \pr[A_i]$ falls $A = \bigcup_{i=1}^\infty A_i$ (disjunkt)
		\item $\pr [\emptyset] = 0$
		\item (Additivität) $\pr [A_1 \cup \ldots \cup A_k] = \pr [A_1] + \ldots + \pr [A_k]$ falls $A_i \cap A_j =
			\emptyset$ für $i \neq j$
		\item $\pr [A^c] = 1 - \pr [A]$
		\item $\pr [A \cup B] = \pr [A] + \pr [B] - \pr [A \cap B]$
	\end{properties}
\end{definition}
\begin{definition}{Wahrscheinlichkeitsraum} 
	$(\Omega, \F, \pr)$
\end{definition}
\begin{tprop}{De Morgan}
	Sei $(A_i)_{i \geq 1}$ eine Folge von beliebigen Mengen
	\begin{equation*}
		\left( \bigcup_{i=1}^\infty A_i \right)^c = \bigcap_{i=1}^\infty (A_i)^c
	\end{equation*}
	Gilt auch für endliche Folgen.
\end{tprop}
Das Laplace-Modell ist definiert durch:
\begin{itemize}
	\item $F = \powset (\Omega)$
	\item $\pr: \F \rightarrow [0,1]$ definiert durch $\forall A \in \F ~ \pr [A] = \frac{|A|}{|\Omega|}$
\end{itemize}

\subsubsection{Wichtige Ungleichungen}%
\label{ssub:wichtige_ungleichungen}

\begin{tprop}{Monotonie}
	Sei $A,B \in \F$ dann $A \subset B \Rightarrow \pr [A] \leq \pr [B]$
\end{tprop}
\begin{tprop}{Unoin Bound}
	Seien $A_1 , A_2 , \ldots$ eine Folge von Ereignissen, dann:
	\begin{equation*}
		\pr \left[ \bigcup_{i=1}^\infty A_i \right] \leq \sum^{\infty}_{i=1} \pr [A_i]
	\end{equation*}
	Unoin Bound gilt auch für endliche Folgen.
\end{tprop}


\subsection{Bedingte Wahrscheinlichkeit}%
\label{sub:bedingte_wahrscheinlichkeit}

\begin{tprop}{Bedingte Wahrsch.}
	Sei $(\Omega, \F , \pr)$. Seien $A,B$ zwei Ereignisse mit $\pr [B] > 0$. $A$ gegeben $B$: 
	\begin{equation*}
		\pr [A | B] = \frac{\pr [A \cap B]}{\pr [B]} 
	\end{equation*}
	$\pr [B|B] = 1$
\end{tprop}
Partition: $\Omega = A_1 \cup \ldots \cup A_n$ (paarweise disjunkt)
\begin{tprop}{Totale Wahrsch.}
	Sei $B_1 , \ldots B_n$ eine Partition von $\Omega$ mit $\pr [B_i] > 0$ für jedes $1 \leq i \leq n$:
	\begin{equation*}
		\forall A \in \F \quad \pr [A] = \sum_{i=1}^{n} \pr [A | B_i] \pr [B_i]
	\end{equation*}
\end{tprop}
\begin{tprop}{Bayes}
	Sei $B_1, \ldots , B_n \in \F$ eine Partition von $\Omega$ mit $\pr [B_i] > 0$ für jedes $i$. Für jedes Ereigniss $A$
	mit $\pr [A]> 0$ gilt:
	\begin{equation*}
		\forall i = 1, \ldots , n \quad \pr [B_i | A] = \frac{\pr [A | B_i] \pr [B_i]}{\sum_{j=1}^{n} \pr [A | B_j] \pr
		[B_j]} 
	\end{equation*}
\end{tprop}


\subsection{Unabhängigkeit}%
\label{sub:unabhangigkeit}

\begin{definition}{Unabhängigkeit zweier Ereignisse}
	Sei $(\Omega, \F , \pr)$. Zwei Ereignisse sind \emph{unabhängig}, falls 
	\begin{equation*}
		\pr [A \cap B] = \pr [A] \cdot \pr[B]
	\end{equation*}
\end{definition}
Sei $A,B \in \F$ zwei Ereignisse mit $\pr [A], \pr [B] > 0$. Dann sind die folgenden äquivalent:
\begin{enumerate*}
	\item $\pr [A \cap B] = \pr [A] \cdot \pr [B]$
	\item $\pr [A | B] = \pr [A]$
	\item $\pr [B | A] = \pr [B]$
\end{enumerate*}
\begin{definition}{Unabhängigkeit}
	Eine Menge von Ereignisse $(A_i)_{i \in I}$ sind unabhängig falls:
	\begin{equation*}
		\forall J \subset I ~\text{endlich}\quad \pr \left[ \bigcup_{j \in J} A_j \right] = \prod_{j \in J} \pr [A_j]
	\end{equation*}
\end{definition}
Falls $A,B,C$ alle unabhängig sind gelten \textbf{alle} 4 Gleichungen:
\begin{align*}
	\pr [A \cap B] &= \pr [A] \cdot \pr [B]\\
	\pr [A \cap C] &= \pr [A] \cdot \pr [C]\\
	\pr [B \cap C] &= \pr [B] \cdot \pr [C]\\
	\pr [A \cap B \cap C] &= \pr [A] \cdot \pr [B] \cdot \pr [C]
\end{align*}
