\documentclass{article}
\usepackage{enumitem}
\usepackage[left=2cm, top=2cm, right=2cm, bottom=3cm]{geometry}

\begin{document}

\section*{Summary of Session 1 (21.09.2023): Making Data, Making Worlds - An Introduction to Data Practices}

\subsection*{The Ontology of Data}
\begin{itemize}
  \item Emphasised that data do not exist independently of the practices that create them.
  \item Introduced the idea that data are "cooked" according to particular recipes, challenging the notion of raw data.
  \item Discussed the influence of institutional factors on scientific practices.
\end{itemize}

\subsection*{The importance of data}
\begin{itemize}
  \item Emphasised the ubiquity of data in contemporary society and the need to understand how data is created and how it shapes our world.
  \item Discussed the impact of data on knowledge production, social order, power relations, ethics and justice.
\end{itemize}

\subsection*{Big Data Knowledge and Data Science}
\begin{itemize}
  \item Discussed the big data "gold rush" and the V4 model (volume, variety, velocity, veracity).
  \item Introduced Data Science as the professional study and analysis of data, highlighting its multidisciplinary nature.
  \item Addressed the assumptions of data science: objectivity, neutrality and the belief that data is a given.
\end{itemize}
\subsection*{Critical data studies}
\begin{itemize}  \item Explores the emergence of critical data studies as a field that challenges the positivist assumptions of data science.
  \item Highlights key perspectives from scholars such as Boyd and Crawford, Dalton and Thatcher, Kitchin and Lauriault, and Iliadis and Russo.
  \item Emphasised the critical mindset that seeks to understand the power and social ordering effects of data.
\end{itemize}

\subsection*{Conclusion}
The session concluded with questions and the announcement of no class for the following week.

\section*{Session 2 Summary (05.10.2023): Making Data, Making Worlds}

\subsection*{Housekeeping}
\begin{itemize}
    \item Lunch is welcome as long as it doesn't disturb others.
    \item 10 min bio-break in each session.
\end{itemize}

\subsection*{Review of Session 1}
\begin{itemize}
    \item Discuss key takeaways in pairs.
    \item Emphasis on putting "science in perspective" and developing a "critical mindset".
\end{itemize}

\subsection*{Understanding Data}
\begin{itemize}
    \item Data are not taken for granted; they are actively produced.
    \item Production involves choices that shape the form and value of data.
    \item Data as a powerful resource for knowledge and social order.
    \item Critical data studies advocate the study of how data is made and remade.
\end{itemize}

\subsection*{Medicine and Data}
\begin{itemize}
    \item Medical practice, especially in hospitals, as an important site for critical data studies.
    \item Hospitals produce massive amounts of data about patients, diseases, therapies, etc.
    \item Emphasis on how doctors create data from ambiguous phenomena, emphasizing the role of knowledge and tools.
\end{itemize}

\subsection*{Case Study: Atherosclerosis (Body Multiple)}
\begin{itemize}
    \item Study of the development and treatment of atherosclerosis in a Dutch hospital.
    \item Emphasis on different ways of knowing, healing, and the impact on coordination, distribution, and inclusion.
\end{itemize}

\subsection*{Atherosclerosis Explained}
\begin{itemize}
    \item Common condition in which plaque builds up in the arteries.
    \item Consequences: reduced blood flow, potential complications such as a heart attack or stroke.
\end{itemize}

\subsection*{Diagnosing Disease - Outpatient vs. Pathology}
\begin{itemize}
    \item Different diagnostic approaches in outpatient clinics and pathology.
    \item Clinical atherosclerosis more important but uncertain.
    \item Emphasis on the importance of reconciling different diagnostic approaches.
\end{itemize}

\subsection*{Medical Practice and Diversity}
\begin{itemize}
    \item Mol's emphasis on the study of medical practices and their effects.
    \item Reality is multiple and varies between different medical practices.
\end{itemize}

\subsection*{Enactment and Performativity}
\begin{itemize}
    \item Introduction to enactment and performativity in the production of knowledge.
    \item Methods are performative, they shape realities.
    \item Ontology is political, influencing which social realities are enacted.
\end{itemize}

\subsection*{Example of Predictive Policing}
\begin{itemize}
    \item Study of predictive policing in Switzerland and Germany.
    \item Focus on prediction of domestic burglaries using different databases and algorithms.
    \item Challenges: reliability, mutation, and decisions affecting real-life consequences.
\end{itemize}

\subsection*{Conclusion}
\begin{itemize}
    \item Summarize the main points of the session.
    \item Encourage questions and further exploration.
\end{itemize}


\section*{Session 3 (12.10.2023) - Classification and Standards}

\subsection*{Rear-view Mirror:}
\begin{itemize}
  \item Data are tools for understanding the world and making decisions.
  \item Different tools for producing data can represent the same phenomenon in different ways.
  \item Data "enact" different realities and influence interventions in the world.
  \item Responsibilities are linked to how we produce data as they shape the world.
\end{itemize}

\subsection*{Let's Talk About Death:}
\begin{itemize}
  \item The importance of gathering knowledge about death is discussed.
  \item Participants work in pairs to discuss how they would gather such knowledge and potential problems.
\end{itemize}

\subsection*{Knowing Death:}
\begin{itemize}
  \item In the 19th century, disease and death shifted from fate to microbial causes.
  \item Public health policy relies on understanding the causes of death to take preventive measures.
\end{itemize}

\subsection*{Tracking Death:}
\begin{itemize}
  \item Challenges in harmonizing the global diagnosis and recording of deaths.
  \item Cultural and medical differences affect what is considered a legitimate cause of death.
\end{itemize}

\subsection*{Towards a Systematic Approach:}
\begin{itemize}
  \item Two main challenges identified: building consensus on causes of death and ways of producing data.
\end{itemize}

\subsection*{The Birth of the ICD:}
\begin{itemize}
  \item The International Statistical Institute adopts the first international classification of diseases in 1893.
  \item The ICD has evolved over time and is currently in its 11th iteration (ICD-11, published in 2018).
\end{itemize}

\subsection*{The Purpose of the ICD:}
\begin{itemize}
  \item To systematically record, analyze, interpret, and compare mortality and morbidity data.
  \item Coding translates diagnoses into alphanumeric codes for easy storage, retrieval, and analysis.
  \item Multiple use cases beyond health statistics.
\end{itemize}

\subsection*{ICD Use Cases Include:}
\begin{itemize}
  \item Cause of death certification and reporting.
  \item Morbidity coding, diagnosis-related grouping, safety surveillance, cancer registries, etc.
\end{itemize}

\subsection*{The Structure of Today's ICD:}
\begin{itemize}
  \item ICD-11 contains 17,000 diagnostic categories and over 100,000 medical diagnostic index terms.
  \item It reflects the density of collisions of classification schemes that have evolved over time.
\end{itemize}

\subsection*{Building Boxes:}
\begin{itemize}
  \item Classification systems involve choices and are not self-evident.
  \item Two ideal types: Aristotelian (empirical, top-down) and prototypical (discursive).
\end{itemize}

\subsection*{ICD as a Compromise:}
\begin{itemize}
  \item A pragmatic compromise that embraces complexity and multiplicity.
  \item Reflects the socio-technical embeddedness of health data.
\end{itemize}

\subsection*{Death Reporting:}
\begin{itemize}
  \item The challenge of selecting a single cause of death.
  \item Recognizes the ambiguity of causes of death, especially in cases of multimorbidity.
\end{itemize}

\subsection*{Standardization of Reporting:}
\begin{itemize}
  \item A globally standardized medical certificate of cause of death takes account of ambiguity.
  \item Provides space to tell a story and explain a causal chain of different conditions.
\end{itemize}

\subsection*{Data and Imagination:}
\begin{itemize}
  \item Data must be imagined to exist and function, to be articulated against the seamlessness of phenomena.
  \item Understanding basic assumptions through the study of classification systems.
\end{itemize}

\subsection*{Summary:}
\begin{itemize}
  \item Classification systems are constructed in accordance with practical goals and constraints.
  \item This doesn't diminish the scientific nature of the data generated.
  \item It highlights the embeddedness of knowledge systems in the real, imperfect world.
\end{itemize}

\section*{Session 4 (17.10.2023)}

Focused on data infrastructures in the context of the course "Making data, making worlds: An introduction to data practices." Here are the main points discussed:

\subsection*{Rear view mirror:}
\begin{itemize}
    \item Building boxes is critical to assembling comparable data at scale, but it is not easy.
    \item Classification systems need to take account of both theoretical and practical considerations.
    \item Global statistics and models involve trade-offs and choices that should be acknowledged.
\end{itemize}

\subsection*{What is an infrastructure?}
\begin{itemize}
    \item Infrastructures are sets of collective facilities necessary for human activities.
    \item They can be physical (bricks, wires) or intangible (protocols, standards).
    \item Infrastructures are often invisible when functioning properly, but become noticeable when they break down.
\end{itemize}

\subsection*{Characteristics of Infrastructure (Bowker and Star 1999):}
\begin{itemize}
    \item Embeddedness: Infrastructure is within and part of other structures.
    \item Transparency: It supports tasks invisibly.
    \item Reach: Extends spatially or temporally beyond a single event.
    \item Learned as part of membership: Outsiders learn about the infrastructure.
    \item Embodies standards: Adapts and connects to other infrastructures in a standardized way.
    \item Built on an installed base: Grows from an existing base, inheriting its strengths and limitations.
    \item Becomes visible on failure: Invisibility becomes visible on failure.
    \item Fixes in modular increments: Changes are not global but incremental.
\end{itemize}

\subsection*{Data Infrastructures:}
\begin{itemize}
    \item Digital means of storing, sharing, connecting and consuming data over the Internet.
    \item Includes servers, cables, catalogues, directories and more.
    \item Critical to many aspects of life, including information sharing, communication, government and business.
\end{itemize}

\subsection*{Application areas:}
\begin{itemize}
    \item Data infrastructures are essential for information exchange, governance, economics and are emerging as business models.
\end{itemize}

\subsection*{Databases for EU internal security}
\begin{itemize}
    \item The Schengen visa system relies on data infrastructures to exchange information.
    \item Supranational databases pool data from individual Member States to manage security and border control.
\end{itemize}

\subsection*{Development of a database for information exchange:}
\begin{itemize}
    \item One example is the Visa Information System (VIS).
    \item A study of the construction of the VIS reveals frictions and coordination processes in a multi-level governance system.
\end{itemize}

\subsection*{Implications:}
\begin{itemize}
    \item Building and maintaining data infrastructures involve social and political negotiations.
    \item Ongoing coordination is required for data generation, formats, standards, security and more.
    \item Database design is not predetermined and has power implications.
\end{itemize}

\subsection*{Broader perspective on infrastructure (Crawford 2021):}
\begin{itemize}
    \item Artificial Intelligence is embodied, material and connected to various global systems.
    \item The environmental and social impacts of data infrastructures are significant.
\end{itemize}

\subsection*{Summary:}
\begin{itemize}
    \item Infrastructures are ubiquitous and form the backbone of communication.
    \item Construction and maintenance require coordination and alignment.
    \item Data infrastructures are embedded in global trade and capital structures.
\end{itemize}


\section*{Session 5 (02.11.2023) "Data Journeys and Data Friction"}
The main points discussed are:

\subsection*{Reflections on Data Infrastructures:}
\begin{itemize}
    \item Data infrastructures are powerful tools for communication, business, and governance.
    \item Building and maintaining these infrastructures requires coordination and alignment.
    \item Data infrastructures are embedded in global structures of trade and capital.
\end{itemize}

\subsection*{Challenges in Classification Systems and Standards:}
\begin{itemize}
    \item Classification systems and standards involve trade-offs, especially on a large scale.
    \item Challenges arise when these systems need to be universally applicable and compatible with different local infrastructures, cultures, and practices.
\end{itemize}

\subsection*{Measuring climate and data frictions:}
\begin{itemize}
    \item Discussion of how climate scientists produce knowledge about climate and patterns of climate change.
    \item Introduction to the concept of "data frictions" in the context of global climate models.
\end{itemize}

\subsection*{Climate modelling and data frictions (Edwards 2010):}
\begin{itemize}
    \item Examine global climate models and the history of efforts to collect weather and climate records.
    \item Highlights the existence of "data frictions" that complicate the production of global data sets.
\end{itemize}

\subsection*{Definitions in meteorology:}
\begin{itemize}
    \item Weather vs. climate and the longitudinal function of climate.
    \item Climate change as long-term shifts in temperature and weather patterns.
\end{itemize}

\subsection*{Modelling weather/climate:}
\begin{itemize}
    \item Weather and climate modelling involves complex processes and a large number of variables.
    \item The need for global data has led to the development of computational infrastructures for weather prediction.
\end{itemize}

\subsection*{Generating global data:}
\begin{itemize}
    \item Efforts since the 1940s to make global data possible.
    \item Challenges and the role of standards in facilitating global data flows.
\end{itemize}

\subsection*{Concept of friction:}
\begin{itemize}
    \item Friction in physical systems as resistance at interfaces.
    \item Introduction of computational friction in information systems and the challenges it poses.
\end{itemize}

\subsection*{Data Friction:}
\begin{itemize}
    \item Define "data friction" as the cost in time, energy, and attention required for various data-related activities.
\end{itemize}

\subsection*{Data Journeys (Leonelli 2020):}
\begin{itemize}
    \item Introduces the data journey approach, focusing on the life of data as it moves through space and time.
    \item Emphasis on data friction at different stages of data movement.
\end{itemize}

\subsection*{Examples of making data global:}
\begin{itemize}
    \item Strategies used to make data global, including automation, manual inspection, and interpolation.
    \item The role of computer models in creating consistent global datasets.
\end{itemize}

\subsection*{Implications of global data for meteorology:}
\begin{itemize}
    \item The changing meaning of "data" in meteorology today.
    \item The merging of data and theory to create a "data image" rather than a traditional data set.
\end{itemize}

\subsection*{Data frictions in security contexts:}
\begin{itemize}
    \item Study of Passenger Information Units in the EU and the data frictions in reusing commercial data for security.
    \item Several efforts are needed to make data suitable for security intelligence.
\end{itemize}

\subsection*{Reducing friction in security data:}
\begin{itemize}
    \item Challenges in reducing friction, conflicts between quality requirements, and establishing quality control processes.
    \item Practical adaptations to make data-driven intelligence actionable.
\end{itemize}

\subsection*{Summary:}
\begin{itemize}
    \item Data, as it travels, creates friction that must be resolved.
    \item Trust in data is rare due to deviations and inconsistencies, requiring active labor to fix and extract meaningful knowledge.
    \item Fixing data changes its shape and informational value.
\end{itemize}

\section*{Session 6 (9.11.2023)}
Focused on the role of metadata, particularly in the context of surveillance practices. The main points discussed were:
\subsection*{Metadata Overview:}
\begin{itemize}
    \item Metadata, in the context of this session, refers to structured information that describes, explains, locates, or manages an information resource.
    \item Metadata is often generated in an automated way and plays a crucial role in ensuring access to information resources both in time and in space.
\end{itemize}

\subsection*{Types of metadata: Descriptive metadata.}
\begin{itemize}
    \item Descriptive metadata is used to discover and locate data, including keywords for search queries.
    \item Administrative metadata includes technical, preservation, and legal information necessary for the operational capability of information systems.
    \item Structural metadata establishes links between smaller pieces of data and describes the structure, types, and relationships of data.
\end{itemize}

\subsection*{Examples of descriptive metadata:}
\begin{itemize}
    \item The session provided examples of descriptive metadata, such as library catalogue records and online trading platforms, to illustrate how metadata facilitates the identification and retrieval of information.
\end{itemize}

\subsection*{Challenges in systematising description:}
\begin{itemize}
    \item Uniformity of both categories and category content is critical to metadata systems and requires adherence to standards.
    \item Standardisation involves addressing issues of semantics, syntax, and content rules.
\end{itemize}

\subsection*{Example standard: Dublin Core:}
\begin{itemize}
    \item Dublin Core is presented as an example of a descriptive metadata standard, consisting of fifteen key metadata elements for describing digital or physical resources.
\end{itemize}

\subsection*{Administrative and structural metadata:}
\begin{itemize}
    \item Administrative metadata, including technical details and rights information, is essential for data preservation and access.
    \item Structural metadata establishes links between different parts of the resource, allowing more complex objects to be assembled.
\end{itemize}

\subsection*{Metadata Surveillance:}
\begin{itemize}
    \item This session discussed the implications of metadata in the context of surveillance, highlighting how metadata surveillance capabilities have expanded the scale and scope of surveillance activities.
\end{itemize}

\subsection*{The relevance of metadata:}
\begin{itemize}
    \item Despite their technical nature, metadata can reveal profound insights about individuals, their habits, and the connections between entire networks.
\end{itemize}

\subsection*{Evolution of the meaning of metadata:}
\begin{itemize}
    \item The meaning of metadata has evolved, especially in the wake of the Snowden revelations, from a naive technical understanding to a recognition of its role in socio-technical data assemblages.
\end{itemize}

\subsection*{Conclusion:}
\begin{itemize}
    \item Metadata serves primarily as an organisational tool for data management but has also become a focal point for global surveillance programmes, raising concerns about individual privacy rights.
\end{itemize}

The session highlighted the importance of understanding the role of metadata in shaping knowledge and its implications in surveillance contexts.
\section*{Session 7 (16.11.2023)}
Data capitalism and data justice in the context of the course "Making data, making worlds: An introduction to data practices". Key points discussed included:

\subsection*{Metadata Overview:}
\begin{itemize}
    \item Metadata primarily serves as an organisational tool for data management.
    \item Although considered technical and neutral, metadata has become a focal point for global surveillance and counter-terrorism programmes.
\end{itemize}

\subsection*{What is capitalism?}
\begin{itemize}
    \item Capitalism is an economic system based on private property, the profit motive, competitive markets and wage labour.
    \item Proponents argue that it promotes efficient production, economic growth, innovation and social/financial freedom. Critics point to its exploitative, environmentally unsustainable and inequitable nature.
\end{itemize}

\subsection*{Data and capitalism:}
\begin{itemize}
    \item Data is described as the "new oil" in the data business model.
    \item The equation is (more) data + (better) algorithms = value added.
    \item Surplus value has implications for research, healthcare, governance, international security and the economy.
\end{itemize}

\subsection*{Data and platforms:}
\begin{itemize}
    \item Platforms are essential to the data business model.
    \item Characteristics of successful platforms include infrastructure, data monetisation, network effects, monopolisation, cross-subsidisation, rule-setting and extraction.
    \item Examples of successful platforms include Apple's App Store, Google's search engine, Uber and eBay.
\end{itemize}

\subsection*{Surveillance capitalism:}
\begin{itemize}
    \item Surveillance capitalism exploits and controls human behaviour through data extraction.
    \item Key mechanisms include commodification and prediction, leading to behavioural futures markets.
\end{itemize}

\subsection*{Example: Pokémon Go:}
\begin{itemize}
    \item Pokémon Go illustrates how data can be extensively collected (location, device information) and commodified for targeted advertising.
\end{itemize}

\subsection*{Instrumental power:}
\begin{itemize}
    \item Surveillance capitalism introduces instrumental power, which shapes human behaviour on a large scale through automated computer architectures.
\end{itemize}

\subsection*{Eight Theses on Surveillance Capitalism:}
\begin{itemize}
    \item Surveillance capitalism is a new economic order with unprecedented concentrations of wealth, knowledge and power.
    \item It poses significant threats to humanity and democracy.
\end{itemize}

\subsection*{Data Justice:}
\begin{itemize}
    \item Data justice examines the social impact of data.
    \item Data harms include exploitation, discrimination, loss of privacy, surveillance, manipulation, exclusion and injustice.
\end{itemize}

\subsection*{Data activism:}
\begin{itemize}
    \item Data activism addresses data injustices through digital means, including data collection, visualisation and resistance to surveillance.
\end{itemize}

\subsection*{Summary:}
\begin{itemize}
    \item Data and capitalism are closely linked, resulting in new forms of capitalism based on platforms and data extraction.
    \item Data capitalism introduces new forms of harm with implications for justice.
\end{itemize}

The session highlights the need to integrate data into discussions of justice and introduces the concept of data activism to address data injustices.


\section*{Session 8 on Data Policy}
\textit{Explored the evolving landscape of data regulation and its implications. Key points included:}

\subsection*{Flashback:}
\begin{itemize}
    \item Capitalism and justice issues are intertwined.
    \item Data has led to new business models centered on platforms and data extraction.
    \item Data capitalism introduces new forms of harm with implications for justice.
\end{itemize}

\subsection*{Generalizability of Cases:}
\begin{itemize}
    \item Analysis of US tech companies may have global relevance due to de-territorialization of business models.
    \item Challenges arise due to cultural differences, which require differentiated approaches.
\end{itemize}

\subsection*{Uber's Business Model:}
\begin{itemize}
    \item Uber's success is based on its platform that connects transport providers and customers.
    \item Monetization is through commissions and the extraction of interaction data for analysis.
    \item Uber's disruptive business strategy faces regulatory challenges worldwide.
\end{itemize}

\subsection*{Regulatory Dilemma:}
\begin{itemize}
    \item The Collingridge Dilemma poses challenges in regulating rapid technological innovation.
    \item Uber's strategy capitalizes on the dilemma, betting that regulation won't catch up.
\end{itemize}

\subsection*{The Regulatory Response:}
\begin{itemize}
    \item Uber's success has been uneven around the world, with more acceptance in common law countries and bans in civil law countries.
    \item The lack of regulation for new data-driven business models is driving the data policy debate.
\end{itemize}

\subsection*{The Right to be Forgotten:}
\begin{itemize}
    \item The Costeja González v Google case led to the establishment of the "right to be forgotten" in the EU.
    \item The EU's General Data Protection Regulation (GDPR) enshrines this right and regulates the processing of personal data.
\end{itemize}

\subsection*{GDPR Principles:}
\begin{itemize}
    \item The GDPR limits the use of data, requires purpose limitation, and addresses automated decision making.
    \item It empowers citizens with rights such as access, rectification, data portability, and challenging algorithmic decisions.
\end{itemize}

\subsection*{EU as Protector:}
\begin{itemize}
    \item Through the GDPR, the EU is protecting citizens from powerful corporations in the digital age.
    \item The GDPR serves as a model for data policy worldwide, emphasizing the right to privacy.
\end{itemize}

\subsection*{Enforcement and Critique:}
\begin{itemize}
    \item GDPR fines indicate active enforcement, but critics cite bureaucratic complexity and stifled innovation.
\end{itemize}

\subsection*{Clashing Visions:}
\begin{itemize}
    \item The EU leads the way in data regulation, while the US lacks a GDPR equivalent, leading to clashes in global business practices.
\end{itemize}

\subsection*{Global Trends:}
\begin{itemize}
    \item One global trend emphasizes citizen empowerment in the data ecosystem.
    \item Data protection is a key regulatory pillar, but a consistent global framework remains elusive.
\end{itemize}

\subsection*{Summary:}
\begin{itemize}
    \item Data policy is a response to societal datafication, but regulatory development lags behind technological progress.
    \item The choice of regulation depends on political prioritizations regarding economic order and the rights of citizens/consumers.
    \item Regulation is coined by jurisdictional questions and different visions of data policy.
\end{itemize}

\textit{This session highlighted the ongoing evolution of data policy and emphasized the need for a comprehensive global regulatory framework.}
\section*{Session 9 (30.11.2023) - Data Governance}

\subsection*{Looking Back:}
\begin{itemize}
    \item Data policy emerges in response to societal datafication and evolving business and government capabilities.
    \item Technological innovation outpaces the development of an appropriate regulatory environment.
    \item Regulatory decisions depend on political priorities and vary by jurisdiction and vision of data policy.
\end{itemize}

\subsection*{Aggregation of data}
\subsection*{Data sharing}
\subsection*{EU JHA databases}

\subsection*{What could go wrong?}
\begin{itemize}
    \item Challenges in EU databases include data quality issues, inconsistencies and misuse.
    \item Discussion on how to ensure good data quality in centralised EU databases.
\end{itemize}

\subsection*{The policy strategy I}
\begin{itemize}
    \item European Commission 2021: Implementing Regulation on automated data quality control mechanisms.
    \item Sorting data into categories: "good quality", "low quality", "rejected".
\end{itemize}

\subsection*{Policy Strategy II}
\begin{itemize}
    \item Council of the EU 2020: Roadmap for standardisation for data quality.
    \item Data quality standards for biometric and alphanumeric data, reference catalogue and cybersecurity.
\end{itemize}

\subsection*{Data quality as an "asset"}
\begin{itemize}
    \item Reliable data is essential for meaningful use; poor data quality increases costs and leads to sub-optimal outcomes.
\end{itemize}

\subsection*{Dimensions of data quality}
\subsubsection*{I: Accuracy}
\subsection*{II: Completeness}
\subsubsection*{III: Trustworthiness of sources}
\subsubsection*{IV: Timeliness}
\subsubsection*{V: Accessibility}


\subsection*{Interdependencies and trade-offs}
\begin{itemize}
    \item Different dimensions of data quality are interdependent; trade-offs force organisations to prioritise.
\end{itemize}

\subsection*{Doing data quality}
\begin{itemize}
    \item Ensuring quality data involves complex processes and data governance.
\end{itemize}

\subsection*{The CURATE Project}
\begin{itemize}
    \item European Research Council funded project (6/2022 - 5/2027) addressing data quality challenges in law enforcement databases.
    \item Research questions focus on the origins of "bad" data, its effects and strategies for improvement.
\end{itemize}

\subsection*{From data policy to data governance}
\begin{itemize}
    \item Data policy governs the production and use of data; data governance governs the use of data within an organisation.
\end{itemize}

\subsection*{Implications}
\subsection*{Practice}
\begin{itemize}
    \item Professionalisation of data quality is essential; this includes defining requirements, roles, responsibilities and allocating resources.
\end{itemize}
\subsubsection*{Policy}
\begin{itemize}
    \item Calls for further reform and harmonisation, in particular for reliable data to be entered into EU databases.
\end{itemize}
\subsubsection*{Research}
\begin{itemize}
    \item Stresses the need for more academic engagement with data quality, especially in the public sector.
\end{itemize}

\subsection*{Step 2: Zoom in}
\begin{itemize}
    \item Qualitative research on data journeys in European countries, involving practitioners to gain a deeper understanding of data practices.
\end{itemize}

\subsection*{summary}
\begin{itemize}
    \item Trustworthiness and reliability of data are key challenges for data-driven decision making.
    \item Data quality is multidimensional, involves trade-offs and requires appropriate governance.
    \item An under-researched aspect, especially in the public sector.
\end{itemize}

\section*{Session 10 (07.12.2023) - Data Preservation}

\subsection*{The Challenge of Data-Driven Decision Making}
\begin{itemize}
  \item Often defined in practice as data quality.
  \item Multidimensional concept with trade-offs.
  \item Organizational governance critical to trustworthiness.
\end{itemize}

\subsection*{Data Lifecycle}
\begin{itemize}
  \item Data created for a specific purpose.
  \item Moved, merged, and used in different ways throughout its life.
  \item Options at the end of lifecycle: preservation or death.
\end{itemize}

\subsection*{Preservation}
\begin{itemize}
  \item Principles, policies, and rules for extending the life of data.
  \item Addresses changing technologies and migration needs.
  \item Active curation to prevent damage and loss.
\end{itemize}

\subsection*{The Archive}
\begin{itemize}
  \item National archives, libraries, government agencies, museums, etc.
  \item Preserves the full set of records with supporting documentation.
  \item Actively structured, curated, documented, and planned.
  \item Long-term effort for future re-use of data.
\end{itemize}

\subsection*{Life and Death Decisions}
\begin{itemize}
  \item Not everything is worth preserving, especially in the digital age.
  \item Archivists and librarians sort and appraise files.
  \item Some policies, such as the "right to erase," encourage data destruction.
\end{itemize}

\subsection*{Data Death}
\begin{itemize}
  \item Irreversible deletion without backup.
  \item Intentional or accidental, implying unavailability.
  \item Some deleted data may be "resurrected" under certain circumstances.
\end{itemize}

\subsection*{The Politics of Digital Preservation and Loss}
\begin{itemize}
  \item Thylstrup (2018) examines who preserves digital artifacts.
  \item Highlights decisions about what to keep and what to discard.
  \item Raises concerns about power dynamics, legal concepts, and societal implications.
\end{itemize}

\subsection*{The Internet Never Forgets - Or Does It?}
\begin{itemize}
  \item Data frequently disappears, e.g. the average web page is deleted within 100 days.
  \item Initiatives such as Internet Archive, Project Gutenberg, Europeana, Google Books aim to digitize cultural artifacts.
  \item Brewster Kahle's perspective on large-scale data deletion.
\end{itemize}

\subsection*{Digital Cultural Heritage Projects}
\begin{itemize}
  \item Internet Archive, Project Gutenberg, Europeana, Google Books.
  \item Aim to digitize, preserve, and make accessible cultural artifacts.
  \item Controversies surrounding these initiatives.
\end{itemize}

\subsection*{Internet Archive vs. Publishers}
\begin{itemize}
  \item Lawsuit over copyright infringement.
  \item Publishers sue Internet Archive for lending digital copies.
  \item Debate over "controlled digital lending."
\end{itemize}

\subsection*{Ruling and Implications}
\begin{itemize}
  \item New York Southern District Court sides with publishers.
  \item Internet Archive to appeal.
  \item Implications for public access to information vs. intellectual property rights.
\end{itemize}

\subsection*{Digitalization and Democracy}
\begin{itemize}
  \item Public discourse influenced by social media.
  \item Challenges associated with content moderation algorithms.
  \item Fundamental questions for democracy.
\end{itemize}

\subsection*{Digitization and Culture}
\begin{itemize}
  \item Cultural forms heavily dependent on digital formats.
  \item Shift from physical to digital availability.
  \item Power dynamics in the hands of for-profit companies.
\end{itemize}

\subsection*{Reclaiming Control}
\begin{itemize}
  \item Thylstrup (2018) emphasizes the need to rethink power structures.
  \item Calls for stronger institutional safeguards, rules, and stable support.
\end{itemize}

\subsection*{Summary}
\begin{itemize}
  \item Data preservation practices are critical to cultural heritage.
  \item Driven by choices, not a simple story of progress.
  \item Growing struggle to define ownership of humanity's digital memory.
\end{itemize}

\subsection*{The Performativity of Data}
\begin{itemize}
  \item A concept that highlights how data bring phenomena into being.
  \item Power struggles implicit in the creation and analysis of data.
\end{itemize}

\subsection*{Classification Systems}
\begin{itemize}
  \item Conventions that define the translation of empirical phenomena into data.
  \item Rarely fixed, changing over time based on social and political conventions.
\end{itemize}

\subsection*{Infrastructures}
\begin{itemize}
  \item Key to the creation, sharing, and analysis of data.
  \item Construction and maintenance require coordination and alignment.
  \item Raises issues of data inclusivity.
\end{itemize}

\subsection*{Data Journeys}
\begin{itemize}
  \item Friction in data journeys that require resolution.
  \item Active labor required to 'fix' data for meaningful knowledge extraction.
  \item Changes in data form and informational value during fixing.
\end{itemize}

\subsection*{Metadata}
\begin{itemize}
  \item Primarily organizational, but also used for surveillance.
  \item Collection of information about activities and networks.
  \item Extends to uninvolved individuals.
\end{itemize}

\subsection*{Data Capitalism}
\begin{itemize}
  \item Data productive of new business models.
  \item Monetization of human nature.
  \item Implications for behavioral prediction and nudging.
\end{itemize}

\subsection*{Data Justice}
\begin{itemize}
  \item Harms of data-driven decision making.
  \item Social sorting, discrimination, lack of transparency.
  \item Implications for equity and life chances.
\end{itemize}

\subsection*{Data Politics}
\begin{itemize}
  \item Emerging responses to datafication.
  \item Lagging behind technological innovation.
  \item Regulatory choices influenced by political prioritization.
\end{itemize}

\subsection*{Data Governance}
\begin{itemize}
  \item Challenges in ensuring trustworthiness and reliability.
  \item Multidimensional concept with trade-offs.
  \item Under-researched, especially in the public sector.
\end{itemize}

\subsection*{Data Deaths (Repeated)}
\begin{itemize}
  \item Data preservation practices key to cultural heritage.
  \item Driven by choices, not a simple story of progress.
  \item Growing struggle to define ownership of humanity's digital memory.
\end{itemize}


\end{document}
