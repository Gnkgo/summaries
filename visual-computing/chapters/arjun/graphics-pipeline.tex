\section{Graphics Pipeline}

\begin{compactenum}
        \item Modelling Transform (Object to World Space)
        \item Viewing Transform (World to Camera Space)
        \item Primitive Processing (Output primitives from transformed vertices)
        \item 3D-Clipping (Remove primitives outside the frustum)
        \item Screen-Space Projection (Project from 3D to 2D screen space)
        \item Scan Conversion (Discretize continuous primitives)
        \item Lighting, Shading, Texturing
        \item Occlusion Handling (Update Color using Z-buffer)
        \item Display
\end{compactenum}
\greenbf{Programmer's View:} \\
%\includegraphics*[width = \columnwidth]{arjun/pipeline-programmer.png} \\

\includegraphics*[width = \columnwidth]{jo/rasterization.png} \\
\begin{tikzpicture}[
    node distance = 1cm,
    mvertex/.style = {draw, rectangle, align=center, draw=H2, color=white, fill=H3, thick},
    vertex/.style = {draw, rectangle, align=center, draw=H1, thick},
    every edge/.style={draw, ->, auto, semithick}
]
  % Define vertices
  \node[vertex] at (0, 0) (attr) {Attributes};
  \node[mvertex] at (1, -0.6) (vs) {Vertex Shader};
  \node[vertex] at (1.5, 0.2) (vvar) {Varying \\ (per vertex)};
  \node[mvertex] at (2.8, -0.6) (inter) {Interpolation};
  \node[vertex] at (3.3, 0.2) (fvar) {Varying \\ (per fragment)};
  \node[mvertex] at (4.8, -0.6) (fs) {Fragment Shader};
  \node[vertex] at (5.3, 0) (fc) {Fragment Color};
  \node[vertex] at (2.8, -1.2) (unif) {Uniforms (constants)};

  \draw (unif) edge (vs);
  \draw (unif) edge (fs);
  \draw (attr) edge (vs);
  \draw (vs) edge (vvar);
  \draw (vvar) edge (inter);
  \draw (inter) edge (fvar);
  \draw (fvar) edge (fs);
  \draw (fs) edge (fc);
\end{tikzpicture}
\greenbf{Vertex Processing:} Per-vertex operations e.g Transforms and Lighting flow control. This is done with the Vertex Shader.  Input: uniforms and per-vertex attributes. Output: Varying per vertex\\
\greenbf{Fragment Processing:} Per-fragment operations e.g. Shading and Texturing Blending. This is done with the Fragment Shader. Input: Uniform and varying per-fragment attributes. Output: Per-fragment color\\
\greenbf{Inputs/Outputs:}
\begin{compactitem}
    \item Uniforms: (V/F) global constant inputs e.g. light position, texture map etc.
    \item Varying: (V/F) value passed from vertex to fragment shader by being interpolated across primitives first. e.g interp. pixel color
\end{compactitem}


