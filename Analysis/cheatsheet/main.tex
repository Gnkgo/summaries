% IMPORTANT NOTE: This is a slightly adapted Copy of https://tex.stackexchange.com/questions/181081/clickable-chapters-on-the-right-side-of-each-page
% The credit goes to Gonzalo Medina, I only made slight changes (Converted to landscape and made Tabs nameable)

% ANOTHER IMPORTANT NOTE: The Boxlayout with Titles was copied from here: https://www.overleaf.com/articles/130-cheat-sheet/ntwtkmpxmgrp
% The Credit goes to Drew Ulick
%\documentstyle[8pt]{extarticle}
%\documentclass[9pt]{article}
\documentclass[8pt]{extarticle}

% Article
% \documentclass[9pt]{extarticle}
\usepackage[landscape, left=0.75cm, top=1cm, right=0.75cm, bottom=1.5cm, footskip=15pt]{geometry}
\usepackage{background}
\usepackage{etoolbox}
\usepackage{graphicx}
\usepackage{totcount}
\usepackage{lipsum}
\usepackage{hyperref}
\usepackage{amsmath}
\usepackage{amssymb}

% Mathematical typesetting & symbols
\usepackage{amsthm, mathtools, amssymb} 
\usepackage{marvosym, wasysym}
\allowdisplaybreaks

% Math helper stuff
\def\limn{\lim_{n\to \infty}}
\def\limxo{\lim_{x\to 0}}
\def\limxi{\lim_{x\to\infty}}
\def\limxn{\lim_{x\to-\infty}}
\def\sumk{\sum_{k=1}^\infty}
\def\sumn{\sum_{n=0}^\infty}
\def\R{\mathbf{R}}
\def\dx{\text{ d}x}

% color text
\usepackage{xcolor}

% Tables
\usepackage{tabularx, multirow}
\usepackage{booktabs}
\renewcommand*{\arraystretch}{2}

% image directory
\graphicspath{ {./assets/} }

% For accessing arrays
\usepackage{etoolbox}

% for emumerating
\usepackage{enumerate}

% for color coding
\usetikzlibrary{backgrounds}

% for light font +C
\usepackage{color}
\definecolor{light}{rgb}{0.5, 0.5, 0.5}
\def\light#1{{\color{light}#1}}



% for multicolumn
\usepackage{multicol}
\usepackage{multirow}

% to have access to the total number of sections
\regtotcounter{section}

% the main part; as background material we place the border, 
% the section (current and other) tabs and the page number
\backgroundsetup{
scale=1,
color=black,
angle=0,
opacity=1,
contents={}
}

%\usepackage{sectsty}
%\subsectionfont{\normalfont\small\bfseries\underline}
%\subsubsectionfont{\normalfont\small}
\begin{document}


\setlength{\columnseprule}{0.4pt}
\pagenumbering{arabic}
\begin{multicols*}{3}

\section{Folgen}
\hypertarget{sec:0}{}

% Konvergenz

  \subsection {Konvergenz}

Die Folge $(a_n)_{n \geq 1}$ \textbf{konvergiert} gegen $a$ für
$n \rightarrow \infty$, falls gilt:
\begin{align*}
  \forall \epsilon > 0 \; \exists n_0 = n_0(\epsilon) \in \mathbf{N} \; \forall n \geq n_0 : |a_n - a| < \epsilon
\end{align*}
Wir schreiben dann:
\begin{align*}
  a = \lim_{n \rightarrow \infty} a_n \text{ oder }a_n \rightarrow a \; (n \rightarrow \infty)
\end{align*}
und nennen $a$ den \textbf{Grendwert/ Limes} der Folge $(a_n)_{n \geq 1}$.
Existiert der Limes nicht, so heisst die Folge \textbf{divergent}. Zu bemerken ist:
\begin{align*}
  (a_n)_{n \geq 1} \text{ konvergent } \Rightarrow (a_n)_{n \geq 1} \text{ beschränkt }
\end{align*}
% Monotone Konvergenz
  \subsection{Monotone Konvergenz}
Sei die Folge $(a_n)_{n \geq 1}$ monoton wachsend und nach oben beschränkt.
Dann konvergiert $(a_n)_{n \geq 1}$ mit Grenzwert:
\begin{align*}
  \lim_{n \rightarrow \infty} a_n = \text{sup} \{a_n : n \geq 1\}
\end{align*}
Ist die Folge $(a_n)_{n \geq 1}$ monoton fallend und nach unten beschränkt,
so konvergiert $(a_n)_{n \geq 1}$ mit Grenzwert:
\begin{align*}
  \lim_{n \rightarrow \infty} a_n = \text{inf} \{a_n : n \geq 1\}
\end{align*}
% Cauchy Kriterium
\subsection{Cauchy Kriterium}

Die Folge $(a_n)_{n \geq 1}$ ist genau dann konvergent, falls:
$$
  \forall \epsilon > 0 \; \exists N \geq 1 \text{ so dass } |a_n - a_m| < \epsilon \; \; \forall n, m \geq N\\
$$
% Rechnen mit Limes

  \subsection {Rechnen mit Limes}

Seien die Folgen $(a_n)_{n \geq 1}$, $(b_n)_{n \geq 1}$ konvergent mit
$\lim_{n \rightarrow \infty} a_n = a$ und $\lim_{n \rightarrow \infty} b_n = b$.
Dann gilt:
\begin{enumerate}[(i)]
  \item $\lim_{n \rightarrow \infty} (a_n + b_n) = \lim_{n \rightarrow \infty} a_n + \lim_{n \rightarrow \infty} b_n$
  \item $\lim_{n \rightarrow \infty} (a_n * b_n) = \lim_{n \rightarrow \infty} a_n * \lim_{n \rightarrow \infty} b_n$
  \item Falls zusätzlich $b_n \neq 0 \; \forall n \geq 1$ und $b \neq 0$ gegeben ist, so gilt: $\lim_{n \rightarrow \infty} (a_n / b_n) = a / b$.
  \item Falls es ein $K \geq 1$ gibt mit $a_n \leq b_n \; \forall n \geq K$, dann folgt $a \leq b$.
\end{enumerate}
% Limes Superior/ Inferior
  \subsection{Limes Superior/ Inferior}
Sei eine Folge $(a_n)_{n \geq 1}$ beschränkt. Wir können dann zwei monotone
Folgen $(b_n)_{n \geq 1}$ und $(c_n)_{n \geq 1}$ definieren, welche dann einen Grenzwert besitzen.
Sei für jedes $n \geq 1$:
\begin{gather*}
  b_n = \inf\{a_k : k \geq n\} \text{ und } c_n = \sup\{a_k : k \geq n\}\\
  b_n \leq b_{n + 1}\\
  c_{n + 1} \leq c_n
\end{gather*}
Da also beide Folgen beschränkt sind und konvergieren, können wir aufgrund von
Monotoner Konvergenz folgern:
\begin{gather*}
  \liminf_{n \rightarrow \infty} a_n := \lim_{n \rightarrow \infty} b_n\\
  \limsup_{n \rightarrow \infty} a_n := \lim_{n \rightarrow \infty} c_n\\
  \liminf_{n \rightarrow \infty} a_n \leq \limsup_{n \rightarrow \infty} a_n
\end{gather*}
Es gilt auch, dass $(a_n)_{n \geq 1}$ genau dann konvergiert, falls $(a_n)_{n \geq 1}$
beschränkt ist und $\liminf_{n \rightarrow \infty} a_n = \limsup_{n \rightarrow \infty} a_n$


% Bolzano-Weierstrass
 \subsection{Bolzano-Weierstrass}
Jede beschränkte Folge besitzt eine konvergente Teilfolge.\\
Wenn $a_n$ monoton wachsend und nach oben beschränkt ist, dann konvergiert $a_n$ mit Grenzwert $\limn a_n = \sup \{a_n : \ n \ge 1\}$.
Wenn $a_n$ monoton fallend und nach unten beschränkt ist, dann konvergiert $a_n$ mit Grenzwert $\limn a_n = \inf \{a_n : \ n \ge 1\}$.


% Sandwichsatz für Folgen
  \subsection {Sandwichsatz für Folgen}
Seien $(a_n)_{n \geq 1}$ und $(b_n)_{n \geq 1}$ konvergente Folgen mit demselben Limes
$\alpha \in \mathbf{R}$. Ist $K \in \mathbf{N}$ und $(c_n)_{n \geq 1}$ eine Folge
mit der Eigenschaft: $$a_n \leq c_n \leq b_n \; \; \; \forall n \geq K$$ so konvergiert
auch $(c_n)_{n \geq 1}$ gegen $\alpha$.
% Limes Binom Trick
  \subsection{Limes Binom Trick}
Gegeben die Summe zweier Wurzeln könnte man wie folgt
vorgehen (Bsp.):
$$
  \lim_{x \rightarrow \infty} (\sqrt{x + 5} - \sqrt{x - 3}) = \lim_{x \rightarrow \infty} (\frac{(x + 5) - (x - 3)}{\sqrt{x + 5} + \sqrt{x - 3}})
$$
% Limes Substitution Trick
  \subsection{Limes Substitution Trick}
Hier ein Beispiel:
$$
  \lim_{x \rightarrow \infty} x^2 (1 - cos(\frac{1}{x}))
$$\\
Substitutiere nun $u = \frac{1}{x}$:
$$
\lim_{u \rightarrow 0} \frac{1 - cos(u)}{u^2} = \lim_{u \rightarrow 0} \frac{sin(u)}{2u} = \lim_{u \rightarrow 0} \frac{cos(u)}{2} = \frac{1}{2}
$$
% Limes Taylor Trick

  \subsection{Limes Taylor Trick}

Mithilfe der Reihenentwicklung von $e^x$ und $sin(x)$:
$$
\lim_{n \rightarrow \infty} \frac{e^{1/n} - 1 - \frac{1}{n}}{1 - n * sin(\frac{1}{n})} = \frac{\frac{1}{2}n^{-2} + \mathcal{O}(n^{-3})}{1 - n(n^{-1} - \frac{1}{6}n^{-3} + \mathcal{O}(n^{-5}))} = 3
$$
\subsection{Strategie - Konvergenz von Folgen}
\begin{enumerate}
 \item Bei Brüchen: Grösste Potenz von $n$ kürzen. Alle Brüche der Form $\frac{a}{n^a}$ streichen, da diese nach 0 gehen.
 \item Bei Wurzeln in Summe im Nenner: Multiplizieren des Nenners und Zählers mit der Differenz der Summe im Nenner. (z.B. $(a+b)$ mit $(a-b)$ multiplizieren)
 \item Bei rekursiven Folgen: Anwendung von Weierstrass zur monotonen Konvergenz
 \item Einschliessungskriterium (Sandwich-Theorem) anwenden.
 \item Mit bekannter Folge vergleichen.
 \item Grenzwert durch einfaches Umformen ermitteln.
 \item Limit per Definition der Konvergenz zeigen.
 \item Anwendung des Cauchy-Kriteriums.
 \item Suchen eines konvergenten Majorant.
 \item Weinen und die Aufgabe überspringen.
\end{enumerate}

\subsection{Strategie - Divergenz von Folgen}
\begin{enumerate}
 \item Suchen einer divergenten Vergleichsfolge.
 \item Alternierende Folgen: Zeige, dass Teilfolgen nicht gleich werden, also $\limn a_{p_1(n)} \ne \limn a_{p_2(n)}$ (mit z.B. gerade/ungerade als Teilfolgen).
\end{enumerate}
\subsection{Induktive Folgen (Induktionstrick)}
\begin{enumerate}
  \item Zeige monoton wachsend / fallend
  \item Zeige beschränkt
  \item Nutze Satz von Weierstrass, d.h. Folge muss gegen Grenzwert konvergieren
  \item Verwende Induktionstrick:
\end{enumerate}
Wenn die Folge konvergiert, hat jede Teilfolge den gleichen Grenzwert. Betrachte die Teilfolge $l(n) = n + 1$ für $d_{n+1} = \sqrt{3d_n - 2}$:
$$d = \lim_{n\to\infty} d_n = \lim_{n\to\infty} d_{n+1} = \sqrt{\lim_{n \to \infty} 3d_n -2} = \sqrt{3d -2}$$
Forme um zu $ d^2 = 3d -2 \to d \in {1,2}$. Nun können wir $d = 2$ nehmen und die Beschränktheit mit $d=2$ per Induktion zeigen.

% Punktweise Konvergenz
 \subsection{Punktweise Konvergenz}
Die Funktionenfolge $(f_n)_{n \geq 0}$ konvergiert punktweise gegen eine Funktion
$f:\mathbf{D} \rightarrow \mathbf{R}$ falls für alle $x \in \mathbf{D}$, $f(x) = \lim_{n \rightarrow \infty} f_n(x)$ gilt. Konkret:
\begin{align*}
  &\forall x \in \mathbf{D} \;\;\; \forall \epsilon > 0 \;\;\; \exists N_{x, \epsilon} \geq 1 \text{ so dass }\\
  &\forall n \geq N \;\;\; |f_n(x) - f(x)| < \epsilon
\end{align*}
% Gleichmässige Konvergenz
\subsection{Gleichmässige Konvergenz}
 Die Funktionenfolge $(f_n)_{n \geq 0}$ konvergiert gleichmässig in $\mathbf{D}$ gegen eine Funktion
$f:\mathbf{D} \rightarrow \mathbf{R}$ falls für alle $x \in \mathbf{D}$, $f(x) = \lim_{n \rightarrow \infty} f_n(x)$ gilt. Konkret:
\begin{align*}
  &\forall \epsilon > 0 \;\;\; \exists N_{\epsilon} \geq 1 \text{ so dass }\\
  &\forall n \geq N \;\;\;\forall x \in \mathbf{D} \;\;\; |f_n(x) - f(x)| < \epsilon
\end{align*}
Weiter ist folgenes Kriterium äquivalent:
\begin{align*}
  &\forall \epsilon > 0 \;\;\; \exists N \geq 1 \text{ so dass }\\
  &\forall n, m \geq N \;\;\;\forall x \in \mathbf{D} \;\;\; |f_n(x) - f_m(x)| < \epsilon
\end{align*}
Konvergiert eine Funktionenfolge $(f_n)_{n \geq 1}, f_n:\mathbf{D} \subset \mathbf{R} \rightarrow \mathbf{R}$
bestehend aus in $\mathbf{D}$ stetigen Funktionen gleichmässig gegen die Funktion $f:\mathbf{D} \rightarrow \mathbf{R}$, so
ist $f$ in $\mathbf{D}$ \textbf{stetig}.\\
\\
\textbf{Beispiel:}
$f_n : \mathbf{R} \rightarrow \mathbf{R}, f_n(x) = \sqrt{|x| + \frac{1}{n^3}}$
Wie lautet der punktweise Limes der Funktionsfolge $f_n$? Konvergiert $f_n$ gleichmässig auf  $\mathbf{R}$?\\ 
\\
\textbf{Punktweise Konvergenz:} Wir fixieren $x \in R$ und bilden den Limes für $n \rightarrow \infty$
$$
\lim_{n \to \infty} f_n(x) = \lim_{n \rightarrow \infty} \sqrt{|x| + \frac{1}{n^3}} = \sqrt{|x|} = f(x)
$$
Die Funktionenfolge konvergiert somit punktweise gegen den punktweisen Grenzwert $f(x) = \sqrt{|x|}$\\
\\
\textbf{Gleichmässige Konvergenz:} Wir müssen zeigen, dass $\lim_{n \to \infty} sup_{x \in \mathbf{R}} |f_n(x) - f(x)| = 0$ gilt. Wir berechnen also zuerst den Ausdruck $sup_{x \in \mathbf{R}} |f_n(x) - f(x)|$
\begin{align*}
	&sup_{x \in \mathbf{R}} |f_n(x) - f(x)| = sup_{x \in \mathbf{R}} \left|\sqrt{|x| + \frac{1}{n^3}} - \sqrt{|x|}\right|\\
	& =  sup_{x \in \mathbf{R}} \left|\left(\sqrt{|x| + \frac{1}{n^3}} - \sqrt{|x|}\right) \left(\frac{\sqrt{|x| + \frac{1}{n^3}} + \sqrt{|x|}}{\sqrt{|x| + \frac{1}{n^3}} + \sqrt{|x|}}\right)\right| \\
	& = sup_{x \in \mathbf{R}} \left|\frac{\frac{1}{n^3}}{\sqrt{|x| + \frac{1}{n^3}} + \sqrt{|x|}}\right|
\end{align*}

Da $|x|$ positiv ist, wird das Supremum von $\left|\frac{\frac{1}{n^3}}{\sqrt{|x| + \frac{1}{n^3}} + \sqrt{|x|}}\right|$ bei $x = 0$ angenommen. Es gilt somit

\begin{align*}
 & sup_{x \in \mathbf{R}} |f_n(x) - f(x)| = sup_{x \in \mathbf{R}} \left|\frac{\frac{1}{n^3}}{\sqrt{|x| + \frac{1}{n^3}} + \sqrt{|x|}}\right|\\
  &= \frac{\frac{1}{n^3}}{\sqrt{\frac{1}{n^3}}} = \frac{1}{n^{\frac{3}{2}}}
\end{align*}
Somit konvergiert die Funktionenfolge $f_n$ auf $\mathbf{R}$ gleichmässig gegen $f$.




% Grenzwerte von Funktionen
  \subsection {Grenzwerte von Funktionen}
Sei $f:\mathbf{D} \rightarrow \mathbf{R}$, $x_0 \in \mathbf{R}$ ein Häufungspunkt von $\mathbf{D}$.
Dann ist $A \in \mathbf{R}$ der Grenzwert von $f(x)$ für $x \rightarrow x_0$, bezeichnet mit
$$
  \lim_{x \rightarrow x_0} f(x) = A
$$
falls $\forall \epsilon > 0 \;\;\; \exists \delta > 0$ so dass
$$
  \forall x \in \mathbf{D} \cap (]x_0-\delta, x_0 + \delta[ \setminus \{x_0\}) : |f(x) - A| < \epsilon
$$
% Rechnen mit Limes Für Funktionen
\subsection{Rechnen mit Limes Für Funktionen}
Seien die Funktionen $f, g: \mathbf{D} \rightarrow \mathbf{R}$ konvergent mit
$\lim_{x \rightarrow x_0} f(x) = A$ und $\lim_{x \rightarrow x_0} g(x) = B$. Sei weiter
$x_0 \in \mathbf{R}$ ein Häufungspunkt von $\mathbf{D}$.
Dann gilt:
\begin{enumerate}[(i)]
  \item $\lim_{x \rightarrow x_0} (f + g)(x) = \lim_{x \rightarrow x_0} f(x) + \lim_{x \rightarrow x_0} g(x)$
  \item $\lim_{x \rightarrow x_0} (f * g)(x) = \lim_{x \rightarrow x_0} f(x) * \lim_{x \rightarrow x_0} g(x)$
  \item Sei $f, g: \mathbf{D} \rightarrow \mathbf{R} \text{ mit } f \leq g$. Dann folgt:
  $$
    \lim_{x \rightarrow x_0} f(x) \leq \lim_{x \rightarrow x_0} g(x)
  $$
  falls beide Grenzwerte existieren.
  \item Seien $\mathbf{D}, \mathbf{E} \subset \mathbf{R}$ und $f: \mathbf{D} \rightarrow \mathbf{E}$ eine Funktion. Wir nehmen an, dass $$y_0 := \lim_{x \rightarrow x_0} f(x)$$ existiert
  und $y_0 \in \mathbf{E}$. Falls $g:\mathbf{E} \rightarrow \mathbf{R}$ stetig in $y_0$ folgt:
  $$\lim_{x \rightarrow x_0} g(f(x)) = g(y_0)$$
\end{enumerate}
% Sandwichsatz für Funktionen
  \subsection{Sandwichsatz für Funktionen}
Falls $g_1 \leq f \leq g_2$ und
  $$
  \lim_{x \rightarrow x_0} g_1(x) = \lim_{x \rightarrow x_0} g_2(x)
  $$
  dann existiert $\lim_{x \rightarrow x_0} f(x)$ und
  $$
  \lim_{x \rightarrow x_0} f(x) = \lim_{x \rightarrow x_0} g_1(x)
  $$
% Links- und Rechtsseitige Grenzwerte
 \subsection{Links- und Rechtsseitige Grenzwerte}
Sei $f: \mathbf{D} \rightarrow \mathbf{R}$, $x_0 \in \mathbf{R}$. Wir nehmen an, dass
$x_0$ ein Häufungspunkt von $\mathbf{D} \;\cap\; ]x_0, +\infty[$ ist. Falls der Grenzwert
der Eingeschränkten Funktionen $f$ im Bereich $\mathbf{D} \;\cap\; [x_0, +\infty[$ für
$x \rightarrow x_0$ existiert, wird er mit $\lim_{x \rightarrow x_0^+} f(x)$ bezeichnet und
nennt sich \textbf{rechtsseitiger Grenzwert} von $f$ bei $x_0$.\\
Der \textbf{linksseitige Grenzwert} ist analog definiert für den Bereich $\mathbf{D} \;\cap\; ]-\infty, x_0]$
für $x \rightarrow x_0$, falls er existiert. Es wird mit $\lim_{x \rightarrow x_0^-} f(x)$ bezeichnet.\\
Besitzt die Funktion $f(x)$ an der Stelle $x_0$ den Grenzwert $L$, so gilt:
$$
  \lim_{x \rightarrow x_0+} f(x) = \lim_{x \rightarrow x_0-} f(x) = \lim_{x \rightarrow x_0} f(x) = L
$$
\newpage
\section{Reihen}
\hypertarget{sec:1}{}
% Konvergenz
  \subsection{Konvergenz}
Die Reihe $\sum_{k = 1}^{\infty} a_k$ ist konvergent, falls die Folge der
Partialsummen $(S_n)_{n \geq 1} = \sum_{k = 1}^{n} a_k$
konvergiert. In diesem Fall definieren wir:
\begin{align*}
  \sum_{k = 1}^{\infty} a_k := \lim_{n \rightarrow \infty} S_n
\end{align*}\\
% Monotone Konvergenz
  \subsection {Monotone Konvergenz}
Sei $\sum_{k = 1}^\infty a_k$ eine Reihe mit $a_k \geq 0 \;\; \forall k \in \mathbf{N}$.
Die Reihe konvergiert genau dann, wenn die Folge der Parialsummen $(S_n)_{n \geq 1}$ nach
oben beschränkt ist.
% Cauchy Kriterium
\subsection{Cauchy Kriterium}
Die Reihe $\sum_{k = 1}^{\infty} a_k$ ist geau dann konvergent, falls:
$$
  \forall \epsilon > 0 \; \exists N \geq 1 \text{ mit } \left| \sum_{k = n}^m a_k \right| < \epsilon \; \; \; \forall m \geq n \geq N
$$
% Rechnen mit Konvergenten Reihen
\subsection{Rechnen mit Konvergenten Reihen}
Seien $\sum_{k = 1}^\infty a_k$ und $\sum_{k = 1}^\infty b_k$ konvergent, sowie $\alpha \in \mathbf{C}$.
Dann gilt:
\begin{enumerate}[(i)]
  \item $\sum_{k = 1}^\infty (a_k + b_k)$ ist konvergent und $\sum_{k = 1}^\infty (a_k + b_k) = \sum_{k = 1}^\infty a_k + \sum_{k = 1}^\infty b_k$
  \item $\sum_{k = 1}^\infty \alpha * a_k$ ist konvergent und $\sum_{k = 1}^\infty \alpha * a_k = \alpha * \sum_{k = 1}^\infty a_k$
\end{enumerate}
Konvergieren die Reihen $\sum_{i = 0}^\infty a_i$ und $\sum_{j = 0}^\infty b_j$ absolut,
so gilt zusätzlich für das Cauchy Produkt:
\begin{enumerate}[(iii)]
  \item $\sum_{n = 0}^\infty (\sum_{j = 0}^n a_{n - j} * b_j) = (\sum_{i = 0}^\infty a_i) * (\sum_{j = 0}^\infty b_j)$
\end{enumerate}
% Absolute Konvergenz
\subsection {Absolute Konvergenz}
Eine Reihe $\sum_{k = 1}^{\infty} a_k$ heisst \textbf{absolut konvergent},
falls $\sum_{k = 1}^{\infty} |a_k|$ konvergiert. Weiter sind \textbf{absolut konvergente Reihen
auch konvergent} und es gilt:
$$
  \left| \sum_{k = 1}^{\infty} a_k \right| \leq \sum_{k = 1}^{\infty} |a_k|
$$
% Reihen Umordnung

\subsection{Reihen Umordnung}

Konvergiert $\sum_{n = 1}^\infty a_n$ absolut, dann konvergiert auch jede Umordnung
der Reihe $a_n' = a_{\phi(n)}$ (wobei $\phi$ bijektiv) und hat denselben Grenzwert.

% Potenzreihe
\subsection {Potenzreihe}
Die Potenzreihe $\sum_{k = 0}^\infty c_k z^k$ konvergiert absolut für alle $z \in \mathbf{C}$
mit $|z| < \rho$ (divergiert bei $|z| > \rho$), wobei
$$
  \rho := \begin{cases}
    +\infty &\text{Falls} \limsup_{k \rightarrow \infty} \sqrt[k]{|c_k|} = 0\\
    \frac{1}{\limsup_{k \rightarrow \infty} \sqrt[k]{|c_k|}} &\text{Falls} \limsup_{k \rightarrow \infty} \sqrt[k]{|c_k|} > 0\\
  \end{cases}
$$
\begin{center}
  \color{red}
  \textbf{Ränder Prüfen!}
\end{center}
% Nullfolgenkriterium
\subsection{Nullfolgenkriterium}
$$
  \lim_{n \rightarrow \infty} |a_n| \neq 0 \Rightarrow \sum_{n = 0}^\infty a_n \text{ divergiert}
$$
% Wurzelkriterium

\subsection{Wurzelkriterium}

\begin{align*}
  \limsup_{n \rightarrow \infty} \sqrt[n]{|a_n|} < 1 &\Rightarrow \sum_{n = 1}^\infty a_n \text{ konvergiert absolut.}\\
  \limsup_{n \rightarrow \infty} \sqrt[n]{|a_n|} > 1 &\Rightarrow \sum_{n = 1}^\infty a_n \text{ und } \sum_{n = 1}^\infty |a_n| \text{ divergieren.}
\end{align*}
% Quotientenkriterium

\subsection{Quotientenkriterium}

Sei $(a_n)_{n \geq 1}$ eine Folge mit $a_n \neq 0$
\begin{align*}
  \limsup_{n \rightarrow \infty} \frac{|a_{n+1|}}{|a_n|} < 1 &\Rightarrow \sum_{n = 1}^\infty a_n \text{ konvergiert absolut.}\\
  \liminf_{n \rightarrow \infty} \frac{|a_{n+1|}}{|a_n|} > 1 &\Rightarrow \sum_{n = 1}^\infty a_n \text{ divergiert.}
\end{align*}
% Vergleichssatz

\subsection{Vergleichssatz}

Seien $\sum_{k = 1}^\infty a_k$ und $\sum_{k = 1}^\infty b_k$ Reihen mit:
\begin{align*}
  0 \leq a_k &\leq b_k \; \; \; \forall k \geq 1\\
  \sum_{k = 1}^\infty b_k \text{ konvergent} &\Rightarrow \sum_{k = 1}^\infty a_k \text{ konvergent}\\
  \sum_{k = 1}^\infty a_k \text{ divergent} &\Rightarrow \sum_{k = 1}^\infty b_k \text{ divergent}
\end{align*}
% Integraltest
\subsection{Integraltest}

Sei $f(x)$ eine stetige, positive und monoton fallende
Funktion auf $[k, \infty)$ und $f(n) = a_n$:
\begin{align*}
\int_{k}^{\infty} f(x) dx \text{ konvergiert} \Rightarrow \sum_{n = k}^{\infty} a_n \text{ konvergiert}\\
\int_{k}^{\infty} f(x) dx \text{ divergiert}  \Rightarrow \sum_{n = k}^{\infty} a_n \text{ divergiert}
\end{align*}
% Leibniz Kriterium

\subsection{Leibniz Kriterium}

Sei $(a_n)_{n \geq 1}$ monoton fallend mit $a_n \geq 0 \; \; \; \forall n \geq 1$ und
$\lim_{n \rightarrow \infty} a_n = 0$. Dann konvergiert
$$
  S := \sum_{k = 1}^\infty (-1)^{k + 1} a_k
$$
und es gilt: $a_1 - a_2 \leq S \leq a_1$
% Gleichmässige Konvergenz

\subsection{Gleichmässige Konvergenz}

Die Reihe $\sum_{k=0}^\infty f_k(x)$ konvergiert gleichmässig in $\mathbf{D}$ falls die durch
$$
  S_n(x) := \sum_{k=0}^n f_k(x)
$$
definierte Funktionenfolge gleichmässig konvergiert.\\
Gilt weiter, dass $f_n: \mathbf{D} \subset \mathbf{R} \rightarrow \mathbf{R}$ eine Folge
stetiger Funktionen ist und eine Folge $C_n$ existiert, so dass
\begin{align*}
  |f_n(x)| \leq C_n \;\;\; \forall x \in \mathbf{D}
\end{align*}
und dass $\sum_{n = 0}^\infty c_n$ konvergiert, dann konvergiert die Reihe
$\sum_{n = 0}^\infty f_n(k)$ gleichmässig in $\mathbf{D}$ und deren Grenzwert
\begin{align*}
  f(x) := \sum_{n = 0}^\infty f_n(k)
\end{align*}
ist eine in $\mathbf{D}$ stetige Funktion.
\subsection{Strategie - Konvergenz von Reihen}
\begin{enumerate}
 \item Ist Reihe ein bekannter Typ? (Teleskopieren, Geometrische/Harmonische Reihe, Zetafunktion, ...)
 \item Ist $\limn a_n = 0$? Wenn nein, divergent.
 \item Quotientenkriterium \& Wurzelkriterium anwenden
 \item Vergleichssatz anwenden, Vergleichsreihen suchen
 \item Leibnizkriterium anwenden
 \item Integral-Test anwenden (Reihe zu Integral)
\end{enumerate}
\newpage
\section{Stetigkeit}
\hypertarget{sec:2}{}
% Stetigkeit einer Funktion in einem Punk

\subsection{Stetigkeit einer Funktion in einem Punk}

Sei $\mathbf{D} \subseteq \mathbf{R}$, $x_0 \in \mathbf{D}$. Die Funktion
$f: \mathbf{D} \rightarrow \mathbf{R}$ ist in $x_0$ \textbf{stetig}, falls
es für jedes $\epsilon > 0$ ein $\delta > 0$ gibt, so dass für alle $x \in \mathbf{D}$
gilt:
\begin{align*}
  |x - x_0| < \delta \Rightarrow |f(x) - f(x_0)| < \epsilon 
\end{align*}
Hier noch eine äquivalente Definition: Falls für jede Folge $(a_n)_{n \geq 1}$ mit
$\lim_{n \rightarrow \infty} a_n = x_0$ folgendes gilt:
\begin{align*}
  f(\lim_{n \rightarrow \infty} a_n) = f(x_0) = \lim_{n \rightarrow \infty} f(a_n)
\end{align*}
ist die Funktion $f$ in $x_0$ stetig.
% Stetigkeit einer Funktion

\subsection{Stetigkeit einer Funktion}

Die Funktion $f: \mathbf{D} \rightarrow \mathbf{R}$ ist \textbf{stetig}, falls sie
in jedem Punkt $x \in \mathbf{D}$ stetig ist.
% Gleichmässige Stetigkeit einer Funktion

\subsection{Gleichmässige Stetigkeit einer Funktion}

Eine Funktion $f:\mathbf{D} \rightarrow \mathbf{R}$ ist in $\mathbf{D}$ gleichmässig
stetig falls $$ \forall \epsilon > 0 \; \exists \delta_{\epsilon} > 0 \;\forall x, y \in \mathbf{D}\;\;\; |x-y| < \delta \Rightarrow |f(x) - f(y)| < \epsilon $$
wobei Funktionen $f:[a,\,b] \rightarrow \mathbf{R}$ welche in einem kompakten Invervall stetig sind im selben Intervall glm. stetig sind.\\
\textbf{Beispiel}: Ist die Funktion gleichmässig stetig? 
$$
f: [0, \infty)  \rightarrow \mathbf{R}, x \rightarrow \sqrt{x}
$$
Wir fixieren ein $\epsilon > 0$. Wir suchen $\delta > 0$, sodass für alle $x, y \in \Omega$ mit $|x - y| < \delta$ Folgendes gilt:
$$
|f(x) - f(y)| < \epsilon
$$
Die Schwierigkeit bei den Aufgaben, wo nach der gleichmässigen Stetigkeit gefragt wird, ist es, ein $\delta$ zu finden, das unabhängig von $x, y$ ist. Wie kann man in solchen Situationen vorgehen? Man vernucht, den Term $f(x) - f(y)$ durch einen Ausdruck der Form $C|x - y|$ abzuschätzen. In diesem spezifischen Fall, benutzen wir folgende Abschätzung:
$$
|\sqrt{x} - \sqrt{y}| <= \sqrt{x - y} \overset{!}{<} \epsilon \rightarrow |x - y| < \epsilon^2 =: \delta
$$

% Rechnen mit Stetigkeit

\subsection{Rechnen mit Stetigkeit}

Sei $x_0 \in \mathbf{D} \subset \mathbf{R}$, $\lambda \in \mathbf{R}$ und $f: \mathbf{D} \rightarrow \mathbf{R}$,
$g: \mathbf{D} \rightarrow \mathbf{R}$ beide in $x_0$ stetig:
\begin{enumerate}[(i)]
  \item Dann sind $f + g$, $\lambda * f$, $f * g$ stetig in $x_0$
  \item Falls $g(x_0) \neq 0$ dann ist
  \begin{align*}
    \frac{f}{g}: \mathbf{D} \cap \{x \in \mathbf{D}: g(x_0) \neq 0\} &\rightarrow \mathbf{R}\\
    x &\rightarrow \frac{f(x)}{g(x)}
  \end{align*}
  stetig in $x_0$.
  \item Polynomiale Funktionen sind auf ganz $\mathbf{R}$ stetig
  \item Die Trigonometrischen Funktionen $sin: \mathbf{R} \rightarrow \mathbf{R}$ und $cos: \mathbf{R} \rightarrow \mathbf{R}$ sind stetig
  \item Die Exponentialfunktion $e^x$ ist auf ganz $\mathbf{R}$ stetig.
  \item Seien $P, \;Q$ polynomiale Funktionen auf $\mathbf{R}$ mit $Q \neq 0$.
  Seien $x_1, \cdots, x_m$ die Nullstellen von $Q$. Dann ist
  \begin{align*}
    \frac{P}{Q} : \mathbf{R} \setminus \{x_1, \cdots, x_m\} &\rightarrow \mathbf{R}\\
    x &\rightarrow \frac{P(x)}{Q(x)}
  \end{align*}
  stetig.
  \item Seien $\mathbf{D}_1, \mathbf{D}_2 \subset \mathbf{R}$ zwei Teilmengen,
  $f:\mathbf{D}_1 \rightarrow \mathbf{D}_2$ und $g:\mathbf{D}_2 \rightarrow \mathbf{R}$
  funktionen, sowie $x_0 \in \mathbf{D}_1$. Falls $f$ in $x_0$ und $g$ in $f(x_0)$ stetig
  sind, so ist $g(f(x)): \mathbf{D}_1 \rightarrow \mathbf{R}$ in $x_0$ stetig
  \item Sei $\mathbf{D} \subset \mathbf{R}$, $x_0 \in \mathbf{D}$ und $f,\;g: \mathbf{D} \rightarrow \mathbf{R}$
  stetig in $x_0$. Dann sind $|f|$, $max(f,\;g)$ und $min(f, \; g)$ stetig in $x_0$.
\end{enumerate}
% Zwischenwertsatz

\subsection{Zwischenwertsatz}

Sei $\mathbf{I} \subset \mathbf{R}$ ein Intervall, $f:\mathbf{I} \rightarrow \mathbf{R}$ eine
stetige funktion und $a, b \in \mathbf{I}$. Für jedes $y$ zwischen $f(a)$ und $f(b)$
gibt es (mindestens) ein $c$ zwischen $a$ und $b$ mit $f(c) = y$.\\
Es gibt folgende typischen Anwendungsszenarien:
\begin{enumerate}[(i)]
  \item Sei $f:[a, b] \rightarrow \mathbf{R}$ stetig. Falls $f(a)*f(b) < 0$, dann
  $\exists c \in ]\,a, b\,[$ mit $f(c) = 0$ (also eine Nullstelle)
  \item Sei $P(x) = a_nx^n + \cdots + a_0$ ein Polynom mit $a_n \neq 0$ und $n$ ungerade.
  Dann besitzt $P$ mindestens eine Nullstelle in $\mathbf{R}$.
\end{enumerate}
% Min-Max Satz

\subsection{Min-Max Satz}

Sei $f:\mathbf{I}=[a, b] \rightarrow \mathbf{R}$ stetig auf einem kompakten Intervall.
Dann gibt es $u \in [a, b]$ und $v \in [a, b]$ mit:
$$
  f(u) \leq f(x) \leq f(v) \;\;\; \forall x \in [a, b]
$$
Insbesondere ist $f$ beschränkt.
% Satz der Umkehrabbildung

\subsection{Satz der Umkehrabbildung}

Sei $\mathbf{I}$ ein Intervall. Sei $f:\mathbf{{I} \rightarrow \mathbf{R}}$ stetig, \textbf{streng} monoton
wachsend. Dann ist das Bild von $f(\mathbf{I}) =: J$ ein Intervall und die Umkehrfunktion
$f^{-1}: \mathbf{J} \rightarrow \mathbf{I}$ ist stetig, streng monoton wachsend.
\begin{enumerate}[(i)]
  \item Sei $n \geq 1$. Dann ist $f:[0, \infty[ \rightarrow [0, \infty[$ als $x \rightarrow x^n$
  streng monoton wachsend, stetig und surjektiv. Nach dem Umkehrsatz existiert eine streng monoton
  wachsende stetige Umkehrabbildung $f^{-1}:[0, \infty[ \rightarrow [0, \infty[$ als $x \rightarrow \sqrt[n]{x}$
\end{enumerate}
% Stetigkeit gesplitteter Funktionen

\subsection{Stetigkeit gesplitteter Funktionen}

Sind alle abschnitte einer gesplitteten Funktion stetig, müssen wir nur die Übergangstellen prüfen.
Gilt an diesen Stellen $x_0$ $$\lim_{x \rightarrow x_0^+} f(x) = \lim_{x \rightarrow x_0^-} f(x) = f(x_0)$$
so ist die Funktion stetig.
\newpage
\section{Ableiten}
\hypertarget{sec:3}{}
% Differenzierbarkeit

\subsection{Differenzierbarkeit}

Sei $f: \mathbf{D} \rightarrow \mathbf{R}$ und $x_0 \in \mathbf{D}$ ein Häufungspunkt.
Die Funktion $f$ heisst in $x_0$ differenzierbar, falls der Grenzwert $$\lim_{x \rightarrow x_0} \frac{f(x) - f(x_0)}{x - x_0} = \lim_{h \rightarrow 0} \frac{f(x_0 + h) - f(x_0)}{h}$$
existiert. In diesem fall wird der Grenzwert mit $f'(x_0)$ oder $\frac{df}{dx} (x_0)$ bezeichnet und heisst die \textbf{Ableitung}
(oder das Differential) von $f$ an der Stelle $x_0$.
% Differenzierbarkeit & Stetigkeit

\subsection{Differenzierbarkeit \& Stetigkeit}

$$
  f \text{ differenzierbar in } x_0 \Rightarrow f \text{ stetig in } x_0
$$
% Rechenregeln der Ableitung

\subsection{Rechenregeln der Ableitung}

\begin{enumerate}[(i)]
  \item $(f + g)'(x_0) = f'(x_0) + g'(x_0)$
  \item $(f * g)'(x_0) = f'(x_0)*g(x_0) + f(x_0)*g'(x_0)$
  \item $(\frac{f}{g})'(x_0) = \frac{f'(x_0) * g(x_0) - f(x_0) * g'(x_0)}{(g(x_0))^2}$ für $g(x_0) \neq 0$
  \item $(g \circ f)'(x_0) = g'(f(x_0)) * f'(x_0)$
\end{enumerate}
% Aussagen der Ableitung

\subsection{Aussagen der Ableitung}
\begin{enumerate}
  \item $f$ besitzt ein lokales Minimum in $x_0$, wenn $f'(x_0) = 0$ und $f''(x_0) > 0$ oder falls das Vorzeichen von $f'$ um $x_0$ von $-$ zu $+$ wechselt.
  \item $f$ besitzt ein lokales Maximum in $x_0$, wenn $f'(x_0) = 0$ und $f''(x_0) < 0$ oder falls das Vorzeichen von $f'$ um $x_0$ von $+$ zu $-$ wechselt.
  \item $f$ besitzt ein lokales Extremum in $x_0$, wenn $f'(x_0) = 0$ und $f''(x_0) \ne 0$.
  \item $f$ besitzt einen Sattelpunkt in $x_0$, wenn $f'(x_0) = 0$ und $f''(x_0) = 0$.
  \item $f$ besitzt einen Wendepunkt in $x_0$, wenn $f''(x_0) = 0$.
  \item $f$ ist in $x_0$ konvex, wenn $f''(x_0) \ge 0$.
  \item $f$ ist in $x_0$ konkav, wenn $f''(x_0) \le 0$.
\end{enumerate}
% Umkehrsatz
\subsection{Umkehrsatz}

Sei $f: \mathbf{D} \rightarrow \mathbf{E}$ eine bijektive Funktion,
$x_0 \in \mathbf{D}$ ein Häufungspunkt. Wir
nehmen an, dass $f$ in $x_0$ differenzierbar ist und $f'(x_0) \neq 0$.
Dann ist $y_0 := f(x_0)$ ein Häufungspunkt von $\mathbf{E}$
und $f^{-1}$ in $y_0$ differenzierbar. Es gilt:
$$
  (f^{-1})'(y_0) = \frac{1}{f'(x_0)}
$$
% Satz von Rolle

\subsection{Satz von Rolle}

Sei $f:[a, b] \rightarrow \mathbf{R}$ stetig und in $]\,a, b\,[$ differenzierbar. Falls
$f(a) = f(b)$, dann gibt es mindestens einen Punkt $\xi \in ]\,a, b\,[$ mit $f'(\xi) = 0$.
% Mittelwertsatz

\subsection{Mittelwertsatz}

Sei $f:[a, b] \rightarrow \mathbf{R}$ stetig und in $]\,a, b\,[$ differenzierbar. Dann
gibt es $\xi \in ]\,a, b\,[$ mit $f(b) - f(a) = f'(\xi) * (b-a)$
% l'Hôpital

\subsection{l'Hôpital}

Seien $f, g: \; ]\,a, b\,[ \rightarrow \mathbf{R}$ differenzierbar mit $g'(x) \neq 0 \;\;\; \forall x \in ]\,a, b\,[$.
Falls $$\lim_{x \rightarrow b^-} f(x) = 0 \text{ und } \lim_{x \rightarrow b^-} g(x) = 0$$
sowie $$\lim_{x \rightarrow b^-} \frac{f'(x)}{g'(x)} =: \lambda$$ existiert, dann folgt, dass
$$\lim_{x \rightarrow b^-} \frac{f(x)}{g(x)} = \lim_{x \rightarrow b^-} \frac{f'(x)}{g'(x)}$$
wobei der Satz auch gilt wenn
\begin{enumerate}[(i)]
  \item falls $b = +\infty$
  \item falls $x \rightarrow a^+$
  \item falls $\lambda = +\infty$
  \item falls $\lim f = \lim g = \infty$
\end{enumerate}

% Konvexität
\subsection{Konvexität}
\begin{center}
  \includegraphics[scale=0.4]{konvex.png}
\end{center}
\begin{enumerate}[(i)]
  \item Die Summe zweier konvexer Funktionen ist konvex
  \item $f$ ist genau dann konvex, falls für alle $x_0 < x < x_1$ in $\mathbf{I}$
  $$
    \frac{f(x) - f(x_0)}{x - x_0} \leq \frac{f(x_1) - f(x)}{x_1 - x_0}
  $$
\end{enumerate}
% Höhere Ableitungen

\subsection{Höhere Ableitungen}

Sei $f: \mathbf{D} \rightarrow \mathbf{R}$ differenzierbar.
\begin{enumerate}[(i)]
  \item Für $n \geq 2$ ist $f$ $n$-mal differenzierbar in $\mathbf{D}$
  falls $f^{(n-1)}$ in $\mathbf{D}$ differenzierbar ist. Dann ist $f^{(n)} := (f^{(n-1)})'$
  und nennt sich die $n$-te Ableitung von $f$. Wobei zu beachten ist, dass: $n$-mal differenzierbar $\Rightarrow$ $(n-1)$-mal stetig differenzierbar.
  \item Die Funktion $f$ ist $n$-mal \textbf{stetig Differenzierbar}, falls sie $n$-mal
  differenzierbar ist und $f^{(n)}$ stetig ist. Wir definieren weiter die Menge $$C^n(\mathbf{D}) = \{f: \mathbf{D} \rightarrow \mathbf{R} \;|\; f\;n\text{-mal stetig diff'bar}\}$$
  \item Die Funktion $f$ ist in $\mathbf{D}$ \textbf{glatt} falls sie $\forall n \geq 1$ $n$-mal differenzierbar ist.
  $$C^\infty(\mathbf{D}) = \{f: \mathbf{D} \rightarrow \mathbf{R} \;|\; f\;\text{glatt}\}$$
\end{enumerate}
% Rechenregeln höherer Ableitungen

\subsection{Rechenregeln höherer Ableitungen}

Seien $f, g: \mathbf{D} \rightarrow \mathbf{R}$ $n$-mal differenzierbar:
\begin{enumerate}[(i)]
  \item $(f + g)^{(n)} = f^{(n)} + g^{(n)}$
  \item $(f * g)^{(n)} = \sum_{k = 0}^{n} \begin{pmatrix}
    n\\
    k
  \end{pmatrix} f^{(k)} * g^{(n - k)}$
  \item $\frac{f}{g}$ ist $n$-mal differenzierbar falls $g(x) \neq 0 \;\;\; \forall x \in \mathbf{D}$
  \item $(g \circ f)$ ist $n$-mal differenzierbar
  \item $e^x$, $sin(x)$ und $cos(x)$ sind glatte Funktionen
  \item Alle Polynome sind glatte Funktion
\end{enumerate}
% Taylor Approximation

\subsection{Taylor Approximation}

Sei $f: [\, a, b\,] \rightarrow \mathbf{R}$ stetig und in $]\, a, b\,[$
$(n+1)$-mal differenzierbar. Für jedes $a < x \leq b$ gibt es $\xi \in ]\,a, x\,[$ mit:
$$
f(x) = \sum_{k = 0}^n \frac{f^{(k)}(a)}{k!} * (x - a)^k + \frac{f^{(n + 1)}(\xi)}{(n + 1)!} * (x - a)^{n + 1}
$$
Man bemerke: der letzte Term $\frac{f^{(n + 1)}(\xi)}{(n + 1)!} * (x - a)^{n + 1}$ wird meist zur Fehlerabschätzung innerhalb eines Bereichs von $a$ verwendet.
Als Beispiel betrachte man $p(x) = x^3 + x + 1$ an der Stelle $a = 1$. Hier ist
die Taylor Approximation $$T_3 = 3 + 4(x - 1) + \frac{6}{2!}(x-1)^2 + \frac{6}{3!}(x-1)^3 = p(x)$$
und der Fehler für $\xi \in ]\,0, 2\,[$
$$
  |\text{Fehler}| \leq \frac{f^{(4)}(\xi)}{(4)!} * (x - 1)^{4} \leq \frac{0}{(4)!} * (1)^{4} = 0\\
$$
Ein weiteres Beispiel: Approximiere $\sqrt{9.2}$ mit einen Taylor Polynom
zweiten Grades.
\begin{align*}
  f(x) &= \sqrt{x}\\
  f'(x) &= \frac{1}{2} * x^{-0.5}\\
  f''(x) &= - \frac{1}{4} * x^{-1.5}\\
  f'''(x) &= \frac{3}{8} * x^{-2.5}\\
  T_2 f(x) &= f(x_0) + f'(x_0) * (x - x_0) + f''(x_0) * (x - x_0)^2\\
  R &= \frac{f'''(\xi)}{3!} * (x - x_0)^3 \text{ für $\xi \in (9, 9.2)$}\\
  &\Rightarrow x_0 = 9, \xi = 9\\
\end{align*}

% Spezielle Punkte bestimmen

\subsection{Spezielle Punkte bestimmen}

Sei $n \geq 0$, $a < x_0 < b$ und $f: [\,a, b\,] \rightarrow \mathbf{R}$ in $]\,a, b\,[$ $(n+1)$-mal stetig differenzierbar.
Annahme: $f'(x_0) = f''(x_0) = \cdots = f^{(n)}(x_0) = 0$
\begin{enumerate}[(i)]
  \item Falls $n$ gerade ist und $x_0$ eine lokale Extremalstelle, folgt $f^{(n+1)}(x_0) = 0$
  \item Falls $n$ ungerade ist und $f^{(n+1)}(x_0) > 0$ so ist $x_0$ eine strikte lokale Minimalstelle
  \item Falls $n$ ungerade ist und $f^{(n+1)}(x_0) < 0$ so ist $x_0$ eine strikte lokale Maximalstelle
\end{enumerate}
Ist $x_0$ jedoch keine Extremalstelle ($(i)$ von oben nicht erfüllt) bleiben zwei Optionen:
\begin{enumerate}[(i)]
  \item $f'(x_0) = 0 \land x_0 \text{ keine Extremalstelle} \Rightarrow x_0$ ist ein Sattelpunkt
  \item $f''(x_0) = 0 \land x_0 \text{ keine Extremalstelle} \Rightarrow x_0$ ist ein Wendepunkt
\end{enumerate}
% Integrale Ableiten

\subsection{Integrale Ableiten}

Hier ein Beispiel für die Ableitung eines Integrals:
\begin{align*}
  f(x) = -\int_2^{x^2} e^{-t^2} dt\\
  h(x) = e^{-t^2}\\
  \Rightarrow f(x) = - H(x^2) + H(2)\\
  \Rightarrow f'(x) = - h(x^2) 2x = -e^{-x^4} 2x
\end{align*}
\newpage
\section{Integrieren}
\hypertarget{sec:4}{}
% Partition

\subsection{Partition}

Eine Zerlegung eines Intervalls $I = [\,a, b\,]$. Ist eine endliche Teilmenge
$P = \{a = x_0,\, x_1,\, \cdots,\, x_n = b\} \subset I$ wobei $x_0 < x_1 < \cdots < x_n$ und $\{a,\, b\} \subset P$
\\
Man bemerke: eine Partition $P'$ ist eine verfeinerung von $P$ falls $P \subset P'$
% Feinheit einer Partition

\subsection{Feinheit einer Partition}

Die \textbf{Feinheit} der Partition ist definiert durch $\delta(P) := \max_{1 \leq i \leq n} \delta_i = \max_{1 \leq i \leq n} (x_i - x_{i - 1})$
% Riehmannsche Summe

\subsection{Riehmannsche Summe}

Sei $\xi_i \in I_i$ zwischen Punkten. Jede Summe der Form
$$
  S(f, P, \xi) := \sum_{i = 1}^n f(\xi_i) * (x_i - x_{i - 1}) = \sum_{i = 1}^n f(\xi_i) * \delta_i
$$
nennt man eine \textbf{Riehmannsche Summe} der Partition $P$ und den Zwischenpunkten $\xi = \{\xi_1,\, \cdots,\, \xi_n\}$
% Unter-/ Obersumme

\subsection{Unter-/ Obersumme}

Wir definieren die Untersumme
$$
  s(f, P) := \sum_{i = 1}^n (\inf_{x \in I_i} f(x)) * \delta_i
$$
und die Obersumme
$$
  S(f, P) := \sum_{i = 1}^n (\sup_{x \in I_i} f(x)) * \delta_i
$$
% Eigenschaften der Unter-/ Obersumme

\subsection{Eigenschaften der Unter-/ Obersumme}

Sei $f: [a,\, b] \rightarrow \mathbf{R}$ eine beschränkte Funktion, sowie $P,\, Q \in P(I)$.
\begin{enumerate}[(i)]
  \item $P \subset Q \Rightarrow s(f, P) \leq s(f, Q) \leq S(f, Q) \leq S(f, P)$
  \item $\sup_{P \in \mathcal{P}(I)} s(f, P) \leq \inf_{Q \in \mathcal{P}(I)} S(f, Q)$
\end{enumerate}
% (Riehmann) Integrierbar

\subsection{(Riehmann) Integrierbar}

Sei $f:[a,\,b] \rightarrow \mathbf{R}$ beschränkt. Wir definieren zuerst $s(f) := \sup_{P \in \mathcal{P}(I)} s(f, P)$ sowie analog $S(f) := \inf_{P \in \mathcal{P}(I)} S(f, P)$.\\
Gilt
$$
  s(f) = S(f)
$$
so ist die $f$ Riehmann-Integrierbar und wird mit $\int_a^b f(x)dx$ bezeichnet.\\
Weiter sind folgende Aussagen äquivalent:
\begin{enumerate}[(i)]
  \item $f:[a,\, b]\rightarrow \mathbf{R}$ ist integrierbar mit $A := \int_a^b f(x) dx$
  \item $\forall \epsilon > 0 \;\;\; \exists P \in \mathcal{P}(I) \text{ mit } S(f, P) - s(f, P) < \epsilon$
  \item $\forall \epsilon > 0 \; \exists \delta > 0 \text{ so dass für jede Partition } P \in \mathcal{P}(I) $ mit $ \delta(P) < \delta $ und $ \xi_1, \cdots, \xi_n $ Zwischenpunkten $x_{k-1} \leq \xi_k \leq x_k:\;\;\; |A - S(f, P, \xi)| < \epsilon$
  \item Der Grenzwert $\lim_{\delta(P) \rightarrow 0} S(f, P, \xi) = \int_a^b f(x)dx$ existiert
\end{enumerate}
% Integrierbarkeit schnell zeigen

\subsection{Integrierbarkeit schnell zeigen}

Es gilt weiter für $f,g:[a,\,b] \rightarrow \mathbf{R}$ beschränkt, integrierbar und $\lambda \in \mathbf{R}$:
\begin{enumerate}[(i)]
  \item $f$ stetig $\Rightarrow f$ Integrierbar
  \item $f\text{ ist monoton} \Rightarrow f \text{ ist integrierbar}$
  \item $f + g$, $\lambda * f$, $f * g$, $|f|$, $max(f, g)$, $min(f, g)$ sind integrierbar sowie auch $\frac{f}{g}$ falls $|g(x)| \geq \beta > 0 \;\;\; \forall x \in [a,\,b]$
  \item Jedes Polynom auf $[a,\, b]$ ist integrierbar, auch $\frac{P(x)}{Q(x)}$ falls $Q(x)$ keine Nullstelle besitzt.
\end{enumerate}
% Majoranten Kriterium

\subsection{Majoranten Kriterium}

\begin{enumerate}[(i)]
  \item Falls $|f(x)| \leq g(x) \;\;\; \forall x \geq a$ und $g(x)$ auf $[a,\,\infty[$ integrierbar ist, so ist $f$ auf $[a,\,\infty[$ integrierbar.
  \item Falls $0 \leq g(x) \leq f(x)$ und $\int_a^\infty g(x) dx$ divergiert, so divergiert auch $\int_a^\infty f(x) dx$
  \item Sei $f:[1, \infty[ \rightarrow [0, \infty[$ monoton fallend. Dann konvergiert $\sum_{k = 1}^\infty f(k)$ genau dann, wenn $\int_1^\infty f(x) dx$ konvergent und in diesem
  Fall gilt: $0 \leq \sum_{k = 1}^\infty f(k) - \int_1^\infty f(x) dx \leq f(1)$
\end{enumerate}
% Rechenregeln für Integrale

\subsection{Rechenregeln für Integrale}

Es gelten folgene Rechenregeln:
\begin{enumerate}[(i)]
  \item $\int_a^b f(x) dx + \int_b^c f(x) dx = \int_a^c f(x) dx $ für $a < b < c$ mit $f:[a,\,c] \rightarrow \mathbf{R}$ und auf $[a,\,b]$ sowie $[b,\,c]$ integrierbar
  \item $\int_a^b(\alpha * f_1(x) + \beta * f_2(x)) dx = \alpha * \int_a^b f_1(x) dx + \beta * \int_a^b f_2(x) dx$ für $f_1$, $f_2: I \subseteq \mathbf{R} \rightarrow \mathbf{R}$ beschränkt Integrierbar mit endpunkten $a, b$ sowie $\alpha, \beta \in \mathbf{R}$
\end{enumerate}
% Abschätzungen von Integralen

\subsection{Abschätzungen von Integralen}

Es gibt folgende Absätzungen:
\begin{enumerate}[(i)]
  \item Seien $f, g: [a,\,b] \rightarrow \mathbf{R}$ beschränkt und integrierbar und $f(x) \leq g(x)\;\;\; \forall x \in [a,\,b]$. Dann folgt $\int_a^b f(x) dx \leq \int_a^b g(x) dx$
  \item Falls $f:[a,\,b] \rightarrow \mathbf{R}$ beschränkt und integrierbar folgt $|\int_a^b f(x) dx| \leq \int_a^b |f(x)| dx$
  \item $|\int_a^b f(x)g(x) dx| \leq \sqrt{\int_a^b f^2(x)} * \sqrt{\int_a^b g^2(x)}$
\end{enumerate}
% Mittelwertsatz der Integralrechnung

\subsection{Mittelwertsatz der Integralrechnung}

Seien $f,g:[a,\,b] \rightarrow \mathbf{R}$ wobei $f$ stetig und $g$ beschränkt integrierbar mit $g(x) \geq 0 \;\;\; \forall x \in [a,\,b]$ ist.
Dann gibt es $c \in [a,\,b]$ mit $$\int_a^b f(x) * g(x) dx = f(c) * \int_a^b g(x) dx$$
und falls $g \equiv 1$ erhalten wir $$\int_a^b f(x) dx = f(c) * (b - a)$$
% Stammfunktion

\subsection{Stammfunktion}

Sei $a < b$, $f:[a,\,b] \rightarrow \mathbf{R}$ stetig. Eine Funktion $F:[a,\,b] \rightarrow \mathbf{R}$ heisst
\textbf{Stammfunktion} von $f$ falls $F$ stetig differenzierbar in $[a,\, b]$ ist und $F' = f$ in $[a,\,b]$ gilt.
% Fundamentalsatz der Analysis

\subsection{Fundamentalsatz der Analysis}

Sei $f:[a,\,b] \rightarrow \mathbf{R}$ stetig. Dann gibt es eine Stammfunktion $F$ von $f$, die bis auf eine additive Konstante eindeutig bestimmt ist und es gilt:
$$\int_a^b f(x) dx = F(b) - f(a)$$

% Partielle Integration

\subsection{Partielle Integration}

Seien $a < b$ reele Zahlen und $f,g:[a,\,b] \rightarrow \mathbf{R}$ stetig differenzierbar.
Dann gilt:
$$
  \int_a^b f(x) * g'(x) dx = f(x) * g(x) \Big|_a^b - \int_a^b f'(x) * g(x) dx
$$
beziehungsweise für unbestimmte Integrale
$$
\int f(x) * g'(x) dx = f(x) * g(x) - \int f'(x) * g(x) dx
$$

\begin{itemize}
\item Polynome ableiten, wiederholende Fuktionen ($sin(x), cos(x), e^x$) integrieren
\item manchmal mit 1 multiplizieren
\end{itemize}
% Methode der Substitution

\subsection{Methode der Substitution}

Die Methode der Substitution ist die Umkehrung der Kettenregel. Sei $a < b$, $\phi:[a,\,b] \rightarrow \mathbf{R}$
stetig differenzierbar und $I \subset \mathbf{R}$ ein Intervall mit $\phi([a,\,b]) \subset I$ und $f: I \rightarrow \mathbf{R}$ eine stetige Funktion.
Dann ist
$$
  \int_a^b f(\phi(t)) * \phi'(t) dt = \int_{\phi(a)}^{\phi(b)} f(x) dx
$$
wobei für unbestimmte Integrale gilt:
$$
  \int f(\phi(t)) * \phi'(t) dt + C = \int f(x) dx \Big|_{x = \phi(t)}
$$
Hier ein Beispiel:
\begin{align*}
  \int 2x * cos(x^2) dx \Rightarrow u = x^2, \frac{du}{dx} = 2x, dx = \frac{du}{2x}\\
  = \int 2x * cos(u) * \frac{du}{2x} = \int cos(u) du = sin(x^2) + C\\
\end{align*}
% Integration konvergenter Reihen

\subsection{Integration konvergenter Reihen}

Sei $f_n:[a,\,b] \rightarrow \mathbf{R}$ eine Folge von beschränkten integrierbaren
Funktionen die gleichmässig gegen eine Funktion $f:[a,\,b] \rightarrow \mathbf{R}$ konvergiert.
Dann ist $f$ beschränkt, integrierbar und
$$
  \lim_{n \rightarrow \infty} \int_a^b f_n(x) dx = \int_a^b \lim_{n \rightarrow \infty} f_n(x) dx = \int_a^b f(x) dx
$$
Es gilt weiter: Sei $f_n:[a,\,b] \rightarrow \mathbf{R}$ eine Folge beschränkter und integrierbarer Funktionen so dass $\sum_{n = 0}^\infty f_n$ auf $[a,\,b]$
gleichmässig konvergiert. Dann gilt
$$
  \sum_{n = 0}^\infty \int_a^b f_n(x) dx = \int_a^b \sum_{n = 0}^\infty f_n(x) dx
$$
Es gilt weiter: Sei $f(x) := \sum c_k x^k$ eine Potenzreihe mit positiven Konvergenzradius $\rho > 0$.
Dann ist für jedes $0 \leq r < \rho$ $f$ auf $[-r,\,r]$ integrierbar und es gilt $\forall x \in ]-\rho,\,\rho[$
$$
  \int_0^x f(t) dt = \sum_{k = 0}^\infty \frac{c_k}{k+1} x^{k+1}
$$
% Stirlings Estimate
%
%\subsection{Stirlings Estimate};
%
%$$n! = \sqrt{2\pi n} (\frac{n}{e})^n exp(\frac{1}{12n} + R_3(n)) $$
% Uneigentliche Integrale

\subsection{Uneigentliche Integrale}

Sei $f:[a,\,\infty[ \rightarrow \mathbf{R}$ beschränkt und integrierbar auf $[a,\,b]$ für alle $b > a$. Wir definieren
$$
  \lim_{b \rightarrow \infty} \int_a^b f(x) dx = \int_a^\infty f(x) dx
$$
falls existent und sagen dass $f$ auf $[a, \infty[$ integrierbar ist.\\
Analog: Sei $f$ eine Funktion auf jedem Intervall $[a+\epsilon,\,b]\;\;\;\forall \epsilon > 0$
beschränkt und integrierbar. $f:]a,\,b] \rightarrow \mathbf{R}$ ist integrierbar falls der Folgende Grenzwert existiert, welchen wir als
$$
  \lim_{\epsilon \rightarrow 0+} \int_{a + \epsilon}^b f(x) dx := \int_a^b f(x) dx
$$
definieren. (Gilt auch symmetrisch für $[a,\,b-\epsilon]\;\;\;\forall \epsilon > 0$)
Wobei wir anmerken, dass $$\int_{-\infty}^\infty f(x) dx := \int_{-\infty}^c f(x) dx + \int_c^\infty f(x) dx$$
\\
% Partialbruchzerlegung

\subsection{Partialbruchzerlegung}

Seien $P, Q$ Polynome mit $grad(P) < grad(Q)$ und $Q$ mit der Produktzerlegung $Q(x) = \prod_{j = 1}^l \big( (x- \alpha_j)^2 + \beta_j^2\big)^{m_j} \prod_{i = 1}^k (x - \gamma_i)^{n_i}$. Dann gibt es $A_{ij}, B_{ij}, C_{ij}$ reelle Zahlen mit
$$
  \frac{P(x)}{Q(x)} = \sum_{i = 1}^l \sum_{j = 1}^{m_i} \frac{(A_{ij} + B_{ij}x)}{\big( (x- \alpha_i)^2 + \beta_i^2\big)^j} + \sum_{i = 1}^k \sum_{j = 1}^{n_i} \frac{C_{ij}}{(x-\gamma_i)^j}
$$
Bei einer Partialbruchzerlegung geht man folgendermassen vor:
\begin{enumerate}[(i)]
  \item Sei $R(x) = \frac{P(x)}{Q(x)}$. Falls $grad(P) \geq grad(Q)$ wenden wir Polynomdivision an.
  \item $Q$ lässt sich nun als $Q(x) = \prod_{j = 1}^l \big( (x- \alpha_j)^2 + \beta_j^2\big)^{m_j} \prod_{i = 1}^k (x - \gamma_i)^{n_i}$ zerlegen. Das sind
  die Komplexen und reelen Nullstellen mit ihrer vielfachheit.
  \item Wir bilden nun die "hässliche" Summe von oben
  \item Wir bestimmen mithilfe von Koeffizientenvergleich (Nennerpolynom Multiplizieren) die unbekannten $A_{ij}, B_{ij}, C_{ij}$.
\end{enumerate}
Hier ein einfaches Beispiel:
\begin{align*}
  &\frac{5x + 1}{x^2 + x - 2} = \frac{5x + 1}{(x - 1)(x + 2)} = \frac{A}{x - 1} + \frac{B}{x + 2} \\
  &\Rightarrow \text{ Löse } 5x + 1 = A(x + 2) + B(x - 1) \\
  &\Rightarrow \text{ Setze } x = -2,\;1
\end{align*}
Mit mehreren Linearen Faktoren:
\begin{align*}
  &\frac{-2x^2 + x + 8}{x (x-2)^2} = \frac{A}{x} + \frac{B}{x - 2} + \frac{C}{(x-2)^2} \\
  &\Rightarrow \text{ Löse } -2x^2 + x + 8 = A(x - 2)^2 + Bx(x - 2) + Cx
\end{align*}
Ohne reelle Nullstellen:
\begin{align*}
  &\frac{2x^2 -3x + 3}{(x-1)(x^2 + 1)} = \frac{A}{x - 1} + \frac{Bx + C}{x^2 + 1} \\
  &\Rightarrow \text{ Löse } 2x^2 -3x + 3 = A(x^2 + 1) + (Bx + C)(x - 1) \\
  &\Rightarrow \text{ Setze } x = 0,\;1,\;2
\end{align*}
% Unbestimmte Integral

\subsection{Unbestimmte Integral}

Das Unbestimmte Integral ist die Menge aller Stammfunktionen und sozusagen fast alles von oben gilt. $+C$ ist \textbf{sehr} wichtig.
\newpage
\section{Sonstiges}
\subsection{Rewrite Function}
$h(x) = max{f(x), g(x)} = \frac{1}{2}(f(x) + g(x)) + \frac{1}{2}|f(x) -g(x)|$

\subsection{Stirling Formula}
$n! \sim \sqrt{2\pi n}(\frac{n}{e})^n, n \rightarrow \infty$

\subsection{Proof of "Null-Reihe"}
Eine Reihe konvergiert, wenn die Folge ihrer Partialsumme $(s_n) n \in \mathbb{N} $ mit: $s_n = \sum_{i=1}^{n} a_i$ konvergiert, das heisst, es existiert ein Granzwert $s$, sodass
	$\lim_{n \to \infty} s_n = s$  Durch Umstellung der Reihe und mit den Rechenregeln für Grenzwerte gild dann
	$\lim_{n \to \infty} a_n = \lim_{n \to \infty} (s_n - s_{n - 1} = \lim_{n \to \infty} s_n - \lim_{n \to \infty} s_{n-1} =  s - s = 0$

\hypertarget{sec:5}{}
% Injektiv/ Surjektiv

\subsection{Injektiv/ Surjektiv}

Injektiv: $\forall a, b \in X, \;f(a) = f(b) \Rightarrow a = b$\\
\textit{$f$ ist injektiv Beweis: wenn Ableitung $>$ 0: dann streng monton wachsend auf ganz $\R$ und somit injektiv}\\
%TODO!!
Surjektiv: $\forall y \in Y, \, \exists x \in X, f(x) = y$
\textit{$f$ ist surjektiv Beweis: anhand von Zwischenwertsatz beweisen}\\
% Suprenum

\subsection{Suprenum}

Sei $A \subseteq \mathbf{R}$, $ A \neq \emptyset$ und $A$ nach oben beschränkt. Dann gibt es
eine kleinste obere Schranke von $A$. Es gibt also ein $c \in \mathbf{R}$ so dass:
\begin{enumerate}
  \item $\forall a \in A \; \; \; a \leq c$
  \item Falls $\forall a \in A \; \; \; a \leq x$ ist $c \leq x$
\end{enumerate}
Man bezeichnet $c := \sup A$ 
% Infimum

\subsection{Infimum}

Analog zum Suprenum die grösste untere Schranke.
% Dreiecksungleichung

\subsection{Dreiecksungleichung}

\begin{center}
  $\forall x, y \in \mathbf{R} : ||x| - |y|| \leq |x \pm y| \leq |x| + |y|$
\end{center}
% Bernoulli Ungleichung

\subsection{Bernoulli Ungleichung}

\begin{center}
  $ \forall x \in \mathbf{R} \geq -1$ und $n \in \mathbf{N}: (1 + x)^{n} \geq 1 + nx$
\end{center}
% Exponentialfunktion

\subsection{Exponentialfunktion}

\begin{align*}
  exp(z) &= \lim_{n \rightarrow \infty} (1 + \frac{z}{n})^n
\end{align*}
Die reelle Exponentialfunktion $exp: \mathbf{R} \rightarrow ]0, \infty[$ ist streng monoton wachsend,
stetig und surjektiv.\\
Es gelten weiter folgende Rechenregeln:
\begin{enumerate}[(i)]
  \item $exp(x + y) = exp(x) * exp(y)$
  \item $x^a := exp(a * ln(x))$
  \item $x^0 = 1 \;\;\; \forall x \in \mathbf{R}$
  \item $exp(iz) = cos(z) + i*sin(z) \;\;\; \forall z \in \mathbf{C}$
  \item $exp(i*\frac{\pi}{2}) = i$
  \item $exp(i\pi) = -1$ und $exp(2\pi i) = 1$
  \item Für $a > 0$ ist $]0, +\infty[ \rightarrow ]0, +\infty[$ als $x \rightarrow x^a$ eine
  streng monoton wachsende stetige Bijektion
\end{enumerate}
Merke: $e^x$ entspricht $exp(x)$.
% Natürliche Logaritmus

\subsection{Natürliche Logaritmus}

Der natürliche Logaritmus wir als $ln: ]0, \infty[ \rightarrow \mathbf{R}$ bezeichnet
und ist eine streng monoton wachsende stetige funktion. Es gilt auch, dass
\begin{enumerate}[(i)]
  \item $ln(1) = 0$
  \item $ln(e) = 1$
  \item $ln(a * b) = ln(a) + ln(b)$
  \item $ln(a / b) = ln(a) - ln(b)$
  \item $ln(x^a) = a * ln(x)$
  \item $x^a * x^b = x^{a + b}$
  \item $(x^a)^b = x^{a * b}$
  \item $ln(1+x) = \sum_{n=1}^{\infty} \frac{(-1)^{n-1}}{n} x^n \;\;\; (-1 < x \leq 1)$
\end{enumerate}
% Faktorisierungs Lemma

\subsection{Faktorisierungs Lemma}

$$
  a^n - b^n = (a-b)(a^{n-1} + ba^{n-2} + \cdots + b^{n-2}a + b^{n-1})
$$
% Sinus Abschätzung

\subsection{Sinus Abschätzung}

Es gilt $|\sin(x)| \leq |x|$ mit folgendem Beweis:
\begin{align*}
  f(x) &= x - \sin(x), x \geq 0 \\
  f'(x) &= 1 - \cos(x) \geq 0
\end{align*}
Weil $f(0) = 0$, $f(x) \geq 0$ für $x > 0$. Dann $|\sin(x)| \leq |x|$ einfach. 
% Polynomiale Funktion

\subsection{Polynomiale Funktion}

Eine \textbf{Polynomiale Funktion} $P:\mathbf{R} \rightarrow \mathbf{R}$ ist eine Funktion
der Form $P(x) = a_nx^n + a_{n-1}x^{n-1} + \cdots + a_0$ wobei $a_0, a_1, \cdots, a_n \in \mathbf{R}$.
Falls $a_n \neq 0$, ist $n$ der Grad von $P$. 
% Kompaktes Intervall

\subsection{Kompaktes Intervall}

EIn Intervall $\subset \mathbf{R}$ ist kompakt, wenn es von der Form $\mathbf{I} = [a, b]$,
$a \leq b$ ist.
% Funktionenfolge

\subsection{Funktionenfolge}

Eine Funktionenfolge ist eine Abbildung:
\begin{align*}
  f:\mathbf{N} &\rightarrow \mathbf{R}^\mathbf{D} = \{f:\mathbf{D} \rightarrow \mathbf{R}\}\\
  n &\rightarrow f_n
\end{align*}
wobei $f_n: \mathbf{D} \rightarrow \mathbf{R}$ eine Funktion ist. Für jedes $x \in \mathbf{D}$
erhält man eine Folge $(f_n(x))_{n \geq 1}$ reeller Zahlen.
% Trigonometrische Funktionen


\subsection{Trigonometrische Funktionen}
\begin{align*}
\exp(x) &= \sumn \frac{x^n}{n!} & r &= \infty \\
\sin(x) &= \sumn (-1)^n \frac{x^{2n + 1}}{(2n + 1)!} & r &= \infty \\
\cos(x) &= \sumn (-1)^n \frac{x^{2n}}{(2n)!} & r &= \infty \\
\ln(x + 1) &= \sumk (-1)^{k+1} \frac{x^k}{k} & r &= 1
\end{align*}

\includegraphics[width=7cm]{taylor.png}
\includegraphics[width=6cm]{sincostan.png}\\
\includegraphics[width=6cm]{arcsinArccosArctan.png}\\
\includegraphics[width=6cm]{sinhCoshTanh.png}


\begin{enumerate}[(i)]
  \item $cos(z) = cos(-z)$
  \item $sin(-z) = -sin(z)$
  \item $cos^2(z) + sin^2(z) = 1 \;\;\; \forall z \in \mathbf{C}$
\end{enumerate}

% Häufungspunkt

\subsection{Häufungspunkt}

$x_0 \in \mathbf{R}$ ist ein \textbf{Häufungspunkt} der Menge $\mathbf{D}$,
falls $\forall \delta > 0 \;\;\; (]x_0 - \delta, x_0 + \delta[ \setminus \{x_0\}) \cap \mathbf{D} \neq \emptyset$
% Lokales Extremum

\subsection{Lokales Extremum}

Eine Funktion $f$ besitzt ein lokales Extremum in $x_0$ falls es entweder ein lokales Minimum oder lokales Maximum von $f$ ist.
% Lokales Minimum

\subsection{Lokales Minimum}

Die Funktion $f$ besitzt ein lokales Minimum in $x_0$ falls es $\delta > 0$ gibt mit:
$$f(x) \geq f(x_0) \;\;\; \forall x \in ]x_0 - \delta, x_0 + \delta[\; \cap \; \mathbf{D}$$
% Lokales Maximum

\subsection{Lokales Maximum}

Die Funktion $f$ besitzt ein lokales Maximum in $x_0$ falls es $\delta > 0$ gibt mit:
$$f(x) \leq f(x_0) \;\;\; \forall x \in ]x_0 - \delta, x_0 + \delta[\; \cap \; \mathbf{D}$$
% Kritische Stelle

\subsection{Kritische Stelle}

Eine \textbf{kritische Stelle} einer Funktion ist ein $x_0$ an der $f'(x_0)$ null
oder undefiniert ist.
Kurze Notiz am Rande, ein stationärer Punkt ist: \\
 $x \in \R$ mit $f'(x) = 0$
% Hyperbol Funktionen

\subsection{Hyperbol Funktionen}

\begin{enumerate}[(i)]
  \item $cosh(x) := \frac{e^x + e^{-x}}{2}: \mathbf{R} \rightarrow [1, \infty]$
  \item $sinh(x) := \frac{e^x - e^{-x}}{2}: \mathbf{R} \rightarrow \mathbf{R}$
  \item $tanh(x) := \frac{e^x - e^{-x}}{e^x + e^{-x}}: \mathbf{R} \rightarrow [-1, 1]$
\end{enumerate}
und es gilt $cosh^2(x) - sinh^2(x) = 1$



% Funktionen Verknüpfung
\subsection{Funktionen Verknüpfung}

$
x \mapsto (g \circ f)(x) := g(f(x))
$
%
%\subsection{Sin/Cos Werte}
%\begin{center}
%\includegraphics[scale=0.3]{values.png}
%\end{center}



\section{Trigonometrie}

\subsection{Regeln}
\subsubsection{Periodizität}
\begin{itemize}
 \item $\sin(\alpha + 2 \pi) = \sin(\alpha) \quad \cos(\alpha + 2 \pi) = \cos(\alpha)$
 \item $\tan(\alpha + \pi) = \tan(\alpha) \quad \cot(\alpha + \pi) = \cot(\alpha)$
\end{itemize}

\subsubsection{Parität}
\begin{itemize}
 \item $\sin(-\alpha) = - \sin(\alpha) \quad \cos(-\alpha) = \cos(\alpha)$
 \item $\tan(-\alpha) = - \tan(\alpha) \quad \cot(-\alpha) = - \cot(\alpha)$
\end{itemize}

\subsubsection{Ergänzung}
\begin{itemize}
 \item $\sin(\pi - \alpha) = \sin(\alpha) \quad \cos(\pi - \alpha) = - \cos(\alpha)$
 \item $\tan(\pi - \alpha) = -\tan(\alpha) \quad \cot(\pi - \alpha) = - \cot(\alpha)$
\end{itemize}


\subsubsection{Komplemente}
\begin{itemize}
 \item $\sin(\pi/2 - \alpha) = \cos(\alpha) \quad \cos(\pi/2 - \alpha) = \sin(\alpha)$
 \item $\tan(\pi/2 - \alpha) = -\tan(\alpha) \quad \cot(\pi/2 - \alpha) = -\cot(\alpha)$
\end{itemize}

\subsubsection{Doppelwinkel}
\begin{itemize}
 \item $\sin(2\alpha) = 2 \sin(\alpha) \cos(\alpha)$
 \item $\cos(2\alpha) = \cos^2(\alpha) - \sin^2(\alpha) = 1 - 2 \sin^2(\alpha)$
 \item $\tan(2\alpha) = \frac{2\tan(\alpha)}{1 - \tan^2(\alpha)}$
\end{itemize}

\subsubsection{Addition}
\begin{itemize}
 \item $\sin(\alpha + \beta) = \sin(\alpha) \cos(\beta) + \cos(\alpha) \sin(\beta)$
 \item $\cos(\alpha + \beta) = \cos(\alpha) \cos(\beta) - \sin(\alpha) \sin(\beta)$
 \item $\tan(\alpha + \beta) = \frac{\tan(\alpha) + \tan(\beta)}{1 - \tan(\alpha) \tan(\beta)}$
\end{itemize}

\subsubsection{Subtraktion}
\begin{itemize}
 \item $\sin(\alpha - \beta) = \sin(\alpha) \cos(\beta) - \cos(\alpha)\sin(\beta)$
 \item $\cos(\alpha - \beta) = \cos(\alpha) \cos(\beta) + \sin(\alpha)\sin(\beta)$
 \item $\tan(\alpha - \beta) = \frac{\tan(\alpha) - \tan(\beta)}{1+\tan(\alpha) \tan(\beta)}$
\end{itemize}

\subsubsection{Multiplikation}
\begin{itemize}
 \item $\sin(\alpha) \sin(\beta) = -\frac{\cos(\alpha + \beta) - \cos(\alpha - \beta)}{2}$
 \item $\cos(\alpha) \cos(\beta) =  \frac{\cos(\alpha + \beta) + \cos(\alpha - \beta)}{2}$
 \item $\sin(\alpha) \cos(\beta) =  \frac{\sin(\alpha + \beta) + \sin(\alpha - \beta)}{2}$
\end{itemize}

\subsubsection{Potenzen}
\begin{itemize}
 \item $\sin^2(\alpha) = \frac{1}{2}(1-\cos(2\alpha))$
 \item $\cos^2(\alpha) = \frac{1}{2}(1+\cos(2\alpha))$
 \item $\tan^2(\alpha) = \frac{1-\cos(2\alpha)}{1+\cos(2\alpha)}$
\end{itemize}

\subsubsection{Diverse}

\begin{itemize}
 \item $\sin^2(\alpha) + \cos^2(\alpha) = 1$
 \item $\cosh^2(\alpha) - \sinh^2(\alpha) = 1$
 \item $\sin(z) = \frac{e^{iz} - e^{-iz}}{2}$ und $\cos(z) = \frac{e^{iz} + e^{-iz}}{2}$
 \item $\tan(x) = \frac{\sin(x)}{\cos(x)} \;\;\; \forall z \not \in \{\frac{\pi}{2} + \pi k\}$
 \item $\cot(x) = \frac{\cos(x)}{\sin(x)}$
 \item $\arcsin(x) = \sin(x)\cos(x)$
 \item $\cos(\arccos(x)) = x$
 \item $\sin(\arccos(x)) = \frac{1}{\sqrt{1 - x^2}}$
 \item $\sin(\arctan(x)) = \frac{x}{\sqrt{x^2 + 1}}$
 \item $\cos(\arctan(x)) = \frac{1}{\sqrt{x^2 + 1}}$
 \item $\sin(x) = \frac{\tan(x)}{\sqrt{1 + \tan(x)^2}}$
 \item $\cos(x) = \frac{1}{\sqrt{1 + \tan(x)^2}}$
 %\item $\sin(z) := z - \frac{z^3}{3!} + \frac{z^5}{5!} - \frac{z^7}{7!} + \cdots = %\sum_{n = 0}^\infty \frac{(-1)^n z^{2n + 1}}{(2n+1)!}$
 % \item $\cos(z) := 1 - \frac{z^2}{2!} + \frac{z^4}{4!} - \frac{z^6}{6!} + \cdots = \sum_{n = 0}^\infty \frac{(-1)^n z^{2n}}{(2n)!}$

\end{itemize}


\newpage

\section{Exercises}
\subsection{Multiple Choice}
\begin{itemize}
 \item The composition of continuous functions is continuous
 \item Falls $g(x) = f(x)^2$ differenzierbar ist, dann ist $f$ nicht unbedingt differenzierbar
 \item $cos(x)$ gerade, $sin(x)$ ungerade $\rightarrow$ hilfreich wenn Integral von $\int_{-y}^{y}$. Bei ungeraden kürzt sich es weg, bei geraden kann man $2\int_{0}^{y}$
 \item Stetigkeitspunkte: Zuerst Schnittpunkte finden, dann zeigen, dass Punkt $x_0$ stetig ist
 \item $f, g : [0, 1] \rightarrow [0, 1]$. 
\begin{itemize}
 \item Falls $f, g$ injektiv, dann $f \circ g$ injektiv
 \item Falls f, g surjektiv, $f \circ g$ nicht unbedingt surjektiv
 \item $f \circ g \neq g \circ f$
\end{itemize}
 \item $f, g$ Funktionen
\begin{itemize}
 \item $f \circ g$ stetig $f, g$ nicht unbedingt stetig
 \item Nicht für jede Folge mit Grenzwert $x$ gilt $\lim_{n \to  \infty} f(x_n) = f(x)$
\end{itemize}
 \item $x^x = e ^{x\log(x)}$
 \item falls $\sum_{n = 1}^{\infty} a_n$ konvergiert, $c_n ) (-1)^n a_n$ konvergiert gegen 0
 \item $f: [0, 1] \rightarrow \mathbf{R}$ stetig, integrierbare Funktion, $a_n = \int_{0}^{1} f(x^2) dx$, Falls $f$ monoton wachsend ist, so ist $a_k$ nicht unbedingt monoton wachsend.
 \item $f_k$ Folge von dfferenzierbaren Funktionen auf $[0, 1]$. Sei $f$ eine Funktion auf $[0, 1]$ definiert. $f_k$ konvergiert gleichmässig zu f für $k \rightarrow \infty$. $f$ ist beschränkt
\begin{itemize}
 \item $f_k$ sind alle differenzierbar und daher stetig
 \item $f$ ist auch stetig, weil $f_k$ zu $f$ gleichmässig konvergiert
 \item beschränkt, weil es stetig auf kompakten Intervall definierte Funktion ist
 \item \color{red} Nicht stetige Funktionen, auch wenn sie nur auf $[0, 1]$ definiert sind, können unbeschränkt sein
\end{itemize}
 \item $f: \mathbf{R} \rightarrow \mathbf{R}$ beliebig oft stetig differenzierbare Funktion
\begin{itemize}
 \item $f$ hat Taylorreihe bei $x_0 = 0$
 \item Der Konvergenzradius der Taylorreihe ist $\geq0$ aber nicht notwendigerweise $>0$
 \item Wenn $f$ durch eine Potenzreihe gegeben ist, so ist dieser gleich der Taylorreihe \color{red}
 \item  dort, wo die Taylorreihe konvertiert, stellt sie nicht unbedingt die Funktion dar
\end{itemize}
 \item Sei $\sum_{k = 1}^{\infty} a_k$ eine Reihe
\begin{itemize}
\color{red}
 \item Falls $\forall \epsilon > 0 \, \exists N \geq 1$ so dass $\sum_{k = n}^{n + 100} |a_k| < \epsilon \forall n \geq N $ dann ist die Reihe $\sum_{k = 1}^{\infty}$ nicht unbedingt konvergent (Gegenbeispiel: $\frac{1}{n})$ \color{black}
 \item  Falls $\sum_{k = 1}^{\infty}$ konvergiert, so folgt $\forall \epsilon > 0 \, \exists N \geq1$ so dass  $\sum_{k = n}^{n + 100} |a_k| < \epsilon \, \forall n \geq N$
\end{itemize}
 \item Sei $f: \mathbf{R} \rightarrow [0, \infty[$ so dass $lim_{x \to 0} f(x) \neq 0$
\begin{itemize}
 \item es exisitiert eine Folge $(x_n)$ mit $lim_{n \to \infty} x_n = 0$ und ein $\epsilon$ so dass $|f(x_n)| > \epsilon, n \exists \mathbf{N}$
 \item For alle $x \exists \mathbf{R}$ gilt $f(x) > 0$ \color{red}
 \item  $\forall \epsilon > 0, \exists  \delta > 0$ sodass $0 < |x| < \delta \rightarrow f(x) > \epsilon$ STIMMT NICHT
 \item Für jede Folge $(x_n)$ mit $lim_ {n \to \infty} x_n = 0$ gilt $lim_{n \to \infty} f(x_n) \neq 0$ STIMMT NICHT
\end{itemize}
 \item Seien $f : X  \rightarrow Y, g: Y \rightarrow Z$ Funktionen, so dass $g \circ f: X \rightarrow Z$ eine Bijektion ist: f ist injektiv, g ist surjektiv
 \item differenzierbar $\rightarrow$ stetig $\rightarrow$ integrierbar
 \item Sei $a, b \in \mathbf{R}$ mit $a < b$ und $f: [a, b] \rightarrow \mathbf{R}$ eine Funktion. Sei $f_n: [a, b] \rightarrow \mathbf{R}, n \geq 1$ So dass die Funktionenfolge $(f_n)_{n \geq 1}$ auf $[a, b]$ gleichmässig gegen $f$ konvergiert
\begin{itemize}
 \item Sei $x_0 \in ]a, b[$ Falls $f_n$ für alle $n \geq 1 in x_0$ differenzierbar ist, so ist $f$ nicht unbedingt diferenzierbar. (Im Allgemeinen braucht die Grenzfunktion nicht einmal differenzierbar zu sein, und wenn sie es ist, muss ihre Ableitung keineswegs geich dem Grenzwert der Ableitung der Folge sein.
\end{itemize}
 \item $f: [0, 1] \rightarrow [0, 1]$ stetig und nicht konstant
\begin{itemize}
 \item Das Bild $f([0, 1]) \subset [0, 1]$ ist ein abgeschossenes Intervall. D.h. es gibt $a, b  \in [0, 1]$ mit $a < b$, so dass $f([0, 1]) = [a, b]$
\end{itemize}
 \item $f: \mathbf{R} \rightarrow \mathbf{R}$ stetig bei $x_0 = 0$, mit $f(x_0) > 0$
\begin{itemize}
 \item Es existiert $\epsilon , \delta > 0$ so dass $f(x) > \epsilon$ für alle $x \in (-\delta, \delta)$ gilt
\end{itemize}
 \item $a < b, g: \mathbf{R} \rightarrow \mathbf{R}$ beschränkt und $f: [a, b] \rightarrow \mathbf{R}$ beschränkt mit $f(a) < f(b)$
\begin{itemize}
  \item Falls $f$ stetig ist, gibt es $x_0 \in [a, b]$, so dass $\int_{a}^{b}xf(x)dx = \frac{f(x_0)}{2} (b^2 -a^2)$
\end{itemize}
 \item Sei $f$ eine ungerade Funktion dann ist $f^{(i)} (0) = 0$ für $i$ gerade
 \item Sei $\phi$ eine Abbildung einer Reihe $\sum_{k = 1}^{\infty} a_k$ und $b_n = a_{\phi_{(n)}}$
\begin{itemize}
 \item Wenn die Reihe absolut konvergiert und $\phi$ injektiv ist, dann ist die Reihe mit $b_n$ auch konvergent (Wenn nicht absolut konvergent, dann kann man jeden möglichen Wert bekommen, Surjektiv funktioniert nicht, Annahme $a_n = \frac{1}{n^2}$)
\end{itemize}
\end{itemize}
\begin{center}
\includegraphics[width=7.5cm]{proof2.png} \\
\includegraphics[width=7.5cm]{proof3.png} \\
\includegraphics[width=7.5cm]{proof4.png} \\
\includegraphics[width=7.5cm]{proof5.png} \\

\end{center}
\newpage
\section{Tabellen}
\subsection{Grenzwerte}
\begin{center}
\includegraphics[width=8cm]{typischeReihen.png}
\end{center}
\begin{center}
  \begin{tabularx}{\linewidth}{XX}

    $\limxi \frac{1}{x} = 0$ & $\limxi 1 + \frac{1}{x} = 1$ \\
    $\limxi e^x = \infty$ & $\limxn e^x = 0$ \\
    $\limxi e^{-x} = 0$ & $\limxn e^{-x} = \infty$ \\
    $\limxi \frac{e^x}{x^m} = \infty$ & $\limxn xe^x = 0$ \\
    $\limxi \ln(x) = \infty$ & $\limxo \ln(x) = -\infty$ \\
    $\limxi (1+x)^{\frac{1}{x}} = 1$ & $\limxo (1+x)^{\frac{1}{x}} = e$ \\
    $\limxi (1+\frac{1}{x})^b = 1$ & $\limxi n^{\frac{1}{n}} = 1$ \\
    $\lim_{x\to\pm\infty} (1 + \frac{1}{x})^x = e$ & $\limxi (1-\frac{1}{x})^x = \frac{1}{e}$ \\
    $\lim_{x\to\pm\infty} (1 + \frac{k}{x})^{mx} = e^{km}$ & $\limxi (\frac{x}{x+k})^x = e^{-k}$ \\
    $\limxo \frac{a^x -1}{x} = \ln(a), \newline \forall a > 0$ &
    $\limxi x^a q^x = 0, \newline \forall 0 \le q < 1$ \\
  \end{tabularx}
  \begin{tabularx}{\linewidth}{XX}
    $\limxo \frac{\sin x}{x} = 1$ & $\limxo \frac{\sin kx}{x} = k$\\
    $\limxo \frac{1}{\cos x} = 1$ & $\limxo \frac{\cos x -1}{x} = 0$ \\
    $\limxo \frac{\log 1 - x}{x} = -1$ & $\limxo x \log x = 0$\\
    $\limxo \frac{1 - \cos x}{x^2} = \frac{1}{2}$ & $\limxo \frac{e^x-1}{x} = 1$ \\
    $\limxo \frac{x}{\arctan x} = 1$ & $\limxi \arctan x = \frac{\pi}{2}$ \\
    $\limxo \frac{e^{ax}-1}{x} = a$ & $\limxo \frac{\ln(x+1)}{x} = 1$ \\
    $\lim_{x\to 1} \frac{\ln(x)}{x-1} = 1$ & $\limxi \frac{\log(x)}{x^a} = 0$ \\
    $\limxi \sqrt[x]{x} = 1$ & $\limxi \frac{2x}{2^x} = 0$ \\
   % \bottomrule
  \end{tabularx}
\end{center}



\subsection{Ableitungen}

\begin{center}
  % the c>{\centering\arraybackslash}X is a workaround to have a column fill up all space and still be centered
  \begin{tabularx}{\linewidth}{c>{\centering\arraybackslash}Xc}

  
  $\mathbf{F(x)}$ & $\mathbf{f(x)}$ & $\mathbf{f'(x)}$ \\
  %\midrule
  $(x-1)e^x $ & $xe^x$ & $(x+1)e^x$ \\ 
  $\frac{x^{-a+1}}{-a+1}$ & $\frac{1}{x^a}$ & $\frac{a}{x^{a+1}}$ \\
  $\frac{x^{a+1}}{a+1}$ & $x^a \ (a \ne -1)$ & $a \cdot x^{a-1}$ \\
  $\frac{1}{k \ln(a)}a^{kx}$ & $a^{kx}$ & $ka^{kx} \ln(a)$ \\
  $\ln |x|$ & $\frac{1}{x}$ & $-\frac{1}{x^2}$ \\
  $\frac{2}{3}x^{3/2}$ & $\sqrt{x}$ & $\frac{1}{2\sqrt{x}}$\\
  $-\cos(x)$ & $\sin(x)$ & $\cos(x)$ \\
  $ $ & $\frac{\sin(x)^2}{2} $ & $\sin(x)\cos(x)$ \\ 
  $\sin(x)$ & $\cos(x)$ & $-\sin(x)$ \\
  $\frac{1}{2}(x-\frac{1}{2}\sin(2x))$ & $\sin^2(x)$ & $2 \sin(x)\cos(x)$ \\
  $\tan(x) - x$ & $\tan(x)^2$ & $2\sec(x)^2 \tan(x)$\\
  $-\cot(x) - x$ & $\cot(x)^2$ & $-2 \cot(x) \csc(x)^2$\\
  $\frac{1}{2}(x + \frac{1}{2}\sin(2x))$ & $\cos^2(x)$ & $-2\sin(x)\cos(x)$ \\
  \multirow{2}*{$-\ln|\cos(x)|$} & \multirow{2}*{$\tan(x)$} & $\frac{1}{\cos^2(x)}$  \\
  & & $1 + \tan^2(x)$ \\
  $\cosh(x)$ & $\sinh(x)$ & $\cosh(x)$ \\
  $\log(\cosh(x))$ & $\tanh(x)$ & $\frac{1}{\cosh^2(x)}$ \\
  $\ln | \sin(x)|$ & $\cot(x)$ & $-\frac{1}{\sin^2(x)}$ \\
  $\frac{1}{c} \cdot e^{cx}$ & $e^{cx}$ & $c \cdot e^{cx}$ \\
  $x(\ln |x| - 1)$ & $\ln |x|$ & $\frac{1}{x}$ \\
  $\frac{1}{2}(\ln(x))^2$ & $\frac{\ln(x)}{x}$ & $\frac{1 - \ln(x)}{x^2}$ \\
  $\frac{x}{\ln(a)} (\ln|x| -1)$ & $\log_a |x|$ & $\frac{1}{\ln(a)x}$ \\

  %\bottomrule
  \end{tabularx}
\end{center}

%\subsection{Weitere Ableitungen}
\begin{center}
  \begin{tabularx}{\linewidth}{>{\centering\arraybackslash}X>{\centering\arraybackslash}X}
  
  $\mathbf{F(x)}$ & $\mathbf{f(x)}$ \\
  \midrule
  $\arcsin(x) / \arccos(x)$ & $\frac{1 / -1}{\sqrt{1 - x^2}}$ \\
  $\arctan(x)$ & $\frac{1}{1 + x^2}$ \\ 

  $x\arcsin(x) + \sqrt{1 - x^2}$ & $\arcsin(x)$\\
  $x\arccos(x) - \sqrt{1 - x^2}$ & $\arccos(x)$\\
  $x\arctan(x) - \frac{1}{2}\ln(1+x^2)$ & $\arctan(x)$\\
  $\ln(\cosh(x))$ & $\tanh(x)$\\

   
  $x^x \ (x > 0)$ & $x^x \cdot (1 + \ln{x})$ \\
$f(x)^{g(x)}$ & $e^{g(x) ln(f(x))}$\\
$f(x) = cos(\alpha)$ & $f(x)^n = sin(x + n\frac{\pi}{2})$\\
$f(x) = \frac{1}{ax + b}$ & $f(x)^n = (-1)^n * a^n * n! * (ax + b)^{-n+1}$\\
  $-\ln(\cos(x))$ & $\tan(x)$\\
  $\ln(\sin(x))$ & $\cot(x)$\\
  $\ln(\tan(\frac{x}{2}))$ & $\frac{1}{\sin(x))}$\\
  $\ln{(\tan(\frac{x}{2} + \frac{\pi}{4})}$ & $\frac{1}{cos(x)}$\\

  \bottomrule
  \end{tabularx}
\end{center}
%\subsection{Integrale}
\begin{center}
 \begin{tabularx}{\linewidth}{>{\centering\arraybackslash}X>{\centering\arraybackslash}X}
  
  $\mathbf{f(x)}$ & $\mathbf{F(x)}$ \\
  \midrule
  $\int f'(x) f(x) \dx$ & $\frac{1}{2}(f(x))^2$ \\
  $\int \frac{f'(x)}{f(x)} \dx$ & $\ln|f(x)|$ \\
  $\int_{-\infty}^\infty e^{-x^2} \dx$ & $\sqrt{\pi}$ \\
  $\int (ax+b)^n \dx$ & $\frac{1}{a(n+1)}(ax+b)^{n+1}$ \\
  $\int x(ax+b)^n \dx$ & $\frac{(ax+b)^{n+2}}{(n+2)a^2} - \frac{b(ax+b)^{n+1}}{(n+1)a^2}$ \\
  $\int (ax^p+b)^n x^{p-1} \dx$ & $\frac{(ax^p+b)^{n+1}}{ap(n+1)}$ \\
  $\int (ax^p + b)^{-1} x^{p-1} \dx$ & $\frac{1}{ap} \ln |ax^p + b|$ \\
  $\int \frac{ax+b}{cx+d} \dx$ & $\frac{ax}{c} - \frac{ad-bc}{c^2} \ln |cx +d|$ \\
  $\int \frac{1}{x^2+a^2} \dx$ & $\frac{1}{a} \arctan \frac{x}{a}$ \\
  $\int \frac{1}{x^2 - a^2} \dx$ & $\frac{1}{2a} \ln\left| \frac{x-a}{x+a} \right|$ \\
  $\int \sqrt{a^2+x^2} \dx $ & $\frac{x}{2}f(x) + \frac{a^2}{2}\ln(x+f(x))$ \\
  \bottomrule
 \end{tabularx}
\end{center}

\end{multicols*}
\end{document}