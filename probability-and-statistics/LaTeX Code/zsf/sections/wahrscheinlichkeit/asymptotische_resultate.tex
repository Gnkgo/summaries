\section{Asymptotische Resultate}%
\label{sec:asymptotische_resultate}

In diesem Kapitel ist $(\Omega, \F , \pr)$ und eine unendliche Folge von \iid ZV $X_1, X_2, \ldots$ fixiert.
Also gegeben ein ZV $X_i : \Omega \rightarrow \R$ sodass
\begin{equation*}
	\begin{matrix}
		\forall i_1 < \ldots < i_k\\
		\forall x_1 , \ldots , x_k \in \R
	\end{matrix}
	\quad \pr [X_{i_1} \leq x_1, \ldots , X_{x_k} \leq x_k] = F(x_1) \cdot \ldots \cdot F(x_k)
\end{equation*}
Der \emph{Empirische Durchschnitt} ist definiert durch
\begin{equation*}
	\frac{S_n}{n} = \frac{X_1 + \ldots + X_n}{n} 
\end{equation*}


\subsection{Gesetz der grossen Zahlen}%
\label{sub:gesetz_der_grossen_zahlen}

\begin{tcolorbox}[lemmacore]
	Sei $\E [|X_1|] < \infty$. Falls wir $m = \E [X_1]$ definieren folgt
	\begin{equation*}
		\lim_{n \rightarrow \infty} \frac{X_1 + \ldots  + X_n}{n} = m \quad\text{fast sicher}
	\end{equation*}
	Da die ZV \iid macht es keinen Unterschied, dass das Theorem nur durch $X_1$ definiert ist.
\end{tcolorbox}
\begin{itemize}
	\item Falls $X_1 , X_2, \ldots$ \iid $\bern (p)$ 
		\begin{equation*}
			\lim_{n \rightarrow \infty} \frac{X_1 + \ldots  + X_n}{n} = p \quad\text{fast sicher}
		\end{equation*}
	\item Falls $T_1 , T_2, \ldots$ \iid $\expd (\lambda)$ 
		\begin{equation*}
			\lim_{n \rightarrow \infty} \frac{T_1 + \ldots  + T_n}{n} = \lambda \quad\text{fast sicher}
		\end{equation*}
\end{itemize}
\begin{tcolorbox}[lemmacore]
	Sei $X_1 , X_2 , \ldots$ eine Folge von \iid ZV mit $\E[|X_1|] < \infty$. Dann
	\begin{equation*}
		\lim_{n \rightarrow \infty} \frac{1}{n} \sum_{i=1}^{n} X_i = \E[X_1]
	\end{equation*}
	fast sicher.
\end{tcolorbox}


\subsection{Anwendung: Monte-Carlo Integration}%
\label{sub:anwendung_monte_carlo_integration}



Das Gesetz der grossen Zahlen kann verwendet werden, um komplizierte Integrale zu approximieren.\\
Sei $g : [0,1] \rightarrow \R$ sodass $\int_{0}^{1} |g(x)| \dif x < \infty$. Das Ziel ist es $I = \int_{0}^{1} g(x) \dif
x$ zu berechnen.\\
Um $I$ zu approximieren, interpretiere es als Erwartungswert. Sei $U$ eine normalverteilte ZV in $[0,1]$. Dann ist
\begin{equation*}
	\E [g(U)] = \int_{0}^{1} g(x) \dif x = I
\end{equation*}
Sei $X_1 , X_2 , \ldots$ \iid sodass $\forall n ~ X_n = g(U_n)$ wobei $U_1 , U2 , \ldots$ eine Folge von \iid
normalverteilten ZV in $[0,1]$ ist. Dann ist
\begin{equation*}
	\E [|X_1|] = \int_{0}^{1} |g(x)| \dif x < \infty
\end{equation*}
und $\E [X_1] = I$. Mit dem Gesetz der grossen Zahlen folgt
\begin{equation*}
	\lim_{n \rightarrow \infty} \frac{g(U_1) + \ldots + g (U_n)}{n} = I
\end{equation*}


\subsection{Konvergenz der Verteilung}%
\label{sub:konvergenz_der_verteilung}

Zwei ZV $X$ und $Y$ haben ähnliche wahrscheinlichkeitstechnische Eigenschaften wenn ihre Verteilungsfunktionen $F_X$ und
$F_Y$ nahe beieinander sind.
\begin{definition}{Konvergenz der Verteilung}
	Seien $(X_n)_{n \in \N}$ und $X$ ZVs. 
	\begin{equation*}
		X_n \overset{\text{Approx}}{\approx} X ~\text{wenn} ~ n \rightarrow \infty
	\end{equation*}
	falls für jedes $x \in \R$
	\begin{equation*}
		\lim_{n \rightarrow \infty} \pr [X_n \leq x] = \pr [X \leq x]
	\end{equation*}
\end{definition}
\subsection{Grenzwertsätze}%
\label{sub:grenzwertsaetze}

\begin{center}
	Wie weit ist $\frac{X_1 + \ldots + X_n}{n}$ typischerweise von $m$ entfernt?
\end{center}

\subsubsection{Fluktuation von Normalverteilten ZV}%
\label{ssub:fluktuation_von_normalverteilten_zv}
 
Sei $Z = \frac{X_1 + \ldots + X_n}{n} -m$ eine ZV. $Z \sim \normd \left( \overline{m} = 0, \overline{\sigma}^2 =
\frac{1}{n} \sigma^2 \right)$. Die Standardabweichung $\overline{\sigma} =
\frac{1}{\sqrt{n}} \sigma$ repräsentiert die typischen Abweichungen von $Z$.\\
Grob gesagt ist die Abweichung zwischen $\frac{X_1 + \ldots + X_n}{n}$ und $m$ von Ordnung $\frac{\sigma}{\sqrt{n}}$.\\
Um Abweichungen der Ordnung $1$ zu bekommen, skalieren wir $Z$ um einen Faktor $\frac{\sqrt{n}}{\sigma}$ (immer noch
normalverteilt) $S_n = \sum_{i=1}^{n} X_i$
\begin{equation*}
	\frac{\sqrt{n}}{\sigma} Z = \frac{X_1 + \ldots + X_n - n \cdot m}{\sqrt{\sigma^2 n}} = \frac{S_n - n \cdot
	m}{\sqrt{\sigma^2 n}} 
\end{equation*}
\hrule

\begin{definition}{Zentraler Grenzwertsatz}
	Sei $\E [X_1^2]$ wohldefiniert und endlich. Für \\$m = \E [X_1]$ und $\sigma^2 = \Var (X_1)$ folgt
	\begin{equation*}
		\pr \left[ \frac{S_n - n \cdot m}{\sqrt{\sigma^2 n}} \leq a \right] \xrightarrow[n \rightarrow \infty]{} \Phi (a) =
		\frac{1}{\sqrt{2 \pi}} \int_{- \infty}^{a} e^{-x^2/2} \dif x 
	\end{equation*}
	für jedes $a \in \R$ $\Phi (-a) = 1- \Phi(a)$.
\end{definition}
\begin{tcolorbox}[lemmacore]
	Für grosse $n$ sieht die Verteilung von 
	\begin{equation*}
		Z_n = \frac{S_n - n \cdot m}{\sqrt{\sigma^2 n}}
	\end{equation*}
	wie $\normd (0,1)$ aus. Mit $Z_n \approx Z$ für $n \rightarrow \infty$ ist $Z \sim \normd (0,1)$.
\end{tcolorbox}
